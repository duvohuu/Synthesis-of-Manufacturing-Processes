\chapter{CÁC QUÁ TRÌNH GIA CÔNG KHÔNG TRUYỀN THỐNG VÀ CẮT BẰNG NĂNG LƯỢNG NHIỆT}
     \section{Định nghĩa các quá trình gia công không truyền thống}
          \hspace{0.6cm}Là một nhóm các quy trình loại bỏ vật liệu thừa bằng các kỹ thuật liên quan đến năng lượng cơ học, nhiệt, điện hoặc hóa học (hoặc sự kết hợp của các loại năng lượng này). Nhóm các quá trình gia công này không sử dụng các dụng cụ cắt gọt như các phương pháp truyền thống. Được phát triển từ sau thế chiến II nhằm đáp ứng các yêu cầu gia công mà các phương pháp gia công truyền thống không đáp ứng được.
          \section{Tầm quan trọng của các phương pháp gia công không truyền thống}
          \hspace*{0.6cm}Gia công các kim loại và phi kim loại mới được phát triển có tính chất đặc biệt khiến chúng khó hoặc không thể gia công bằng các phương pháp thông thường.
          \newline
          \hspace*{0.6cm}Gia công hình dạng chi tiết phức tạp mà không thể dễ dàng thực hiện bằng phương pháp gia công thông thường
          \newline
          \hspace*{0.6cm}Tránh làm hư hỏng bề mặt thường xảy ra trong quá trình gia công cơ khí thông thường.
     \section{Phân loại}
          \begin{itemize}
               \item Cơ học (Mechanical): làm xói mòn vật liệu cần gia công thông qua một dòng hạt mài hoặc chất lỏng (hoặc cả hai) ở vận tốc cao.
               \item Điện (Electrical): Sử dụng năng lượng điện hóa để bóc tách vật liệu (ngược lại với quy trình mạ điện).
               \item Nhiệt (Thermal): Năng lượng nhiệt được áp dụng lên một phần nhỏ bề mặt chi tiết, làm cho phần đó bị nóng chảy và/hoặc bay hơi.
               \item Hóa học (Chemical): Các chất ăn mòn hóa học loại bỏ vật liệu một cách có chọn lọc từ các phần của chi tiết gia công, trong khi các phần khác được bảo vệ bởi một lớp che chắn (mặt nạ/mask).
          \end{itemize}
     \section{Các quá trình năng lượng cơ học (Mechanical Energy Proceses)}
          \hspace*{0.6cm}Các quá trình gia công không truyền thống bằng cơ học có các loại cụ thể:
          \begin{itemize}
               \item Ultrasonic Machining (USM)
               \item Water Jet Cutting (WJC)
               \item Abrasive Water Jet Machining (WJM)
               \item Abrasive Jet Machining (AJM)
               \item Abrasive Flow Machining (AFM)
          \end{itemize}
          \hspace*{0.6cm}Cụ thể
          \subsection{Ultrasonic Machining (USM)}
               \subsubsection{Định nghĩa}
                    \hspace*{0.6cm}Các hạt mài được chứa trong huyền phù (slurry) được gia tốc đến vận tốc cao và va đập vào bề mặt chi tiết gia công nhờ dụng cụ dao động với biên độ nhỏ và tần số cao.
                    \newline
                    \hspace*{0.6cm}Phương pháp thực hiện
                    \begin{itemize}
                         \item Chuyển động dao động của dụng cụ vuông góc với bề mặt chi tiết gia công.
                         \item Các hạt mài thực hiện quá trình bóc tách vật liệu
                         \item Dụng cụ được tiến dao chậm vào chi tiết
                         \item Hình dạng của dụng cụ được sao chép (tạo hình) lên chi tiết gia công
                    \end{itemize}
                    \begin{figure}[H]
                         \centering
                         \includegraphics[width=0.6\textwidth]{pictures/chapter7/c7_p01.png}
                         \caption{Sơ đồ quá trình gia công siêu âm (Ultrasonic Machining - USM)}
                    \end{figure}
               \subsubsection{Ứng dụng của USM}
                    \hspace*{0.6cm}Ứng dụng của USM bao gồm:
                    \begin{itemize}
                         \item Vật liệu gia công cứng, giòn như gốm, thủy tinh và cacbit.
                         \item Cũng gia công hiệu quả đối với một số kim loại, chẳng hạn thép không gỉ và titan
                         \item Các dạng hình học gia công bao gồm lỗ không tròn và lỗ có trục cong
                         \item Nguyên công “dập tạo hình bề mặt (coining)” – hoa văn/mẫu trên dụng cụ được in/ép truyền lên bề mặt phẳng của chi tiết gia công.
                    \end{itemize}
          \subsection{Water Jet Cutting (WJC)}
                    \subsubsection{Định nghĩa}
                         \hspace*{0.6cm}Sử dụng dòng nước áp suất cao, tốc độ cao hướng vào bề mặt làm việc để cắt.
                         \begin{figure}[H]
                              \centering
                              \includegraphics[width=0.4\textwidth]{pictures/chapter7/c7_p02.png}
                              \caption{Sơ đồ quá trình cắt bằng tia nước (Water Jet Cutting - WJC)}
                         \end{figure}
                    \subsubsection{Ứng dụng của WJC}
                         \hspace*{0.6cm}Ứng dụng của WJC bao gồm:
                         \begin{itemize}
                              \item Thường được tự động hóa bằng CNC hoặc robot công nghiệp để điều khiển đầu phun chuyển động theo quỹ đạo mong muốn
                              \item Dùng để cắt các rãnh/khe hẹp trên vật liệu dạng tấm phẳng như nhựa, dệt may, vật liệu composite, gạch lát sàn, thảm, da và bìa cứng
                              \item Không phù hợp với vật liệu giòn, ví dụ thủy tinh
                         \end{itemize}
                    \subsubsection{Ưu điểm của WJC}
                         \begin{itemize}
                              \item Không gây nghiền nát hoặc cháy xém bề mặt chi tiết gia công.
                              \item Lượng vật liệu hao hụt rất ít (ít nhất - tối thiểu).
                              \item Không gây ô nhiễm môi trường.
                              \item Dễ dàng tự động hóa.
                         \end{itemize}
          \subsection{Abrasive Water Jet Cutting (AWJC)} 
               \hspace*{0.6cm}Là phương pháp kết hơp khi gia công kim loại bằng tia nước (WJC), thường phải bổ sung các hạt mài vào dòng tia.
               \newline
               \hspace*{0.6cm}Các thông số công nghệ bổ sung bao gồm: loại hạt mài, cỡ hạt (grit size) và lưu lượng hạt mài.
               \begin{itemize}
                    \item Hạt mài thường dùng: oxit nhôm ($Al_2O_3$), silic dioxit ($SiO_2$) và garnet (khoáng silicat)
                    \item Cỡ hạt mài nằm trong khoảng 60 đến 120.
                    \item Hạt mài được trộn vào dòng nước với lưu lượng khoảng 0.25 kg/phút (0.5 lb/phút) sau khi tia nước ra khỏi vòi phun.
               \end{itemize}
          \subsection{Abrasive Jet Machining (AJM)}
               \subsubsection{Định nghĩa}
                    \hspace*{0.6cm}Gia công sử dụng luồng khí tốc độ cao chứa các hạt mài mòn nhỏ.
                    \begin{figure}[H]
                         \centering
                         \includegraphics[width=0.5\textwidth]{pictures/chapter7/c7_p03.png}
                         \caption{Sơ đồ quá trình gia công bằng tia mài mòn (Abrasive Jet Machining - AJM)}
                    \end{figure}
               \subsubsection{Ứng dụng của AJM}
                    \hspace*{0.6cm}Ứng dụng của AJM bao gồm
                    \begin{itemize}
                         \item Thường được thực hiện thủ công bởi người vận hành, do con người trực tiếp định hướng đầu phun.
                         \item Chủ yếu được sử dụng như một quá trình hoàn thiện, không phải quá trình cắt.
                         \item Các ứng dụng: tẩy ba via, cắt tỉa và loại bỏ ba via đúc (deflashing), làm sạch và đánh bóng.
                         \item Vật liệu gia công: vật liệu dạng tấm mỏng, phẳng, có tính cứng và giòn (ví dụ: thủy tinh, silic, mica, gốm).
                    \end{itemize}
     \section{Các quá trình gia công điện hóa (Electrochemical Machining Processes)}
          \subsection{Tổng quan}
               \hspace*{0.6cm}Là một nhóm các quá trình gia công trong đó năng lượng điện được sử dụng kết hợp với các phản ứng hóa học để bóc tách vật liệu, có đặc điểm:
               \begin{itemize}
                    \item Là quá trình ngược với mạ điện (electroplating).
                    \item Vật liệu gia công phải là vật dẫn điện.
                    \item Các quá trình bao gồm:
                         \begin{itemize}
                              \item Gia công điện hóa - Electrochemical Machining (ECM)
                              \item Khử ba via bằng điện hóa - Electrochemical Deburring (ECD)
                              \item Mài điện hóa - Electrochemical Grinding (ECG)
                         \end{itemize}
               \end{itemize}
          \subsection{Gia công điện hóa - Electrochemical Machining (ECM)}
               \subsubsection{Định nghĩa}
                    \hspace*{0.6cm}Vật liệu được bóc tách nhờ quá trình hòa tan anot, sử dụng điện cực (dụng cụ) đặt ở rất gần chi tiết gia công, nhưng được ngăn cách bởi dòng dung dịch điện phân chảy nhanh.
                    \begin{figure}[H]
                         \centering
                         \includegraphics[width=0.6\textwidth]{pictures/chapter7/c7_p04.png}
                         \caption{Sơ đồ quá trình gia công điện hóa (Electrochemical Machining - ECM)}
                    \end{figure}
               \subsubsection{Ứng dụng của ECM}
                    \hspace*{0.6cm}Ứng dụng của ECM bao gồm:
                    \begin{itemize}
                         \item Gia công lõm khuôn (die sinking) – tạo các hình dạng và biên dạng phức tạp, không đều cho khuôn dập rèn, khuôn nhựa và các loại dụng cụ khác.
                         \item Khoan nhiều lỗ đồng thời – ECM cho phép khoan nhiều lỗ cùng lúc.
                         \item Gia công các lỗ không tròn.
                         \item Không sử dụng mũi khoan quay trong ECM.
                         \item Khử bavia (deburring).
                    \end{itemize}
          \subsection{Khử ba via bằng điện hóa - Electrochemical Deburring (ECD)}
               \subsubsection{Định nghĩa}
                    \hspace*{0.6cm}Là sự cải tiến/ứng dụng của ECM nhằm loại bỏ ba via hoặc các cạnh sắc tại miệng lỗ trên các chi tiết kim loại được tạo ra bằng phương pháp khoan lỗ xuyên thông thường.
                    \begin{figure}[H]
                         \centering
                         \includegraphics[width=0.6\textwidth]{pictures/chapter7/c7_p05.png}
                         \caption{Sơ đồ quá trình khử ba via bằng điện hóa (Electrochemical Deburring - ECD)}
                    \end{figure}
          \subsection{Mài điện hóa - Electrochemical Grinding (ECG)}
               \subsubsection{Định nghĩa}
                    \hspace*{0.6cm}Là một dạng đặc biệt của ECM, trong đó đá mài có chất liên kết dẫn điện được sử dụng để hỗ trợ quá trình hòa tan anot của bề mặt chi tiết kim loại.
                    \begin{figure}[H]
                         \centering
                         \includegraphics[width=0.55\textwidth]{pictures/chapter7/c7_p06.png}
                         \caption{Sơ đồ quá trình mài điện hóa (Electrochemical Grinding - ECG)}
                    \end{figure}
               \subsubsection{Ưu điểm và ứng dụng của ECG}
               \hspace*{0.6cm}\textbf{Ưu điểm:}
               \begin{itemize}
                    \item Quá trình hòa tan điện hóa (deplating) chiếm khoảng 95\% lượng vật liệu bóc tách.
                    \item Do gia công chủ yếu bằng tác dụng điện hóa, nên đá mài có tuổi thọ cao hơn nhiều so với mài truyền thống.
               \end{itemize}
               \hspace*{0.6cm}\textbf{Ứng dụng:}
               \begin{itemize}
                    \item Mài sắc các dụng cụ hợp kim cứng thiêu kết (cemented carbide).
                    \item Mài kim khâu phẫu thuật và các ống thành mỏng khác, cũng như các chi tiết mỏng, dễ hư hỏng.
               \end{itemize}
     \section{Các quá trình năng lượng nhiệt (Thermal Energy Processes)}
          \subsection{Tổng quan}
          \hspace*{0.6cm}Là một nhóm các quá trình gia công vật liệu có đặc điểm:
          \begin{itemize}
               \item Nhiệt độ cục bộ (tại vị trí gia công) rất cao, vật liệu bị bóc tách do nóng chảy hoặc hóa hơi.
               \item Gây hư hỏng cho bề mặt chi tiết sau gia công.
               \item Trong một số trường hợp, chất lượng bề mặt rất kém, đôi khi cần phải cần các bước gia công tiếp theo để hoàn thiện.
          \end{itemize}
          \hspace*{0.6cm}Các quá trình gồm có:
          \begin{itemize}
               \item Gia công bằng tia lửa điện (Electric Discharge Machining – EDM)
               \item Cắt dây tia lửa điện (Electric Discharge Wire Cutting – Wire EDM)
               \item Gia công bằng chùm tia điện tử (Electron Beam Machining – EBM)
               \item Gia công bằng tia laser (Laser Beam Machining – LBM)
               \item Gia công bằng hồ quang plasma (Plasma Arc Machining – PAM)
               \item Các quá trình cắt nhiệt truyền thống (Conventional thermal cutting processes)
          \end{itemize}
          \hspace{0.6cm}Cụ thể:
          \subsection{Gia công bằng tia lửa điện (Electric Discharge Machining – EDM)}
               \subsubsection{Định nghĩa}
                    \begin{itemize}
                         \item Vật liệu kim loại được bóc tách nhờ một chuỗi các điện tích rời rạc được phóng ra (tia lửa điện), tạo ra nhiệt độ cục bộ tại vị trí gia công, cắt gọt đủ cao để làm nóng chảy hoặc hóa hơi kim loại.
                         \item Chỉ có thể áp dụng đối với vật liệu gia công dẫn điện, có 2 loại chính:
                         \begin{itemize}[label={--}]       
                              \item Gia công tia lửa điện (Electric Discharge Machining – EDM)
                              \item Cắt dây tia lửa điện (Wire Electric Discharge Machining – Wire EDM)
                         \end{itemize}
                    \end{itemize}
          \subsection{Gia công tia lửa điện (Electric Discharge Machining – EDM)}
               \subsubsection{Định nghĩa}
                    \hspace*{0.6cm}Hình ảnh minh họa quá trình EDM được thể hiện như sau:
                    \begin{figure}[H]
                         \centering
                         \includegraphics[width=0.8\textwidth]{pictures/chapter7/c7_p07.png}
                         \caption{Sơ đồ quá trình gia công bằng tia lửa điện (Electric Discharge Machining - EDM)}
                    \end{figure}
                    \hspace*{0.6cm}(a) Sơ đồ bố trí của quá trình gia công (b) Hình phóng to khe hở, thể hiện quá trình phóng điện và sự bóc tách vật liệu kim loại
               \subsubsection{Vật liệu gia công bằng phương pháp EDM}
                    \begin{itemize}
                         \item Vật liệu gia công phải dẫn điện
                         \item Độ cứng và độ bền của vật liệu không ảnh hưởng đến khả năng gia công bằng EDM
                         \item Tốc độ bóc tách vật liệu phụ thuộc vào nhiệt độ nóng chảy của vật liệu gia công
                    \end{itemize}  
               \subsubsection{Ứng dụng của EDM}
                    \hspace*{0.6cm}Ứng dụng của EDM bao gồm:
                    \begin{itemize}
                         \item Chế tạo dụng cụ cho nhiều quá trình gia công cơ khí, bao gồm: khuôn ép phun nhựa, khuôn ép đùn, khuôn kéo dây, khuôn rèn và dập đầu, và khuôn dập tấm.
                         \item Gia công chi tiết sản xuất, như: các chi tiết mỏng manh không đủ độ cứng vững để chịu lực cắt truyền thống, khoan lỗ có trục lỗ tạo góc nhọn với bề mặt, và gia công các kim loại cứng hoặc kim loại đặc biệt (exotic metals).
                    \end{itemize} 
          \subsection{Cắt dây tia lửa điện (Wire Electric Discharge Machining – Wire EDM)}
               \subsubsection{Định nghĩa}
                    \hspace*{0.6cm}Một dạng đặc biệt của EDM sử dụng dây có đường kính nhỏ làm điện cực để cắt một rãnh hẹp trên vật gia công.
                    \begin{figure}[H]
                         \centering
                         \includegraphics[width=0.7\textwidth]{pictures/chapter7/c7_p08.png}
                         \caption{Sơ đồ quá trình cắt dây tia lửa điện (Wire Electric Discharge Machining - Wire EDM)}
                    \end{figure}
               \subsubsection{Quy trình hoạt động của Wire EDM}
                    \begin{itemize}
                         \item Chi tiết gia công được tiến chậm dọc theo dây cắt theo quỹ đạo mong muốn.
                         \item Chuyển động được điều khiển bằng CNC.
                         \item Trong quá trình cắt, dây điện cực được cấp liên tục từ cuộn cấp sang cuộn thu nhằm duy trì đường kính dây không đổi.
                         \item Cần sử dụng chất điện môi, được phun qua các vòi phun hướng vào vùng tiếp xúc dây–chi tiết hoặc nhúng ngập chi tiết trong chất điện môi.
                         \item Độ rộng vết cắt (kerf) và độ cắt vượt mức (overcut) trong cắt dây bằng phóng điện (EDL)
                         \begin{figure}[H]
                              \centering
                              \includegraphics[width=0.5\textwidth]{pictures/chapter7/c7_p09.png}
                              \caption{Độ rộng vết cắt (kerf) và độ cắt vượt mức (overcut) trong cắt dây bằng phóng điện (EDL)}
                         \end{figure}
                    \end{itemize}
               \subsubsection{Ứng dụng của cắt dây - Wire EDM}
                    \hspace*{0.6cm}Ứng dụng của Wire EDM bao gồm:
                    \begin{itemize}
                         \item Phù hợp để chế tạo các chi tiết khuôn dập.
                         \item Do chiều rộng rãnh cắt (kerf) rất nhỏ, nên thường có thể gia công chày và cối (punch \& die) chỉ trong một lần cắt.
                         \item Gia công các dụng cụ và chi tiết khác có biên dạng phức tạp, chẳng hạn dao định hình trên máy tiện, khuôn ép đùn và dưỡng phẳng (flat templates).
                    \end{itemize}