\chapter{MÀI VÀ CÁC QUÁ TRÌNH GIA CÔNG BẰNG HẠT MÀI KHÁC (GRINDING)}
\section{Gia công mài và các quá trình gia công bằng hạt mài khác}

Gia công mài (Grinding) và các quá trình gia công bằng hạt mài khác là nhóm phương pháp gia công cắt gọt trong đó vật liệu được bóc đi nhờ các hạt mài có độ cứng rất cao. Các quá trình này thường được sử dụng ở công đoạn gia công tinh nhằm đạt độ chính xác kích thước cao và chất lượng bề mặt tốt.

\section{Gia công bằng hạt mài}
Gia công bằng hạt mài là quá trình bóc vật liệu thông qua sự tiếp xúc giữa chi tiết gia công và các hạt mài cứng, được liên kết trong bánh mài hoặc tồn tại tự do.

Đặc điểm chính của gia công bằng hạt mài:
\begin{itemize}
    \item Bóc vật liệu bằng các hạt mài rất cứng.
    \item Các hạt mài thường được liên kết trong bánh mài bằng chất kết dính.
    \item Chủ yếu dùng cho các nguyên công hoàn thiện.
    \item Có thể đạt độ nhẵn bề mặt rất cao, tới $0.025~\mu$m.
    \item Đảm bảo dung sai kích thước nhỏ.
\end{itemize}

Các quá trình tiêu biểu gồm mài (grinding), vát (honing) và lapping.

\section{Bánh mài}
Bánh mài là dụng cụ cắt chính trong quá trình mài. Bánh mài bao gồm các hạt mài được liên kết với nhau bằng chất kết dính và có cấu trúc rỗng nhất định.

\subsection*{Cấu tạo và nguyên lý}
\begin{itemize}
    \item Bánh mài phải được cân bằng để tránh rung động khi quay tốc độ cao.
    \item Các hạt mài được giữ bởi chất kết dính, tạo thành công cụ cắt.
\end{itemize}

\subsection*{Các thông số thiết kế quan trọng}
\begin{itemize}
    \item \textbf{Kích thước hạt (grain size)}: ảnh hưởng độ nhẵn và năng suất.
    \item \textbf{Độ cứng bánh mài (wheel grade)}: khả năng giữ hạt mài.
    \item \textbf{Cấu trúc bánh mài (wheel structure)}: khả năng thoát phoi và làm mát.
\end{itemize}

\subsection*{Các tính chất cơ học cần có}
\begin{itemize}
    \item \textbf{Cứng}: để cắt vật liệu.
    \item \textbf{Chống mài mòn}: để làm việc ổn định.
    \item \textbf{Dai (toughness)}: để không vỡ đột ngột.
    \item \textbf{Tự vỡ (friability)}: để tạo cạnh cắt mới khi hạt mài bị cùn.
\end{itemize}

\section{Vật liệu hạt mài}
Các loại hạt mài chính:
\begin{table}[H]
\centering
\begin{tabular}{|p{4cm}|p{5cm}|p{6cm}|}
\hline
\textbf{Vật liệu Hạt mài} & \textbf{Đặc điểm} & \textbf{Ứng dụng (Vật liệu gia công)} \\
\hline
\textbf{Nhôm Oxit ($Al_2O_3$)} & Phổ biến, dai & \textbf{Thép (Steel)}. \\
\hline
\textbf{Silic Cacbua (SiC)} & Cứng hơn $Al_2O_3$, giòn & \textbf{Nhôm, Đồng thau}. \\
\hline
\textbf{CBN (Cubic Boron Nitride)} & Rất cứng, rất đắt & \textbf{Thép dụng cụ đã tôi cứng} (Hardened tool steels). \\
\hline
\textbf{Kim cương (Diamond)} & Cứng nhất, đắt nhất & \textbf{Gốm (Ceramics), Thủy tinh (Glass)}. \\
\hline
\end{tabular}
\end{table}
\section{Kích thước hạt mài và chất kết dính}

\subsection*{Kích thước hạt mài}
\begin{itemize}
    \item Kích thước hạt nhỏ tạo bề mặt hoàn thiện tốt hơn.
    \item Kích thước hạt lớn dùng cho bóc vật liệu nhiều.
    \item Vật liệu gia công cứng hơn cần hạt mài nhỏ hơn.
\end{itemize}

\subsection*{Chất kết dính}
\begin{itemize}
    \item Chịu được lực ly tâm.
    \item Chịu được nhiệt độ cao.
    \item Chống vỡ khi chịu tải trọng va đập.
    \item Giữ các hạt mài một cách cứng chắc.
    \item Cho phép các hạt mài đã mòn được bong ra.
\end{itemize}

\section{Cấu trúc bánh mài}
Cấu trúc bánh mài thể hiện tỷ lệ giữa các thành phần trong bánh mài.
\begin{figure}[H]
    \centering
    \includegraphics[width=0.6\textwidth]{pictures/chapter5/c05_p01.png}
    \caption{Cấu trúc bánh mài}
\end{figure}
\textbf{Cấu trúc mở (Open structure):}
\begin{itemize}
    \item $P_p$ tương đối lớn.
    \item $P_g$ tương đối nhỏ.
\end{itemize}

\textbf{Cấu trúc đặc (Dense structure):}
\begin{itemize}
    \item $P_p$ tương đối nhỏ.
    \item $P_g$ lớn hơn.
\end{itemize}

Quan hệ thể tích:
\[
P_g + P_b + P_p = 1
\]

trong đó $P_g$ là thể tích hạt mài, $P_b$ là thể tích chất kết dính và $P_p$ là thể tích lỗ rỗng.

\section{Độ cứng bánh mài (Wheel Grade)}
\begin{itemize}
    \item Độ bền của liên kết trong việc giữ các hạt mài.
    \begin{itemize}
        \item Phụ thuộc vào lượng chất kết dính.
        \item Cấu trúc bánh mài ($P_b$).
    \end{itemize}
    
    \item \textbf{Bánh mài "mềm"} làm mất hạt mài dễ dàng.
    \begin{itemize}
        \item Tốc độ bóc vật liệu thấp và gia công vật liệu cứng.
    \end{itemize}
    
    \item \textbf{Bánh mài cứng} giữ hạt mài tốt.
    \begin{itemize}
        \item Tốc độ bóc vật liệu cao và gia công vật liệu mềm.
    \end{itemize}
    
    \item Các yếu tố quyết định độ cứng bánh mài: loại hạt mài, kích thước hạt, độ cứng, cấu trúc, chất kết dính.
\end{itemize}
\section{Hình dạng bánh mài theo chuẩn}
\begin{figure}[H]
    \centering
    \includegraphics[width=0.7\textwidth]{pictures/chapter5/c05_p03.png}
    \caption{Hình dạng bánh mài theo chuẩn}
\end{figure}
\section{Cơ chế tác dụng của hạt mài}
Trong quá trình mài, hạt mài tác động lên bề mặt chi tiết theo ba cơ chế:
\begin{itemize}
    \item Cày xới (Plowing)
    \item Ma sát (Rubbing)
    \item Cắt (Cutting)
\end{itemize}
\begin{figure}[H]
    \centering
    \includegraphics[width=0.6\textwidth]{pictures/chapter5/c05_p02.png}
    \caption{Cơ chế tác dụng của hạt mài}
\end{figure}
\section{Các phương trình cơ bản trong mài}

\textbf{Tốc độ bề mặt bánh mài:}
\[
v = \pi D N
\]
\begin{itemize}
    \item $v$ = tốc độ bề mặt bánh mài (surface speed of wheel)
    \item $D$ = đường kính bánh mài (wheel diameter)
    \item $N$ = tốc độ trục chính (spindle speed)
\end{itemize}

\textbf{Chiều dài phoi:}
\[
l_c = (Dd)^{0.5}
\]
\begin{itemize}
    \item $l_c$ = chiều dài phoi (length of chip)
    \item $d$ = chiều độ sâu cắt (depth of cut)
\end{itemize}

\textbf{Tốc độ bóc vật liệu:}
\[
R_{MR} = v_w w d
\]
\begin{itemize}
    \item $R_{MR}$ = tốc độ bóc vật liệu (material removal rate)
    \item $v_w$ = tốc độ chi tiết gia công (speed of work)
    \item $w$ = chiều rộng hoặc bước tiến ngang (width or crossfeed)
\end{itemize}

\textbf{Số phoi tạo ra trong một đơn vị thời gian:}
\[
n_c = vwC
\]
\begin{itemize}
    \item $n_c$ = số phoi tạo ra trong một đơn vị thời gian (number of chips formed per unit time)
    \item $C$ = số hạt mài trên một đơn vị diện tích (number of grits per unit area)
    \item $v$ = tốc độ bánh mài (speed of wheel)
\end{itemize}

\textbf{Năng lượng riêng:}
\[
U = \frac{F_c v}{R_{MR}}
\]
\begin{itemize}
    \item $U$ = năng lượng riêng (specific energy)
    \item $F_c$ = lực cắt (cutting force)
\end{itemize}

\section{Nhiệt độ trên bề mặt gia công}

\begin{itemize}
    \item Quá trình mài đặc trưng bởi nhiệt độ cao và ma sát cao.
    \item Năng lượng lưu lại trên bề mặt mài.
    \item Dẫn đến nhiệt độ bề mặt gia công cao.
\end{itemize}

\textbf{Các tác hại của nhiệt:}
\begin{itemize}
    \item Cháy bề mặt và nứt (Surface burns and cracks).
    \item Hư hỏng tổ chức kim loại ngay bên dưới bề mặt (Metallurgical damage immediately beneath the surface).
    \item Mềm hóa bề mặt đối với vật liệu đã nhiệt luyện (Softening of the work surface if heat treated).
    \item Xuất hiện ứng suất dư trên bề mặt gia công (Residual stresses in the work surface).
\end{itemize}

\section{Biện pháp giảm nhiệt khi mài}
\begin{itemize}
    \item Giảm lượng tiến dao (chiều sâu cắt) $d$ (Decrease infeed - depth of cut).
    \item Giảm tốc độ bánh mài $v$ (Reduce wheel speed).
    \item Giảm số hạt mài hoạt động trên mỗi inch vuông trên bánh mài $C$ (Reduce number of active grits per square inch on the grinding wheel).
    \item Tăng tốc độ chi tiết gia công $v_w$ (Increase work speed).
    \item Sử dụng dung dịch mài (Use a grinding fluid).
\end{itemize}

\section{Sự mòn của bánh mài}
Bánh mài bị mòn theo ba cơ chế:
\begin{enumerate}
    \item \textbf{Vỡ hạt mài (Grain fracture)}:
    \begin{itemize}
        \item Một phần của hạt mài bị bẻ gãy.
    \end{itemize}
    
    \item \textbf{Mòn cùn hạt mài (Attritious wear)}:
    \begin{itemize}
        \item Mài mòn các hạt mài riêng lẻ.
    \end{itemize}
    
    \item \textbf{Bong hạt mài khỏi chất kết dính (Bond fracture)}:
    \begin{itemize}
        \item Hạt mài bị kéo ra khỏi chất kết dính.
    \end{itemize}
\end{enumerate}

\subsection*{Dressing (Mài sắc bánh mài)}
Dressing được thực hiện bằng đĩa quay:
\begin{itemize}
    \item Bẻ gãy các hạt mài cùn để lộ hạt mới.
    \item Loại bỏ phoi bám trong bánh mài.
\end{itemize}

\section{Sửa dạng bánh mài (Truing)}

\begin{itemize}
    \item Sử dụng dụng cụ đầu kim cương di chuyển chậm và chính xác qua bánh mài khi nó quay.
    \item Lượng cắt rất nhỏ (0.025 mm hoặc nhỏ hơn) trên bánh mài.
    \item Làm sắc bánh mài.
    \item Khôi phục hình dạng trụ và đảm bảo độ thẳng trên chu vi bên ngoài.
    \item Dressing làm sắc bánh mài nhưng không đảm bảo hình dạng của bánh mài.
\end{itemize}

\section{Hướng dẫn lựa chọn và sử dụng}

\subsection*{Để tối ưu hóa chất lượng bề mặt, chọn:}
\begin{itemize}
    \item Kích thước hạt nhỏ và cấu trúc bánh mài đặc.
    \item Sử dụng tốc độ bánh mài cao hơn ($v$) và tốc độ chi tiết thấp hơn ($v_w$).
    \item Chiều sâu cắt nhỏ hơn ($d$) và đường kính bánh mài lớn hơn ($D$) cũng giúp cải thiện.
\end{itemize}

\subsection*{Để tối đa hóa tốc độ bóc vật liệu, chọn:}
\begin{itemize}
    \item Kích thước hạt lớn.
    \item Cấu trúc bánh mài mở hơn.
    \item Sử dụng chất kết dính gốm (vitrified bond).
\end{itemize}
\subsection*{Đối với thép và hầu hết gang đúc:}
\begin{itemize}
    \item Sử dụng nhôm oxit làm hạt mài.
\end{itemize}

\subsection*{Đối với hầu hết kim loại màu:}
\begin{itemize}
    \item Sử dụng silic cacbua làm hạt mài.
\end{itemize}

\subsection*{Đối với thép dụng cụ đã tôi cứng và một số hợp kim hàng không:}
\begin{itemize}
    \item Sử dụng bo nitrua lập phương làm hạt mài.
\end{itemize}

\subsection*{Đối với vật liệu mài mòn cứng (ví dụ: gốm, cacbua xi măng và thủy tinh):}
\begin{itemize}
    \item Sử dụng kim cương làm hạt mài.
\end{itemize}

\subsection*{Đối với kim loại mềm:}
\begin{itemize}
    \item Sử dụng kích thước hạt lớn và bánh mài cứng hơn.
\end{itemize}

\subsection*{Đối với kim loại cứng:}
\begin{itemize}
    \item Sử dụng kích thước hạt nhỏ và bánh mài mềm hơn.
\end{itemize}

\section{Phân tích quá trình mài}
\begin{itemize}
    \item Tỷ số mài (Grinding ratio) và tốc độ bề mặt là hàm của tốc độ bánh mài.
    \item Đường cong mòn điển hình của bánh mài.
\end{itemize}
\begin{figure}[H]
    \centering
    \includegraphics[width=0.7\textwidth]{pictures/chapter5/c05_p04.png}
    \caption{Phân tích quá trình mài}
\end{figure}
\section{Các phương pháp mài}

\subsection{Bốn loại mài phẳng (Surface Grinding)}
\begin{enumerate}
    \item \textbf{Trục chính nằm ngang với bàn di chuyển tịnh tiến} (Horizontal spindle with reciprocating worktable) - Loại (a) là máy mài phổ biến nhất.
    \item \textbf{Trục chính nằm ngang với bàn quay} (Horizontal spindle with rotating worktable).
    \item \textbf{Trục chính thẳng đứng với bàn di chuyển tịnh tiến} (Vertical spindle with reciprocating worktable).
    \item \textbf{Trục chính thẳng đứng với bàn quay} (Vertical spindle with rotating worktable).
\end{enumerate}
\begin{figure}[H]
    \centering
    \includegraphics[width=0.9\textwidth]{pictures/chapter5/c05_p05.png}
    \caption{Các loại máy mài phẳng}
\end{figure}
\subsection{Mài trụ (Cylindrical Grinding)}
\begin{itemize}
    \item \textbf{Mài trụ ngoài (External)}: mài bề mặt ngoài chi tiết.
    \item \textbf{Mài trụ trong (Internal)}: mài bề mặt trong chi tiết.
\end{itemize}

\textbf{Hai phương pháp tiến dao:}
\begin{itemize}
    \item \textbf{Tiến dao ngang (Traverse feed)}: bánh mài di chuyển dọc theo chi tiết.
    \item \textbf{Tiến dao chìm (Plunge cut)}: bánh mài chìm vào chi tiết.
\end{itemize}
\begin{figure}[H]
    \centering
    \includegraphics[width=0.8\textwidth]{pictures/chapter5/c05_p06.png}
    \caption{Mài trụ ngoài và mài trụ trong}
\end{figure}
\subsection{Mài không tâm (Centerless Grinding)}
\begin{itemize}
    \item \textbf{Mài vô tâm ngoài (External centerless grinding)}: chi tiết được đỡ giữa bánh mài, bánh điều chỉnh và lưỡi đỡ.
    \item \textbf{Mài vô tâm trong (Internal centerless grinding)}: mài lỗ trong của chi tiết.
\end{itemize}
\begin{figure}[H]
    \centering
    \includegraphics[width=0.7\textwidth]{pictures/chapter5/c05_p07.png}
    \caption{Mài vô tâm ngoài và mài không tâm trong}
\end{figure}

\subsection{Mài tiến dao chậm (Creep Feed Grinding)}

So sánh với mài phẳng thông thường:

\textbf{Mài phẳng thông thường (Conventional surface grinding):}
\begin{itemize}
    \item Chiều sâu cắt nhỏ (infeed $d$).
    \item Chiều dài hành trình điển hình dài.
    \item Tốc độ chi tiết bình thường ($v_w$).
\end{itemize}

\textbf{Mài tiến dao chậm (Creep feed grinding):}
\begin{itemize}
    \item Chiều sâu cắt lớn (depth $d$).
    \item Chiều dài hành trình ngắn hơn (length of pass).
    \item Tốc độ tiến dao rất chậm (work feed - slow).
\end{itemize}
% \begin{figure}[H]
%     \centering
%     \includegraphics[width=0.8\textwidth]{pictures/chapter5/c05_p08.png}
%     \caption{So sánh mài phẳng thông thường và mài tiến dao chậm}
% \end{figure}

\subsection{Mài đĩa và mài băng nhám}

\textbf{Mài đĩa (Disc grinder):}
\begin{itemize}
    \item Sử dụng đĩa mài (abrasive disc) gắn trên trục chính (spindle).
    \item Có vỏ bảo vệ (guard).
    \item Chi tiết gia công đặt trên bàn máy (worktable).
\end{itemize}

\textbf{Mài đai nhám (Abrasive belt grinder):}
\begin{itemize}
    \item Sử dụng đai mài (abrasive belt) di chuyển với tốc độ $v$ (belt speed).
    \item Đai mài được căng giữa con lăn dẫn (drive spindle) và con lăn căng (idler pulley).
    \item Chi tiết được đỡ bởi bàn đỡ (platen).
\end{itemize}
% \begin{figure}[H]
%     \centering
%     \includegraphics[width=0.8\textwidth]{pictures/chapter5/c05_p09.png}
%     \caption{Mài đĩa và mài băng nhám}
% \end{figure}

\section{Các quá trình gia công hoàn thiện}

\subsection{Vát (Honing)}

\textbf{Dụng cụ vát cho bề mặt lỗ trong:}
\begin{itemize}
    \item Trục dẫn động (driver) kết nối qua khớp nối toàn cầu (universal joint).
    \item Thanh mài liên kết (bonded abrasive sticks) - thường có 4 thanh.
    \item Chuyển động quay (rotating motion) kết hợp với chuyển động tịnh tiến (reciprocating motion).
\end{itemize}

\textbf{Đặc điểm và ứng dụng:}
\begin{itemize}
    \item Hoàn thiện các lỗ trong của động cơ đốt trong.
    \item Đạt độ nhẵn bề mặt 0.12 $\mu$m (5 $\mu$-in) hoặc tốt hơn.
    \item Tạo bề mặt vân chéo (cross-hatched surface) giúp giữ dầu bôi trơn.
\end{itemize}
% \begin{figure}[H]
%     \centering
%     \includegraphics[width=0.7\textwidth]{pictures/chapter5/c05_p10.png}
%     \caption{Vát và bề mặt vân chéo}
% \end{figure}

\subsection{Lapping và Superfinishing}

\textbf{Quá trình lapping trong gia công thấu kính:}
\begin{itemize}
    \item Sử dụng đĩa lapping (lap tool) được dẫn động bởi động cơ (motor-driven lap).
    \item Phôi thấu kính (lens blank - work) được gia công.
    \item Sử dụng hỗn hợp lapping (lapping compound) giữa đĩa và chi tiết.
\end{itemize}

\textbf{Superfinishing trên bề mặt trụ ngoài:}
\begin{itemize}
    \item Sử dụng thanh mài liên kết (bonded abrasive stick).
    \item Chuyển động tịnh tiến của thanh mài với tần số cao và biên độ thấp (reciprocating motion of stick - high frequency and low amplitude).
    \item Chi tiết quay chậm (rotation of work - slow).
\end{itemize}
% \begin{figure}[H]
%     \centering
%     \includegraphics[width=0.8\textwidth]{pictures/chapter5/c05_p11.png}
%     \caption{Lapping và Superfinishing}
% \end{figure}
