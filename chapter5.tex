\chapter{MÀI VÀ CÁC QUÁ TRÌNH GIA CÔNG BẰNG HẠT MÀI KHÁC (GRINDING)}
\section{Gia công mài và các quá trình gia công bằng hạt mài khác}

Gia công mài (Grinding) và các quá trình gia công bằng hạt mài khác là nhóm phương pháp gia công cắt gọt trong đó vật liệu được bóc đi nhờ các hạt mài có độ cứng rất cao. Các quá trình này thường được sử dụng ở công đoạn gia công tinh nhằm đạt độ chính xác kích thước cao và chất lượng bề mặt tốt.

\section{Gia công bằng hạt mài}
Gia công bằng hạt mài là quá trình bóc vật liệu thông qua sự tiếp xúc giữa chi tiết gia công và các hạt mài cứng, được liên kết trong bánh mài hoặc tồn tại tự do.

Đặc điểm chính của gia công bằng hạt mài:
\begin{itemize}
    \item Bóc vật liệu bằng các hạt mài rất cứng.
    \item Các hạt mài thường được liên kết trong bánh mài bằng chất kết dính.
    \item Chủ yếu dùng cho các nguyên công hoàn thiện.
    \item Có thể đạt độ nhẵn bề mặt rất cao, tới $0.025~\mu$m.
    \item Đảm bảo dung sai kích thước nhỏ.
\end{itemize}

Các quá trình tiêu biểu gồm mài (grinding), vát (honing) và lapping.

\section{Bánh mài}
Bánh mài là dụng cụ cắt chính trong quá trình mài. Bánh mài bao gồm các hạt mài được liên kết với nhau bằng chất kết dính và có cấu trúc rỗng nhất định.

\subsection*{Cấu tạo và nguyên lý}
\begin{itemize}
    \item Bánh mài phải được cân bằng để tránh rung động khi quay tốc độ cao.
    \item Các hạt mài được giữ bởi chất kết dính, tạo thành công cụ cắt.
\end{itemize}

\subsection*{Các thông số thiết kế quan trọng}
\begin{itemize}
    \item \textbf{Kích thước hạt (grain size)}: ảnh hưởng độ nhẵn và năng suất.
    \item \textbf{Độ cứng bánh mài (wheel grade)}: khả năng giữ hạt mài.
    \item \textbf{Cấu trúc bánh mài (wheel structure)}: khả năng thoát phoi và làm mát.
\end{itemize}

\subsection*{Các tính chất cơ học cần có}
\begin{itemize}
    \item \textbf{Cứng}: để cắt vật liệu.
    \item \textbf{Chống mài mòn}: để làm việc ổn định.
    \item \textbf{Dai (toughness)}: để không vỡ đột ngột.
    \item \textbf{Tự vỡ (friability)}: để tạo cạnh cắt mới khi hạt mài bị cùn.
\end{itemize}

\section{Vật liệu hạt mài}
Các loại hạt mài chính:
\begin{table}[H]
\centering
\begin{tabular}{|p{4cm}|p{5cm}|p{6cm}|}
\hline
\textbf{Vật liệu Hạt mài} & \textbf{Đặc điểm} & \textbf{Ứng dụng (Vật liệu gia công)} \\
\hline
\textbf{Nhôm Oxit ($Al_2O_3$)} & Phổ biến, dai & \textbf{Thép (Steel)}. \\
\hline
\textbf{Silic Cacbua (SiC)} & Cứng hơn $Al_2O_3$, giòn & \textbf{Nhôm, Đồng thau}. \\
\hline
\textbf{CBN (Cubic Boron Nitride)} & Rất cứng, rất đắt & \textbf{Thép dụng cụ đã tôi cứng} (Hardened tool steels). \\
\hline
\textbf{Kim cương (Diamond)} & Cứng nhất, đắt nhất & \textbf{Gốm (Ceramics), Thủy tinh (Glass)}. \\
\hline
\end{tabular}
\end{table}
\section{Kích thước hạt mài và chất kết dính}

\subsection*{Kích thước hạt mài}
\begin{itemize}
    \item Kích thước hạt nhỏ tạo bề mặt hoàn thiện tốt hơn.
    \item Kích thước hạt lớn dùng cho bóc vật liệu nhiều.
    \item Vật liệu gia công cứng hơn cần hạt mài nhỏ hơn.
\end{itemize}

\subsection*{Chất kết dính}
\begin{itemize}
    \item Chịu được lực ly tâm.
    \item Chịu được nhiệt độ cao.
    \item Chống vỡ khi chịu tải trọng va đập.
    \item Giữ các hạt mài một cách cứng chắc.
    \item Cho phép các hạt mài đã mòn được bong ra.
\end{itemize}

\section{Cấu trúc bánh mài}
Cấu trúc bánh mài thể hiện tỷ lệ giữa các thành phần trong bánh mài.
\begin{figure}[H]
    \centering
    \includegraphics[width=0.6\textwidth]{pictures/chapter5/c05_p01.png}
    \caption{Cấu trúc bánh mài}
\end{figure}
\textbf{Cấu trúc mở (Open structure):}
\begin{itemize}
    \item $P_p$ tương đối lớn.
    \item $P_g$ tương đối nhỏ.
\end{itemize}

\textbf{Cấu trúc đặc (Dense structure):}
\begin{itemize}
    \item $P_p$ tương đối nhỏ.
    \item $P_g$ lớn hơn.
\end{itemize}

Quan hệ thể tích:
\[
P_g + P_b + P_p = 1
\]

trong đó $P_g$ là thể tích hạt mài, $P_b$ là thể tích chất kết dính và $P_p$ là thể tích lỗ rỗng.

\section{Độ cứng bánh mài (Wheel Grade)}
\begin{itemize}
    \item Độ bền của liên kết trong việc giữ các hạt mài.
    \begin{itemize}
        \item Phụ thuộc vào lượng chất kết dính.
        \item Cấu trúc bánh mài ($P_b$).
    \end{itemize}
    
    \item \textbf{Bánh mài "mềm"} làm mất hạt mài dễ dàng.
    \begin{itemize}
        \item Tốc độ bóc vật liệu thấp và gia công vật liệu cứng.
    \end{itemize}
    
    \item \textbf{Bánh mài cứng} giữ hạt mài tốt.
    \begin{itemize}
        \item Tốc độ bóc vật liệu cao và gia công vật liệu mềm.
    \end{itemize}
    
    \item Các yếu tố quyết định độ cứng bánh mài: loại hạt mài, kích thước hạt, độ cứng, cấu trúc, chất kết dính.
\end{itemize}
\section{Hình dạng bánh mài theo chuẩn}
\begin{figure}[H]
    \centering
    \includegraphics[width=0.7\textwidth]{pictures/chapter5/c05_p03.png}
    \caption{Hình dạng bánh mài theo chuẩn}
\end{figure}
\section{Cơ chế tác dụng của hạt mài}
Trong quá trình mài, hạt mài tác động lên bề mặt chi tiết theo ba cơ chế:
\begin{itemize}
    \item Cày xới (Plowing)
    \item Ma sát (Rubbing)
    \item Cắt (Cutting)
\end{itemize}
\begin{figure}[H]
    \centering
    \includegraphics[width=0.6\textwidth]{pictures/chapter5/c05_p02.png}
    \caption{Cơ chế tác dụng của hạt mài}
\end{figure}
\section{Các phương trình cơ bản trong mài}

\textbf{Tốc độ bề mặt bánh mài:}
\[
v = \pi D N
\]
\begin{itemize}
    \item $v$ = tốc độ bề mặt bánh mài (surface speed of wheel)
    \item $D$ = đường kính bánh mài (wheel diameter)
    \item $N$ = tốc độ trục chính (spindle speed)
\end{itemize}

\textbf{Chiều dài phoi:}
\[
l_c = (Dd)^{0.5}
\]
\begin{itemize}
    \item $l_c$ = chiều dài phoi (length of chip)
    \item $d$ = chiều độ sâu cắt (depth of cut)
\end{itemize}

\textbf{Tốc độ bóc vật liệu:}
\[
R_{MR} = v_w w d
\]
\begin{itemize}
    \item $R_{MR}$ = tốc độ bóc vật liệu (material removal rate)
    \item $v_w$ = tốc độ chi tiết gia công (speed of work)
    \item $w$ = chiều rộng hoặc bước tiến ngang (width or crossfeed)
\end{itemize}

\textbf{Số phoi tạo ra trong một đơn vị thời gian:}
\[
n_c = vwC
\]
\begin{itemize}
    \item $n_c$ = số phoi tạo ra trong một đơn vị thời gian (number of chips formed per unit time)
    \item $C$ = số hạt mài trên một đơn vị diện tích (number of grits per unit area)
    \item $v$ = tốc độ bánh mài (speed of wheel)
\end{itemize}

\textbf{Năng lượng riêng:}
\[
U = \frac{F_c v}{R_{MR}}
\]
\begin{itemize}
    \item $U$ = năng lượng riêng (specific energy)
    \item $F_c$ = lực cắt (cutting force)
\end{itemize}
