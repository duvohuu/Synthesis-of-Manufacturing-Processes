\chapter{CÁC NGUYÊN CÔNG GIA CÔNG CẮT GỌT VÀ MÁY CÔNG CỤ}
     \section{Tiện và các nguyên công liên quan - Turning}
          \subsection{Quá trình tiện}
               \hspace*{0.6cm}\textbf{Nguyên lý cơ bản:} Sử dụng dụng cụ cắt đơn điểm (single point cutting tool) để loại bỏ vật liệu từ một phôi đang quay nhằm tạo ra hình trụ. Quá trình tiện được thực hiện trên máy tiện (lathe). Các nguyên công của quá trình tiện:
               \begin{itemize}
                    \item Tiện mặt đầu (Facing): Gia công phẳng bề mặt ở đầu phôi.
                    \item Tiện chép hình (Contour turning): Tạo ra các hình dạng biên dạng phức tạp.
                    \item Vát mép (Chamfering): Cắt bỏ cạnh sắc ở góc phôi.
                    \item Tiện cắt đứt (Cutoff): Cắt rời một phần chi tiết ra khỏi phôi.
                    \item Tiện ren (Threading): Tạo ra các đường ren vít.
               \end{itemize}
               \begin{figure}[H]
                    \centering
                    \includegraphics[width=0.4\textwidth]{pictures/chapter8/c8_p01.png}
                    \caption{Minh họa quá trình tiện}
               \end{figure}
          \subsection{Các nguyên công liên quan đến tiện}
               \hspace*{0.6cm}Minh họa các nguyên công liên quan đến tiện:
               \begin{figure}[H]
                    \centering
                    \includegraphics[width=0.5\textwidth]{pictures/chapter8/c8_p02.png}
               \end{figure}
               \begin{figure}[H]
                    \centering
                    \includegraphics[width=0.5\textwidth]{pictures/chapter8/c8_p03.png}
               \end{figure}
               \begin{figure}[H]
                    \centering
                    \includegraphics[width=0.5\textwidth]{pictures/chapter8/c8_p04.png}
                    \caption{Minh họa các nguyên công liên quan đến tiện}
               \end{figure}

               \begin{itemize}
                    \item Khoan tâm (Center drilling): Tạo lỗ tâm trên phôi để định vị khi tiện.
                    \item Khoan (Drilling): Tạo lỗ tròn trên phôi.
                    \item Khoét (Boring): Mở rộng và hoàn thiện lỗ đã khoan.
                    \item Phay mặt đầu (Face milling): Gia công bề mặt phẳng trên đầu phôi.
               \end{itemize}
          \subsection{Các sản phẩm máy tiện khác}
               \begin{itemize}
                    \item Máy tiện tháp dao (Turret lathe)
                    \item Máy tiện mâm cặp (Chucking machine)
                    \item Máy tiện thanh (Bar machine)
                    \item Máy tiện tự động trục vít / máy tiện tự động nhiều dao (Automatic screw machine)
                    \item Máy tiện thanh nhiều trục chính (Multiple-spindle bar machine)
               \end{itemize}
     
     \section{Doa - Boring}
          \subsection{So sánh doa và tiện}
               \hspace*{0.6cm}Sự khác nhau giữa doa và tiện:
               \begin{itemize}
                    \item Doa được thực hiện trên đường kính trong của một lỗ đã có sẵn.
                    \item Tiện được thực hiện trên đường kính ngoài của một trụ đã có sẵn.
                    \item Về bản chất, doa là một nguyên công tiện trong (internal turning).
               \end{itemize}
               \hspace*{0.6cm}Máy doa được phân loại thành máy doa ngang hoặc máy doa đứng – tên gọi thể hiện hướng bố trí trục quay của trục chính máy
          \subsection{Máy doa}
               \hspace*{0.6cm}Máy doa đứng, ứng dụng trong gia công các chi tiết gia công lớn, nặng, có tỷ lệ L/D thấp
               \begin{figure}[H]
                    \centering
                    \includegraphics[width=0.4\textwidth]{pictures/chapter8/c8_p05.png}
                    \caption{Minh họa máy doa đứng}
               \end{figure}

     \section{Khoan và các nguyên công liên quan - Drilling}
          \subsection{Định nghĩa}
               \hspace*{0.6cm}Khoan là quá trình
               \begin{itemize}
                    \item Tạo ra một lỗ tròn trên chi tiết gia công.
                    \item Khác với doa (boring) – doa chỉ có thể mở rộng lỗ đã có sẵn.
                    \item Dụng cụ cắt gọi là mũi khoan (drill / drill bit).
                    \item Máy công cụ sử dụng: máy khoan (drill press).
               \end{itemize}
               \hspace*{0.6cm}Khoan bao gồm khoan suốt (through drilling) và khoan cạn (blind drilling)
               \begin{figure}[H]
                    \centering
                    \includegraphics[width=0.6\textwidth]{pictures/chapter8/c8_p06.png}
                    \caption{Minh họa khoan suốt và khoan cạn}
               \end{figure}
          \subsection{Các nguyên công liên quan đến khoan}
               \hspace*{0.6cm}Các nguyên công liên quan đến khoan:
               \begin{itemize}
                    \item (Reaming) – hoàn thiện kích thước và độ nhẵn của lỗ đã khoan
                    \item Ta rô / tạo ren trong (Tapping) – gia công ren trong cho lỗ
                    \item Khoét bậc (Counterboring) – tạo lỗ bậc đáy phẳng để lắp đầu bulông/vít
                    \item Khoét loe côn (Countersinking) – tạo miệng lỗ hình côn để lắp vít đầu côn
                    \item Khoan tâm (Center drilling) – tạo lỗ tâm dẫn hướng cho khoan hoặc tiện
                    \item Khoét mặt đầu (Spot facing) – gia công phẳng bề mặt xung quanh miệng lỗ
               \end{itemize}
               \begin{figure}[H]
                    \centering
                    \includegraphics[width=0.4\textwidth]{pictures/chapter8/c8_p07.png}
               \end{figure}
               \begin{figure}[H]
                    \centering
                    \includegraphics[width=0.4\textwidth]{pictures/chapter8/c8_p08.png}
                    \caption{Minh họa các nguyên công liên quan đến khoan}
               \end{figure}
     \section{Phay - Milling}
          \subsection{Định nghĩa}
               \hspace*{0.6cm}Nguyên công phay là nguyên công gia công cắt gọt, trong đó phôi chuyển động chạy dao đi qua một dụng cụ cắt quay, dụng cụ có nhiều lưỡi cắt.
               \begin{itemize}
                    \item Trục quay của dao vuông góc với hướng chạy dao
                    \item Dụng cụ cắt gọi là dao phay (milling cutter), Các lưỡi cắt được gọi là răng dao (teeth)
                    \item Máy công cụ sử dụng: máy phay (milling machine)
                    \item Là nguyên công cắt gián đoạn
                    \item Nguyên công phay cơ bản tạo ra bề mặt phẳng, tuy nhiên cũng có thể tạo ra nhiều dạng hình học khác ngoài mặt phẳng
               \end{itemize}
               \hspace*{0.6cm}Nguyên công phay có 2 dạng chính là phay biên dạng (Peripheral milling) và phay mặt (Face milling)
               \begin{figure}[H]
                    \centering
                    \includegraphics[width=0.5\textwidth]{pictures/chapter8/c8_p09.png}
                    \caption{Minh họa phay biên dạng và phay mặt}
               \end{figure}
          \subsection{Phay biên dạng - Peripheral milling}
               \subsubsection{Định nghĩa}
                    \begin{itemize}
                         \item Trục của dao phay song song với bề mặt đang gia công
                         \item Các lưỡi cắt nằm trên chu vi ngoài của dao phay
                    \end{itemize}
               \subsubsection{Các loại phay biên dạng}
                    \begin{figure}[H]
                         \centering
                         \includegraphics[width=0.5\textwidth]{pictures/chapter8/c8_p10.png}
                         \caption{Minh họa các loại phay biên dạng}
                    \end{figure}
          \subsection{Phay mặt - Face milling}
               \subsubsection{Định nghĩa}
                    \begin{itemize}
                         \item Trục của dao phay vuông góc với bề mặt đang gia công
                         \item Các lưỡi cắt nằm cả ở mặt đầu và trên chu vi ngoài của dao phay
                    \end{itemize}
               \subsubsection{Các loại phay mặt}
                    \begin{figure}[H]
                         \centering
                         \includegraphics[width=0.6\textwidth]{pictures/chapter8/c8_p11.png}
                         \caption{Minh họa các loại phay mặt}
                    \end{figure}               
     \section{Các nguyên công gia công khác}
          \subsection{Bào ngang và bào dọc - Shaping and Planing}
               \begin{itemize}
                    \item Cả hai quá trình đều tạo ra bề mặt thẳng, phẳng.
                    \item Đều là nguyên công cắt gián đoạn
                    \item Dao chịu tải va đập khi bắt đầu ăn dao vào chi tiết
                    \item Dao tiện điển hình: dao đơn lưỡi bằng thép gió (thép tốc độ cao)
                    \item Tốc độ cắt thấp do chuyển động qua lại (khởi động–dừng liên tục)
               \end{itemize}
               \begin{figure}[h]
                    \centering
                    \begin{minipage}{0.4\textwidth}
                         \centering
                         \includegraphics[width=\linewidth]{pictures/chapter8/c8_p12.png}
                         \caption{Máy bào dọc}
                    \end{minipage}
                    \hfill
                    \begin{minipage}{0.4\textwidth}
                         \centering
                         \includegraphics[width=\linewidth]{pictures/chapter8/c8_p13.png}
                         \caption{Máy bào ngang}
                    \end{minipage}
               \end{figure}
          \subsection{Chuốt - Broaching}
               \subsubsection{Định nghĩa}
                    \begin{itemize}
                         \item Là nguyên công cắt gọt sử dụng dụng cụ cắt nhiều răng gọi là chuốt (broach) để loại bỏ vật liệu từ chi tiết gia công.
                         \item Ưu điểm:
                         \begin{itemize}[label={--}]
                              \item Chất lượng bề mặt tốt
                              \item Độ chính xác cao
                              \item Có thể gia công nhiều dạng hình học khác nhau của chi tiết
                         \end{itemize}
                         \item Dụng cụ cắt gọi là dao chuốt (broach), do hình dạng hình học phức tạp và thường được thiết kế riêng, nên chi phí dụng cụ cao
                    \end{itemize}
                    \begin{figure}[H]
                         \centering
                         \includegraphics[width=0.4\textwidth]{pictures/chapter8/c8_p14.png}
                         \caption{Minh họa nguyên công chuốt}
                    \end{figure}    
          \subsection{Cưa - Sawing}   
               \subsubsection{Định nghĩa}
               \begin{itemize}
                    \item Cưa là nguyên công trong đó cắt một rãnh hẹp trong chi tiết gia công bằng dụng cụ gồm một dãy răng cắt bố trí sát nhau
                    \item Dụng cụ cắt gọi là lưỡi cưa (saw blade)
                    \item Các chức năng chính:
                    \begin{itemize}[label={--}]
                         \item Cắt phôi thành các đoạn nhỏ hơn
                         \item Cắt bỏ các phần vật liệu không cần thiết của chi tiết
                         \item Cắt theo biên dạng của chi tiết phẳng
                    \end{itemize}
               \end{itemize}   
               \begin{figure}[H]
                    \centering
                    \includegraphics[width=0.6\textwidth]{pictures/chapter8/c8_p15.png}
                    \caption{Minh họa nguyên công cưa}
               \end{figure}               

