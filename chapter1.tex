\chapter{CƠ SỞ QUÁ TRÌNH ĐÚC KIM LOẠI}
    \section{Tổng quan về quá trình đúc}
        \subsection{Quá trình đông đặc} 
            \hspace*{0.6cm}Quá trình đông đặc (Solidification processes) là quá trình trong đó vật liệu ban đầu ở trạng thái lỏng hoặc rất dẻo, sau đó tạo hình nhờ sự đông đặc.

            Phân loại theo vật liệu:
            \begin{itemize}
                \item Kim loại
                \item Ceramics (đặc biệt là thủy tinh)
                \item Polyme và composite nền polyme
            \end{itemize}

            Trong học phần này, nội dung tập trung vào \textbf{đúc kim loại}.

            \subsection{Đúc kim loại} 
            \hspace*{0.6cm}Đúc kim loại là quá trình kim loại lỏng chảy vào khuôn và đông đặc theo hình dạng lòng khuôn.  
            Thuật ngữ \textit{casting} vừa chỉ quá trình, vừa chỉ sản phẩm.

            Các bước cơ bản:
            \begin{enumerate}
                \item Nung chảy kim loại
                \item Rót kim loại lỏng vào khuôn
                \item Để kim loại đông đặc
            \end{enumerate}

    \section{Ưu điểm, nhược điểm và ứng dụng}

        \subsection{Ưu điểm của đúc}
            \begin{itemize}
                \item Tạo được hình dạng rất phức tạp
                \item Có thể tạo bề mặt trong
                \item Có thể đạt net shape hoặc near-net shape
                \item Đúc được chi tiết kích thước rất lớn
                \item Phù hợp với sản xuất hàng loạt
            \end{itemize}

        \subsection{Nhược điểm của đúc}
            \begin{itemize}
                \item Cơ tính thường thấp hơn so với rèn
                \item Độ chính xác kích thước và độ nhẵn bề mặt kém (đặc biệt với đúc cát)
                \item Nguy hiểm do kim loại nóng chảy
                \item Gây ô nhiễm môi trường
            \end{itemize}

        \subsection{Sản phẩm đúc điển hình}
            \begin{itemize}
                \item Chi tiết lớn: block động cơ, bánh xe tàu hỏa, khung máy, chuông
                \item Chi tiết nhỏ: trang sức, mão răng, chảo, tượng nhỏ
            \end{itemize}

    \section{Tổng quan công nghệ đúc}

        \subsection{Xưởng đúc -- Foundry}
            \hspace*{0.6cm}Các công đoạn chính trong xưởng đúc:
            \begin{itemize}
                \item Làm khuôn
                \item Nung chảy kim loại
                \item Rót kim loại
                \item Làm sạch vật đúc
            \end{itemize}

        \subsection{Khuôn đúc}
            \hspace*{0.6cm}Khuôn chứa lòng khuôn quyết định hình dạng sản phẩm.  
            \begin{itemize}
                \item Kích thước khuôn phải lớn hơn kích thước chi tiết thật để bù co ngót trong quá trình đông đặc và làm nguội.
                \item Vật liệu khuôn gồm: cát, thạch cao, gốm, kim loại.

            \end{itemize}
            
        \subsection{Khuôn hở và khuôn kín}
            \begin{itemize}
                \item Khuôn hở: đơn giản, độ chính xác thấp
                \item Khuôn kín: có hệ thống rót, kiểm soát dòng chảy tốt hơn
            \end{itemize}
            \begin{figure}[H]
                \centering
                \includegraphics[width=0.75\textwidth]{pictures/chapter1/c01_p01_Molds.png}
                \caption{Khuôn hở và khuôn kín}
            \end{figure}

        \subsection{Phân loại quá trình đúc}
            \begin{enumerate}
                \item \textbf{Khuôn dùng một lần (Expendable mold)}: khuôn bị phá hủy sau khi đúc, tạo hình rất phức tạp
                \item \textbf{Khuôn vĩnh cửu (Permanent mold)}: khuôn kim loại, dùng cho sản lượng lớn
            \end{enumerate}

    \section{Đúc cát -- Cấu tạo và thuật ngữ}
        \begin{figure}[H]
            \centering
            \includegraphics[width=0.75\textwidth]{pictures/chapter1/c01_p02_SandCastingMold.png}
            \caption{Cấu tạo khuôn đúc cát}
        \end{figure}
        \subsection{Thuật ngữ khuôn đúc cát}
            \begin{itemize}
                \item \textbf{Pouring cup (Phễu rót)}: Nơi rót kim loại lỏng vào khuôn, giúp giảm bắn tóe và nhiễu loạn dòng chảy.
                
                \item \textbf{Downsprue (Ống rót đứng)}: Dẫn kim loại lỏng từ phễu rót xuống hệ thống rãnh dẫn nhờ trọng lực.
                
                \item \textbf{Runner (Rãnh dẫn)}: Dẫn và phân phối kim loại lỏng từ ống rót đến lòng khuôn.
                
                \item \textbf{Riser (Đậu ngót)}: Nguồn kim loại lỏng dự trữ để bù co ngót trong quá trình đông đặc.
                
                \item \textbf{Mold (Khuôn)}: Kết cấu cát chứa lòng khuôn và hệ thống rót, quyết định hình dạng chi tiết đúc.
                
                \item \textbf{Flask (Hộp khuôn)}: Khung giữ cố định khuôn cát, không tham gia tạo hình chi tiết.
                
                \item \textbf{Cast metal in cavity (Kim loại trong lòng khuôn)}: Phần kim loại tạo thành chi tiết đúc sau khi đông đặc.
                
                \item \textbf{Core (Lõi)}: Phần đặt trong khuôn để tạo lỗ rỗng hoặc bề mặt trong của chi tiết đúc.
                
                \item \textbf{Cope (Nửa trên khuôn)}: Phần trên của khuôn cát, thường chứa phễu rót và một phần hệ thống rót.
                
                \item \textbf{Drag (Nửa dưới khuôn)}: Phần dưới của khuôn cát, chứa phần chính của lòng khuôn.
                
                \item \textbf{Parting line (Mặt phân khuôn)}: Mặt tiếp xúc giữa cope và drag, cho phép tháo khuôn và lấy vật đúc.
            \end{itemize}

        \subsection{Tạo lòng khuôn bằng mẫu}
            \begin{itemize}
                \item Lòng khuôn được tạo bằng cách đầm cát xung quanh mẫu có hình dạng chi tiết.
                \item Sau khi lấy mẫu ra, khoang rỗng còn lại trong cát chính là hình dạng chi tiết đúc.
                \item Mẫu thường được chế tạo lớn hơn kích thước thực để bù co ngót khi kim loại đông đặc và nguội.
                \item Cát khuôn có độ ẩm và chứa chất kết dính nhằm giữ ổn định hình dạng khuôn.
            \end{itemize}

        \subsection{Lõi -- Core}
            \hspace*{0.6cm}Lõi dùng để tạo bề mặt trong của chi tiết đúc, thường làm bằng cát.

        \subsection{Hệ thống rót}
            \hspace*{0.6cm}Hệ thống rót gồm:
            \begin{itemize}
                \item Phễu rót
                \item Ống rót
                \item Rãnh dẫn
            \end{itemize}

            Mục tiêu là giảm nhiễu loạn, tránh rỗ khí và điền đầy khuôn.

        \subsection{Đậu ngót -- Riser}
            \hspace*{0.6cm}Đậu ngót là nguồn kim loại lỏng dự trữ để bù co ngót khi đông đặc.  
            Đậu ngót phải đông đặc sau chi tiết chính.

    \section{Nung và rót kim loại}

        \subsection{Nung kim loại}
            Kim loại được nung chảy trong lò nung chuyên dụng.
            Tổng nhiệt cần cung cấp:
            \begin{enumerate}
                \item Nâng nhiệt đến nhiệt độ nóng chảy
                \item Nhiệt nóng chảy
                \item Nâng nhiệt kim loại lỏng đến nhiệt độ rót
            \end{enumerate}

        \subsection{Rót kim loại}
            \hspace*{0.6cm}Kim loại phải điền đầy khuôn trước khi đông đặc.  
            Các yếu tố ảnh hưởng: nhiệt độ rót, tốc độ rót, độ nhiễu loạn dòng chảy.

    \section{Đông đặc kim loại}

        \subsection{Kim loại tinh khiết}
            \begin{itemize}
                \item Quá trình đông đặc:
                    \begin{itemize}
                        \item Kim loại tinh khiết đông đặc tại một nhiệt độ không đổi, bằng nhiệt độ nóng chảy. 
                        \item Ngay sau khi rót, thành khuôn gây làm lạnh nhanh, tạo ra một lớp kim loại rắn mỏng (skin) tại bề mặt tiếp xúc.
                        \item Khi thời gian tăng, lớp rắn dày dần và tạo thành một vỏ rắn (shell) bao quanh kim loại lỏng.
                    \end{itemize}
                \item Cấu trúc hạt sau đông đặc:
                    \begin{itemize}
                        \item Gần thành khuôn: các hạt nhỏ, đẳng trục, định hướng ngẫu nhiên.
                        \item Gần tâm vật đúc: các hạt cột lớn, phát triển hướng về trung tâm.
                        \item Nguyên nhân: tốc độ truyền nhiệt khác nhau giữa thành khuôn và lõi vật đúc.
                    \end{itemize}
            \end{itemize}
        \subsection{Hợp kim}
            \begin{itemize}
                \item Hợp kim không đông đặc tại một nhiệt độ duy nhất mà trong một khoảng nhiệt độ.
                \item Trong quá trình đông đặc dễ xảy ra phân ly thành phần.
                \item Cấu trúc tổ chức của vật đúc hợp kim thường không đồng nhất, đặc biệt ở vùng trung tâm.
            \end{itemize}

        \subsection{Thời gian đông đặc}
            \hspace*{0.6cm}Thời gian đông đặc toàn phần (TTS) là thời gian cần thiết để vật đúc đông đặc hoàn toàn 
            sau khi rót kim loại.

        \subsection{Quy luật Chvorinov}
            \[
            TTS = C_m \left(\frac{V}{A}\right)^n
            \]
            Trong đó:
            \begin{itemize}
                \item $V$: thể tích vật đúc
                \item $A$: diện tích bề mặt
                \item $n \approx 2$
                \item $C_m$: hằng số khuôn phụ thuộc vào
                    \begin{itemize}
                        \item Vật liệu khuôn
                        \item Tính chất nhiệt của kim loại đúc
                        \item Nhiệt độ rót so với nhiệt độ nóng chảy
                    \end{itemize}
            \end{itemize}
            Ý nghĩa:
            \begin{itemize}
                \item Vật có tỷ lệ $V/A$ lớn hơn sẽ nguội và đông đặc chậm hơn.
                \item Để bù co ngót, thời gian đông đặc của đậu ngót phải lớn hơn thời gian đông đặc của vật đúc.
                \item Do $C_m$ của riser và vật đúc là như nhau, riser cần được thiết kế có tỷ lệ $V/A$ lớn hơn.
            \end{itemize}

    \section{Co ngót trong quá trình đông đặc và làm nguội}

        \subsection{Các giai đoạn co ngót}
            \hspace*{0.6cm}Co ngót xảy ra qua ba giai đoạn:
            \begin{enumerate}
                \item Co khi kim loại lỏng nguội
                \item Co khi kim loại đông đặc
                \item Co khi kim loại rắn nguội tiếp
            \end{enumerate}
        \subsection{Co ngót khi đông đặc}
            \begin{itemize}
                \item Xảy ra ở hầu hết kim loại do kim loại rắn có khối lượng riêng lớn hơn kim loại lỏng.
                \item Ngoại lệ: gang có hàm lượng cacbon cao, graphit hóa gây giãn nở bù lại co ngót thể tích.
            \end{itemize}

        \subsection{Dư co ngót (Shrinkage Allowance)}

            \begin{itemize}
                \item Lòng khuôn được làm lớn hơn kích thước chi tiết thật để bù co ngót.
                \item Phần tăng thêm gọi là dư co ngót của mẫu (pattern shrinkage allowance).
                \item Do kích thước chi tiết được biểu diễn theo phương tuyến tính, dư co ngót cũng được áp dụng tuyến tính.
            \end{itemize}

    \section{Đông đặc có hướng}

        \subsection{Khái niệm}
            \begin{itemize}
                \item Vùng xa nguồn kim loại lỏng đông đặc trước.
                \item Quá trình đông đặc tiến dần về phía đậu ngót.
                \item Mục tiêu là tránh rỗ co và đảm bảo kim loại lỏng luôn được cấp từ riser.
            \end{itemize}

        \subsection{Điều khiển đông đặc có hướng}
            \begin{itemize}
                \item Áp dụng quy luật Chvorinov trong thiết kế hình dạng và bố trí riser.
                \item Các vùng có tỷ lệ $V/A$ nhỏ nên được bố trí xa riser.
                \item Sử dụng chill để làm đông đặc nhanh cục bộ.
            \end{itemize}

    \section{Chill -- Miếng làm nguội}

        \begin{itemize}
            \item Chill là bộ phận hấp thụ nhiệt cục bộ, có thể là chill trong hoặc chill ngoài.
            \item Dùng để điều khiển trình tự đông đặc theo mong muốn.
            \item Nếu không sử dụng chill, dễ hình thành rỗ co tại các vùng mỏng.
        \end{itemize}

    \section{Thiết kế đậu ngót}

        \begin{itemize}
            \item Đậu ngót là kim loại thừa và sẽ bị cắt bỏ sau khi đúc.
            \item Mục tiêu là giảm thể tích đậu ngót nhưng vẫn đảm bảo chức năng bù co ngót.
            \item Giải pháp là thiết kế đậu ngót có tỷ lệ $V/A$ lớn nhất.
        \end{itemize}
         \hspace*{0.6cm}Điều kiện thiết kế:
        \[
        TTS_{\text{riser}} > TTS_{\text{casting}}
        \]

        Bài toán áp dụng quy luật Chvorinov kết hợp với quan hệ hình học của đậu ngót (thường là hình trụ).

    \section{Phân tích kỹ thuật quá trình rót}
        \hspace*{0.6cm}Dòng kim loại lỏng tuân theo định luật Bernoulli:
        \[
        v = \sqrt{2gh}
        \]
        \[
        Q = vA
        \]
        \[
        TMF = \frac{V}{Q}
        \]

        \begin{itemize}
            \item $v$: vận tốc dòng chảy
            \item $Q$: lưu lượng kim loại lỏng
            \item $TMF$: thời gian điền đầy khuôn
        \end{itemize}
    \section{TÓM TẮT CÁC PHƯƠNG PHÁP ĐÚC}

        \subsection{Đúc cát (Sand Casting)}
        \begin{itemize}
            \item \textbf{Nguyên lý}: Khuôn được tạo bằng cách đầm cát quanh mẫu, tháo mẫu và rót kim loại lỏng vào lòng khuôn.
            \item \textbf{Ưu điểm}: Đúc được hầu như mọi kim loại; không giới hạn kích thước và khối lượng; chi phí khuôn thấp.
            \item \textbf{Hạn chế}: Độ chính xác và chất lượng bề mặt thấp; dung sai rộng; thường cần gia công sau đúc.
            \item \textbf{Sản phẩm}: Block động cơ, thân máy, khung máy, bánh đà, chi tiết lớn.
        \end{itemize}

        \subsection{Đúc khuôn vỏ (Shell Molding)}
        \begin{itemize}
            \item \textbf{Nguyên lý}: Cát trộn nhựa được phủ lên mẫu kim loại nung nóng để tạo vỏ khuôn mỏng và cứng.
            \item \textbf{Ưu điểm}: Độ chính xác kích thước và bề mặt tốt hơn đúc cát; năng suất cao.
            \item \textbf{Hạn chế}: Kích thước chi tiết hạn chế; mẫu và thiết bị đắt.
            \item \textbf{Sản phẩm}: Vỏ hộp số, chi tiết cơ khí chính xác, chi tiết động cơ nhỏ.
        \end{itemize}

        \subsection{Đúc mẫu cháy (Evaporative Pattern / Lost Foam)}
        \begin{itemize}
            \item \textbf{Nguyên lý}: Mẫu xốp được đặt trong khuôn cát và bị hóa hơi khi rót kim loại nóng chảy.
            \item \textbf{Ưu điểm}: Tạo hình rất phức tạp; không cần mặt phân khuôn.
            \item \textbf{Hạn chế}: Mẫu xốp yếu; không kinh tế cho sản lượng nhỏ.
            \item \textbf{Sản phẩm}: Block động cơ ô tô, chi tiết phức tạp.
        \end{itemize}

        \subsection{Đúc khuôn thạch cao (Plaster Mold Casting)}
        \begin{itemize}
            \item \textbf{Nguyên lý}: Khuôn được làm từ thạch cao bao quanh mẫu, sau đó sấy khô và rót kim loại.
            \item \textbf{Ưu điểm}: Độ chính xác cao; bề mặt nhẵn; đúc được thành mỏng.
            \item \textbf{Hạn chế}: Chỉ dùng cho kim loại có nhiệt độ nóng chảy thấp; khuôn dùng một lần.
            \item \textbf{Sản phẩm}: Chi tiết nhôm, kẽm, phụ kiện chính xác, chi tiết trang trí.
        \end{itemize}

        \subsection{Đúc khuôn gốm (Ceramic Mold Casting)}
        \begin{itemize}
            \item \textbf{Nguyên lý}: Sử dụng vật liệu gốm chịu nhiệt để tạo khuôn đúc chính xác.
            \item \textbf{Ưu điểm}: Đúc được chi tiết phức tạp; độ chính xác cao.
            \item \textbf{Hạn chế}: Kích thước chi tiết hạn chế; chi phí khuôn cao.
            \item \textbf{Sản phẩm}: Cánh tuabin nhỏ, chi tiết chịu nhiệt.
        \end{itemize}

        \subsection{Đúc mẫu chảy (Investment Casting)}
        \begin{itemize}
            \item \textbf{Nguyên lý}: Mẫu sáp được phủ gốm, sau đó nung chảy mẫu và rót kim loại vào vỏ khuôn.
            \item \textbf{Ưu điểm}: Độ chính xác và chất lượng bề mặt rất cao; đúc được hầu hết kim loại.
            \item \textbf{Hạn chế}: Chi phí cao; quy trình phức tạp; kích thước chi tiết hạn chế.
            \item \textbf{Sản phẩm}: Cánh tuabin, chi tiết hàng không, y sinh, bánh răng chính xác.
        \end{itemize}

        \subsection{Đúc khuôn kim loại (Permanent Mold Casting)}
        \begin{itemize}
            \item \textbf{Nguyên lý}: Kim loại lỏng được rót vào khuôn kim loại có thể tái sử dụng nhiều lần.
            \item \textbf{Ưu điểm}: Độ chính xác và bề mặt tốt; ít rỗ; năng suất cao.
            \item \textbf{Hạn chế}: Chi phí khuôn cao; hình dạng chi tiết bị giới hạn.
            \item \textbf{Sản phẩm}: Piston, vỏ bơm, chi tiết nhôm.
        \end{itemize}

        \subsection{Đúc áp lực (Die Casting)}
        \begin{itemize}
            \item \textbf{Nguyên lý}: Kim loại lỏng được ép vào khuôn kim loại bằng áp suất cao.
            \item \textbf{Ưu điểm}: Độ chính xác rất cao; bề mặt rất tốt; phù hợp sản xuất hàng loạt.
            \item \textbf{Hạn chế}: Khuôn rất đắt; giới hạn kích thước; chủ yếu dùng kim loại màu.
            \item \textbf{Sản phẩm}: Vỏ điện tử, vỏ động cơ nhỏ, linh kiện ô tô.
        \end{itemize}

        \subsection{Đúc ly tâm (Centrifugal Casting)}
        \begin{itemize}
            \item \textbf{Nguyên lý}: Kim loại lỏng được đổ vào khuôn quay, nhờ lực ly tâm tạo hình.
            \item \textbf{Ưu điểm}: Chất lượng cao; ít khuyết tật; cơ tính tốt.
            \item \textbf{Hạn chế}: Chỉ tạo được chi tiết dạng tròn hoặc ống; thiết bị đắt.
            \item \textbf{Sản phẩm}: Ống, bạc lót, vòng bi, ống chịu áp lực.
        \end{itemize}
           
