\chapter{CƠ SỞ CỦA GIA CÔNG BIẾN DẠNG KIM LOẠI}
    \section{Tổng quan về tạo hình kim loại}

        \hspace*{0.6cm}Tạo hình kim loại là nhóm lớn các quá trình chế tạo trong đó hình dạng của phôi kim loại được thay đổi bằng cách gây ra biến dạng dẻo.

        \begin{itemize}
            \item Dụng cụ tạo hình (khuôn – die) tạo ra ứng suất vượt quá giới hạn chảy của vật liệu.
            \item Kim loại biến dạng dẻo và nhận hình dạng theo hình học của khuôn.
            \item Các dạng ứng suất thường gặp trong tạo hình kim loại:
            \begin{itemize}
                \item Ứng suất nén (cán, rèn, ép đùn)
                \item Ứng suất kéo (kéo giãn)
                \item Ứng suất uốn (uốn tấm)
                \item Ứng suất trượt (cắt)
            \end{itemize}
        \end{itemize}
        \hspace*{0.6cm}Tính chất vật liệu trong tạo hình kim loại
        \begin{itemize}
            \item Vật liệu dùng cho tạo hình cần có giới hạn chảy thấp để dễ bắt đầu biến dạng dẻo
            và độ dẻo cao để chịu được biến dạng lớn mà không bị phá hủy.

            \item Các tính chất này phụ thuộc mạnh vào nhiệt độ gia công:
            khi nhiệt độ tăng, giới hạn chảy giảm và độ dẻo tăng.

            \item Ngoài nhiệt độ, khả năng tạo hình còn bị ảnh hưởng bởi
            tốc độ biến dạng và ma sát giữa phôi và dụng cụ.
        \end{itemize}

    \section{Phân loại các quá trình tạo hình kim loại}

        Các quá trình tạo hình kim loại được chia thành hai nhóm chính: biến dạng khối và gia công kim loại tấm.

        \subsection{Biến dạng khối}

        \begin{itemize}
            \item Đặc trưng bởi mức độ biến dạng lớn và thay đổi hình dạng đáng kể.
            \item Phôi có tỷ lệ diện tích bề mặt trên thể tích nhỏ.
            \item Phôi ban đầu thường có hình dạng đơn giản như billet trụ hoặc thanh chữ nhật.
        \end{itemize}

        Các quá trình biến dạng khối tiêu biểu:
        \begin{itemize}
            \item Cán (Rolling)
            \item Rèn (Forging)
            \item Ép đùn (Extrusion)
            \item Kéo dây và thanh (Wire and Bar Drawing)
        \end{itemize}
        \begin{figure}[H]
            \centering
            \includegraphics[width=0.5\textwidth]{pictures/chapter3/c03_p01_RollingandForging.png}
            \includegraphics[width=0.5\textwidth]{pictures/chapter3/c03_p02_ExtrusionandDrawing.png}
            \caption{Các quá trình biến dạng khối: (a) cán, (b) rèn, (c) ép đùn, (d) kéo dây và thanh}
        \end{figure}

        \subsection{Gia công kim loại tấm}

        \begin{itemize}
            \item Thực hiện trên kim loại dạng tấm, dải hoặc cuộn.
            \item Phôi có tỷ lệ diện tích bề mặt trên thể tích lớn.
            \item Thường được thực hiện trên máy ép, nên còn gọi là gia công ép (pressworking).
            \item Sản phẩm tạo ra gọi là chi tiết dập (stampings).
            \item Dụng cụ chính gồm chày (punch) và cối (die).
        \end{itemize}

        Các quá trình gia công kim loại tấm tiêu biểu:
        \begin{itemize}
            \item \textbf{Uốn (Bending):}
            làm tấm kim loại biến dạng theo hình dạng khuôn,
            ứng suất chủ yếu là uốn (kéo và nén).

            \item \textbf{Dập vuốt sâu (Deep drawing):}
            biến tấm phẳng thành chi tiết rỗng như cốc hoặc hộp,
            kim loại chảy theo phương chiều sâu của khuôn.
            \begin{figure}[H]
                \centering
                \includegraphics[width=0.6\textwidth]{pictures/chapter3/c03_p03_BendingandDeepDrawing.png}
                \caption{Các quá trình gia công kim loại tấm: (a) uốn, (b) dập vuốt sâu}
            \end{figure}
            \item \textbf{Cắt kim loại tấm (Shearing):} quá trình cắt dựa trên ứng suất trượt do chày tác dụng lên tấm kim loại.Bao gồm giai đoạn chày bắt đầu tiếp xúc với tấm (1) và giai đoạn sau khi cắt hoàn toàn.(2)
            \begin{figure}[H]
                \centering
                \includegraphics[width=0.6\textwidth]{pictures/chapter3/c03_p04_Shearing.png}
                \caption{Quá trình cắt kim loại tấm}
            \end{figure}
        \end{itemize}
    \section{Ứng suất của vật liệu trong tạo hình kim loại}

        \begin{itemize}
            \item Trong tạo hình kim loại, vùng biến dạng dẻo của đường cong ứng suất–biến dạng là vùng quan trọng nhất.
            \item Ứng suất của vật liệu trong vùng dẻo được mô tả bằng đường cong chảy.
        

            \[
            \sigma = K \varepsilon^n
            \]
            với $K$ là hệ số bền và $n$ là số mũ hóa bền biến dạng.

            \item Đường cong chảy được xác định từ
            ứng suất thực và biến dạng thực.
        \end{itemize}

        \subsection{Ứng suất chảy}

            \begin{itemize}
                \item Ứng suất chảy là giá trị ứng suất tức thời cần thiết để tiếp tục làm vật liệu biến dạng dẻo.
                \item Ở nhiệt độ thường, ứng suất chảy tăng theo mức biến dạng do hiện tượng hóa bền biến dạng.
            \end{itemize}

            \[
            Y_f = K \varepsilon^n
            \]

        \subsection{Ứng suất chảy trung bình}

            \hspace*{0.6cm}Ứng suất chảy trung bình được sử dụng để tính toán lực và công suất trong các quá trình tạo hình:

            \[
            \bar{Y}_f = \frac{K \varepsilon^n}{n+1}
            \]

    \section{Ảnh hưởng của nhiệt độ trong tạo hình kim loại}

        \hspace*{0.6cm}Nhiệt độ có ảnh hưởng rất lớn đến ứng xử của vật liệu khi tạo hình.

        \begin{itemize}
            \item Khi nhiệt độ tăng:
            \begin{itemize}
                \item Hệ số bền $K$ giảm
                \item Hệ số hóa bền $n$ giảm
                \item Độ dẻo của vật liệu tăng
            \end{itemize}
            \item Nhờ đó, lực và công suất cần thiết cho quá trình tạo hình giảm.
        \end{itemize}
        \begin{table}[H]
            \caption{So sánh các phương pháp tạo hình kim loại ở các vùng nhiệt độ khác nhau}
            \centering
            \renewcommand{\arraystretch}{1.3}
            \begin{tabularx}{\textwidth}{|
            >{\raggedright\arraybackslash}p{3.5cm}|
            >{\raggedright\arraybackslash}X|
            >{\raggedright\arraybackslash}X|
            >{\raggedright\arraybackslash}X|}
            \hline
            \textbf{Tiêu chí} 
            & \textbf{Gia công nguội (Cold Working)} 
            & \textbf{Gia công ấm (Warm Working)} 
            & \textbf{Gia công nóng (Hot Working)} \\
            \hline

            Nhiệt độ gia công 
            & Nhiệt độ phòng hoặc hơi cao hơn 
            & Trên nhiệt độ phòng, dưới nhiệt độ kết tinh lại (xấp xỉ $0.3T_m$) 
            & Trên nhiệt độ kết tinh lại, thường $>0.5T_m$ \\
            \hline

            Đặc điểm chính 
            & Sản xuất hàng loạt, near net shape hoặc net shape 
            & Trung gian giữa gia công nguội và gia công nóng 
            & Cho phép biến dạng dẻo rất lớn \\
            \hline

            Lực và công suất 
            & Lớn 
            & Nhỏ hơn gia công nguội 
            & Nhỏ nhất \\
            \hline

            Độ chính xác và bề mặt 
            & Độ chính xác cao, bề mặt tốt 
            & Trung bình 
            & Độ chính xác thấp, bề mặt kém do ôxy hóa \\
            \hline

            Ảnh hưởng cơ tính 
            & Hóa bền biến dạng, tăng độ bền và độ cứng 
            & Ít hóa bền hơn gia công nguội 
            & Không hóa bền, cơ tính gần đẳng hướng \\
            \hline

            Khả năng tạo hình 
            & Bị giới hạn bởi độ dẻo và hóa bền 
            & Tạo hình phức tạp hơn gia công nguội 
            & Tạo hình rất phức tạp \\
            \hline

            Yêu cầu nhiệt 
            & Không cần gia nhiệt 
            & Phải gia nhiệt phôi 
            & Phải gia nhiệt ở nhiệt độ cao \\
            \hline

            Hạn chế chính 
            & Lực lớn, có thể cần ủ trung gian 
            & Cần gia nhiệt phôi 
            & Tốn năng lượng, ôxy hóa, mòn khuôn nhanh \\
            \hline
            \end{tabularx}

        \end{table}

    \section{Độ nhạy với tốc độ biến dạng}

        \subsection{Khái niệm}

            \hspace*{0.6cm}Về mặt lý thuyết, trong gia công nóng, kim loại được xem như vật liệu dẻo hoàn toàn,
            với số mũ hóa bền biến dạng $n = 0$.
            Khi đó, sau khi đạt ứng suất chảy, kim loại sẽ tiếp tục biến dạng ở cùng mức ứng suất.

            Tuy nhiên, trong thực tế, đặc biệt ở nhiệt độ cao,
            một hiện tượng bổ sung xuất hiện trong quá trình biến dạng,
            đó là độ nhạy với tốc độ biến dạng.

        \subsection{Tốc độ biến dạng}

            \hspace*{0.6cm}Trong tạo hình kim loại, tốc độ biến dạng liên quan trực tiếp đến tốc độ biến dạng của quá trình.

            \[
            \dot{\varepsilon} = \frac{v}{h}
            \]

            trong đó:
            \begin{itemize}
                \item $\dot{\varepsilon}$ là tốc độ biến dạng thực,
                \item $v$ là vận tốc của chày hoặc bộ phận tạo hình,
                \item $h$ là chiều cao tức thời của phôi đang biến dạng.
            \end{itemize}

        \subsection{Đánh giá tốc độ biến dạng}

            \hspace*{0.6cm}Trong thực tế, việc xác định tốc độ biến dạng gặp nhiều khó khăn do:
            \begin{itemize}
                \item Hình dạng phôi phức tạp,
                \item Sự phân bố không đồng đều của tốc độ biến dạng trong các vùng khác nhau của chi tiết.
            \end{itemize}

            Trong một số quá trình tạo hình kim loại,
            tốc độ biến dạng có thể đạt tới $10^3~\text{s}^{-1}$ hoặc lớn hơn.

        \subsection{Ảnh hưởng của tốc độ biến dạng đến ứng suất chảy}

            \hspace*{0.6cm}Ứng suất chảy phụ thuộc vào nhiệt độ,
            và ở nhiệt độ gia công nóng, còn phụ thuộc vào tốc độ biến dạng.

            \begin{itemize}
                \item Khi tốc độ biến dạng tăng, lực cản đối với biến dạng cũng tăng.
                \item Hiện tượng này được gọi là độ nhạy với tốc độ biến dạng.
            \end{itemize}

        \subsection{Mô hình độ nhạy tốc độ biến dạng}

            \hspace*{0.6cm}Mối quan hệ giữa ứng suất chảy và tốc độ biến dạng được biểu diễn bằng:

            \[
            Y_f = C\,\dot{\varepsilon}^{m}
            \]

            trong đó:
            \begin{itemize}
                \item $C$ là hằng số bền (không trùng với hệ số $K$ trong đường cong chảy),
                \item $m$ là số mũ nhạy với tốc độ biến dạng.
            \end{itemize}

        \subsection{Nhận xét về độ nhạy tốc độ biến dạng}

            \begin{itemize}
                \item Khi nhiệt độ tăng, hằng số $C$ giảm và số mũ $m$ tăng.
                \item Ở nhiệt độ phòng, ảnh hưởng của tốc độ biến dạng gần như không đáng kể,
                và đường cong chảy là đủ để mô tả ứng xử vật liệu.
                \item Khi nhiệt độ tăng cao, tốc độ biến dạng trở thành yếu tố quan trọng
                trong việc xác định ứng suất chảy của kim loại.
            \end{itemize}

    \section{Ma sát và bôi trơn trong tạo hình kim loại}

        \subsection{Ma sát}

        Ma sát trong tạo hình kim loại là yếu tố không mong muốn vì:
        \begin{itemize}
            \item Cản trở dòng chảy kim loại.
            \item Làm tăng lực và công suất tạo hình.
            \item Gây mòn nhanh dụng cụ.
        \end{itemize}

        Ma sát đặc biệt nghiêm trọng trong các quá trình gia công nóng.

        \subsection{Bôi trơn}

        Bôi trơn được sử dụng để giảm tác hại của ma sát trong quá trình tạo hình.

        \begin{itemize}
            \item Giảm lực và công suất.
            \item Giảm mòn dụng cụ.
            \item Cải thiện chất lượng bề mặt chi tiết.
            \item Giúp tản nhiệt cho dụng cụ.
        \end{itemize}

        Các yếu tố cần xem xét khi lựa chọn chất bôi trơn:
        \begin{itemize}
            \item Loại quá trình tạo hình.
            \item Gia công nóng hay nguội.
            \item Vật liệu phôi.
            \item Tính tương thích hóa học với dụng cụ.
            \item Chi phí và khả năng áp dụng.
        \end{itemize}
