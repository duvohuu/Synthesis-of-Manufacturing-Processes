\chapter{GIA CÔNG CẮT GỌT KIM LOẠI}
    \section{Tổng quan về công nghệ gia công cắt gọt}
        \hspace*{0.6cm}Gia công cắt gọt (Machining) là một nhóm các quá trình tạo hình bằng cách \textbf{loại bỏ vật liệu} từ phôi ban đầu để thu được hình dạng, kích thước và độ chính xác mong muốn.

        \begin{itemize}
            \item Gia công bằng dao cắt sắc (Turning, Milling, Drilling)
            \item Gia công bằng hạt mài (Grinding)
            \item Gia công phi truyền thống (EDM, Laser, Waterjet, ...)
        \end{itemize}

        Đặc trưng của gia công cắt gọt là quá trình \textbf{biến dạng cắt (shear deformation)} tạo phoi và bề mặt mới.

    \section{Ưu nhược điểm của gia công cắt gọt}
        \hspace*{0.6cm}\textbf{Ưu điểm:}
        \begin{itemize}
            \item Gia công được nhiều loại vật liệu, đặc biệt là kim loại
            \item Tạo hình chi tiết phức tạp: ren, lỗ chính xác, bề mặt phẳng
            \item Độ chính xác kích thước và độ nhẵn bề mặt cao
        \end{itemize}

        \textbf{Nhược điểm:}
        \begin{itemize}
            \item Lãng phí vật liệu do tạo phoi
            \item Thời gian gia công dài so với các quá trình tạo hình khác
        \end{itemize}

        Gia công cắt gọt thường là \textbf{công đoạn cuối} sau đúc, rèn, cán.

    \section{Các nguyên công gia công chủ yếu}
        \subsection{Gia công truyền thống (Conventional Machining)}
        \hspace*{0.6cm}Đặc trưng bởi việc sử dụng \textbf{dao cắt sắc} để loại bỏ vật liệu dưới dạng phoi.

        \begin{itemize}
            \item Tiện và các nguyên công liên quan (Turning and related operations)
            \item Khoan và các nguyên công liên quan (Drilling and related operations)
            \item Phay (Milling)
            \item Các nguyên công gia công khác (Other machining operations)
        \end{itemize}

        \subsection{Gia công bằng hạt mài (Abrasive Processes)}
        \hspace*{0.6cm}Vật liệu được loại bỏ nhờ \textbf{các hạt mài cứng} có kích thước nhỏ.

        \begin{itemize}
            \item Mài (Grinding operations)
            \item Các quá trình gia công mài khác (Other abrasive processes)
        \end{itemize}

        \subsection{Gia công phi truyền thống (Nontraditional Machining)}
        \hspace*{0.6cm}Không sử dụng dao cắt truyền thống, vật liệu được loại bỏ bằng các dạng năng lượng khác.

        \begin{itemize}
            \item Gia công bằng năng lượng cơ học (Mechanical energy processes)
            \item Gia công điện hóa (Electrochemical machining)
            \item Gia công bằng năng lượng nhiệt (Thermal energy processes)
            \item Gia công hóa học (Chemical machining)
        \end{itemize}


        \subsection{Dụng cụ cắt}
        \begin{itemize}
            \item \textbf{Dao một lưỡi cắt}: tiện
            \item \textbf{Dao nhiều lưỡi cắt}: khoan, phay
        \end{itemize}

        Các bề mặt chính của dao: mặt trước (rake face), mặt sau (flank), mũi dao.

        \subsection{Chế độ cắt}
            Ba thông số cơ bản:
            \begin{itemize}
                \item Tốc độ cắt $v$
                \item Lượng chạy dao $f$
                \item Chiều sâu cắt $d$
            \end{itemize}

            Tốc độ bóc vật liệu:
            \[
            RMR = v f d
            \]

        \subsection{Cắt thô và cắt tinh}
            \begin{itemize}
                \item \textbf{Cắt thô}: bóc nhiều vật liệu, $f$ và $d$ lớn, $v$ nhỏ
                \item \textbf{Cắt tinh}: hoàn thiện kích thước, $f$ và $d$ nhỏ, $v$ lớn
            \end{itemize}

    \section{Mô hình cắt trực giao}
        Mô hình 2D dùng để phân tích cơ học cắt gọt.

    \subsection{Tỷ số chiều dày phoi}
    \[
    r = \frac{t_o}{t_c} < 1
        \]

        \subsection{Góc mặt trượt}
        \[
        \tan \phi = \frac{r \cos \alpha}{1 - r \sin \alpha}
        \]

        \subsection{Biến dạng trượt}
        \[
        \gamma = \tan(\phi - \alpha) + \cot \phi
        \]

        \subsection{Các dạng phoi}
        \begin{enumerate}
            \item Phoi rời – vật liệu giòn, tốc độ thấp
            \item Phoi liên tục – vật liệu dẻo, tốc độ cao
            \item Phoi liên tục có lẹo dao (BUE)
            \item Phoi răng cưa – vật liệu khó gia công, tốc độ cao
        \end{enumerate}
        \begin{figure}[h!]
            \centering
            \begin{subfigure}{0.45\textwidth}
                \centering
                \includegraphics[height=4cm]{pictures/chapter4/c04_p01_DiscontinousChip.png}
                \caption{Phoi rời}
            \end{subfigure}
            \begin{subfigure}{0.45\textwidth}
                \centering
                \includegraphics[height=4cm]{pictures/chapter4/c04_p02_ContinousChip.png}
                \caption{Phoi liên tục}
            \end{subfigure}

            \vspace{0.5cm}

            \begin{subfigure}{0.45\textwidth}
                \centering
                \includegraphics[height=4cm]{pictures/chapter4/c04_p03_ContinousBLEChip.png}
                \caption{Phoi liên tục có lẹo dao}
            \end{subfigure}
            \hfill
            \begin{subfigure}{0.45\textwidth}
                \centering
                \includegraphics[height=4cm]{pictures/chapter4/c04_p04_SerratedChip.png}
                \caption{Phoi răng cưa}
            \end{subfigure}

            \caption{Các dạng phoi trong gia công cắt gọt kim loại}
            \label{fig:chip_types}
        \end{figure}

        \section{Lực cắt trong gia công cắt gọt}

            \subsection{Các lực tác dụng lên phoi}
            \hspace*{0.6cm}Trong quá trình gia công cắt gọt, dao cắt tác dụng lực lên phoi thông qua hai vùng chính: vùng tiếp xúc dao--phoi và vùng trượt.

            Các lực tác dụng lên phoi bao gồm:
            \begin{itemize}
                \item Lực ma sát $F$ và phản lực pháp tuyến $N$ tại mặt trước của dao
                \item Lực trượt $F_s$ và phản lực pháp tuyến $F_n$ trên mặt trượt
            \end{itemize}

            \subsection{Hợp lực tác dụng lên phoi}
            \hspace*{0.6cm}Các lực riêng lẻ có thể được tổng hợp thành các hợp lực:

            \begin{itemize}
                \item Hợp lực $R$ là tổng véc-tơ của lực ma sát $F$ và phản lực $N$
                \item Hợp lực $R'$ là tổng véc-tơ của lực trượt $F_s$ và phản lực $F_n$
            \end{itemize}

            Do phoi ở trạng thái cân bằng lực:
            \begin{itemize}
                \item $R'$ có độ lớn bằng $R$
                \item $R'$ ngược chiều với $R$
                \item $R'$ cùng phương (thẳng hàng) với $R$
            \end{itemize}

            \subsection{Hệ số ma sát trong gia công}
            \hspace*{0.6cm}Hệ số ma sát giữa dao và phoi được xác định bởi:
            \[
            \mu = \frac{F}{N}
            \]

            Góc ma sát $\beta$ liên hệ với hệ số ma sát:
            \[
            \mu = \tan \beta
            \]

            \subsection{Ứng suất trượt trên mặt trượt}
            \hspace*{0.6cm}Ứng suất trượt tác dụng trên mặt trượt được xác định bởi:
            \[
            \tau = \frac{F_s}{A_s}
            \]

            trong đó diện tích mặt trượt:
            \[
            A_s = \frac{t_o w}{\sin \phi}
            \]

            Trong gia công cắt gọt, ứng suất trượt:
            \[
            \tau \approx S
            \]
            với $S$ là giới hạn bền trượt của vật liệu phôi trong điều kiện cắt.

            \subsection{Lực cắt và lực đẩy}
            \hspace*{0.6cm}Các lực $F$, $N$, $F_s$ và $F_n$ không thể đo trực tiếp trong thực tế.

            Các lực có thể đo được trên dao cắt là:
            \begin{itemize}
                \item Lực cắt $F_c$
                \item Lực đẩy (lực hướng tâm) $F_t$
            \end{itemize}

            \subsection{Quan hệ giữa các lực trong gia công cắt gọt}
            Các lực không đo được được xác định thông qua các lực đo được theo các biểu thức:

            \[
            \begin{aligned}
            F   &= F_c \sin \alpha + F_t \cos \alpha \\
            N   &= F_c \cos \alpha - F_t \sin \alpha \\
            F_s &= F_c \cos \phi - F_t \sin \phi \\
            F_n &= F_c \sin \phi + F_t \cos \phi
            \end{aligned}
            \]

            Dựa vào các lực này, có thể xác định ứng suất trượt và hệ số ma sát trong quá trình cắt.

            \subsection{Phương trình Merchant}
            Theo giả thuyết Merchant, trong các khả năng có thể xảy ra của mặt trượt, vật liệu sẽ chọn góc trượt sao cho năng lượng tiêu hao là nhỏ nhất.

            Phương trình Merchant:
            \[
            \phi = 45^\circ + \frac{\alpha}{2} - \frac{\beta}{2}
            \]

            Phương trình được xây dựng cho mô hình cắt trực giao, nhưng có thể mở rộng áp dụng cho gia công ba chiều.

            \subsection{Ý nghĩa của phương trình Merchant}
            Từ phương trình Merchant có thể rút ra các kết luận quan trọng:

            \begin{itemize}
                \item Tăng góc trước của dao $\alpha$ làm tăng góc trượt $\phi$
                \item Giảm góc ma sát $\beta$ (giảm hệ số ma sát) làm tăng $\phi$
            \end{itemize}

            \subsection{Ảnh hưởng của góc trượt lớn}
            Góc trượt $\phi$ càng lớn thì:
            \begin{itemize}
                \item Diện tích mặt trượt càng nhỏ
                \item Lực trượt và lực cắt giảm
                \item Công suất cắt giảm
                \item Nhiệt cắt giảm
            \end{itemize}

            Do đó, tăng góc trượt là mục tiêu quan trọng trong thiết kế dao và lựa chọn điều kiện cắt.


    \section{Công suất và năng lượng trong gia công cắt gọt}

        \subsection{Công suất cắt}
        Gia công cắt gọt kim loại là một quá trình tiêu thụ năng lượng, do đó luôn yêu cầu công suất.

        Công suất cắt được xác định theo biểu thức:
        \[
        P_c = F_c \, v
        \]

        trong đó:
        \begin{itemize}
            \item $P_c$ – công suất cắt
            \item $F_c$ – lực cắt
            \item $v$ – tốc độ cắt
        \end{itemize}

        Công thức cho thấy công suất cắt phụ thuộc trực tiếp vào lực cắt và tốc độ cắt.

        \subsection{Công suất cắt theo hệ đơn vị Anh–Mỹ}
        Trong hệ đơn vị Anh–Mỹ, công suất thường được biểu diễn dưới dạng mã lực (horsepower).

        \[
        HP_c = \frac{F_c \, v}{33{,}000}
        \]

        trong đó:
        \begin{itemize}
            \item $HP_c$ – công suất cắt (hp)
            \item $33{,}000$ ft$\cdot$lb/min tương đương với 1 hp
        \end{itemize}

        \subsection{Công suất tổng của máy công cụ}
        Do tồn tại tổn hao cơ khí, công suất máy cung cấp luôn lớn hơn công suất thực sự dùng để cắt.

        Quan hệ giữa công suất cắt và công suất máy:
        \[
        P_g = \frac{P_c}{E}
        \qquad \text{hoặc} \qquad
        HP_g = \frac{HP_c}{E}
        \]

        trong đó:
        \begin{itemize}
            \item $P_g$, $HP_g$ – công suất tổng của máy
            \item $E$ – hiệu suất cơ khí của máy công cụ
        \end{itemize}

        Giá trị hiệu suất cơ khí điển hình của máy công cụ:
        \[
        E \approx 90\%
        \]

        \subsection{Công suất riêng trong gia công}
        Để đánh giá và so sánh các quá trình gia công, công suất thường được quy đổi theo đơn vị thể tích vật liệu bóc đi, gọi là công suất riêng.

        \[
        P_u = \frac{P_c}{R_{MR}}
        \qquad \text{hoặc} \qquad
        HP_u = \frac{HP_c}{R_{MR}}
        \]

        trong đó:
        \begin{itemize}
            \item $P_u$, $HP_u$ – công suất riêng
            \item $R_{MR}$ – tốc độ bóc vật liệu (Material Removal Rate)
        \end{itemize}

        \subsection{Năng lượng riêng trong gia công}
        Công suất riêng còn được gọi là năng lượng riêng của quá trình gia công.

        \[
        U = P_u = \frac{P_c}{R_{MR}} = \frac{F_c \, v}{v t_o w}
        \]

        trong đó:
        \begin{itemize}
            \item $U$ – năng lượng riêng
            \item $t_o$ – chiều dày lớp cắt trước gia công
            \item $w$ – chiều rộng cắt
        \end{itemize}

        Đơn vị thường dùng của năng lượng riêng:
        \[
        \text{J/mm}^3 \quad \text{hoặc} \quad \text{N·m/mm}^3
        \]

        Năng lượng riêng biểu thị lượng năng lượng cần thiết để bóc một đơn vị thể tích vật liệu.

        \subsection{Nhiệt cắt}
        Khoảng \textbf{98\% năng lượng} biến thành nhiệt tại vùng dao–phoi.

        \textbf{Hệ quả:}
        \begin{itemize}
            \item Giảm tuổi bền dao
            \item Phoi nóng gây nguy hiểm
            \item Sai lệch kích thước do giãn nở nhiệt
        \end{itemize}

        \subsubsection{Công thức Cook}
        \[
        T = 0.4 \frac{U}{\rho C}
        \left( \frac{v t_o}{K} \right)^{0.333}
        \]

        \subsubsection{Quan hệ thực nghiệm}
        \[
        T = K v^m
        \]