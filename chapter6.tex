\chapter{GIA CÔNG KIM LOẠI BỘT (POWDER METALLURGY - PM)}

\section{Giới thiệu về luyện kim bột}

Luyện kim bột (Powder Metallurgy - PM) là công nghệ gia công kim loại trong đó các chi tiết được sản xuất từ bột kim loại.

\subsection*{Quy trình sản xuất PM thông thường}

\begin{enumerate}
    \item \textbf{Ép (Pressing)}:
    \begin{itemize}
        \item Bột được nén vào khuôn để tạo thành hình dạng mong muốn.
        \item Tạo ra phôi xanh (green compact).
        \item Được thực hiện trong máy ép sử dụng chày và khuôn (punch-and-die).
    \end{itemize}
    
    \item \textbf{Thiêu kết (Sintering)}:
    \begin{itemize}
        \item Phôi xanh được gia nhiệt để liên kết các hạt thành khối rắn chắc, cứng.
        \item Nhiệt độ thiêu kết nằm dưới điểm nóng chảy của kim loại.
    \end{itemize}
\end{enumerate}

\section{Tại sao luyện kim bột quan trọng}

\subsection*{Ưu điểm chính của PM}

\begin{itemize}
    \item \textbf{Sản xuất hàng loạt với hình dạng chính xác}:
    \begin{itemize}
        \item Chi tiết PM có thể được sản xuất hàng loạt đạt hình dạng chính xác (net shape) hoặc gần chính xác (near net shape).
        \item Loại bỏ hoặc giảm thiểu nhu cầu thêm một nguyên công.
    \end{itemize}
    
    \item \textbf{Lãng phí vật liệu rất ít}:
    \begin{itemize}
        \item Quá trình PM lãng phí rất ít vật liệu.
        \item Khoảng 97\% bột ban đầu được chuyển thành sản phẩm.
    \end{itemize}
    
    \item \textbf{Chi tiết xốp có độ xốp xác định}:
    \begin{itemize}
        \item Chi tiết PM có thể được làm với mức độ xốp xác định.
        \item Sản xuất các chi tiết kim loại xốp.
        \item Ứng dụng: bộ lọc, ổ bi tự bôi trơn, bánh răng.
    \end{itemize}
\end{itemize}

\subsection*{Các lý do khác tại sao PM quan trọng}

\begin{itemize}
    \item \textbf{Gia công kim loại khó}:
    \begin{itemize}
        \item Một số kim loại khó chế tạo bằng các phương pháp khác có thể được tạo hình bằng luyện kim bột.
        \item Ví dụ: dây tóc đèn Tungsten được làm bằng PM.
    \end{itemize}
    
    \item \textbf{Tạo hợp kim và cermet đặc biệt}:
    \begin{itemize}
        \item Một số tổ hợp hợp kim và cermet được tạo bằng PM không thể sản xuất bằng các phương pháp khác.
    \end{itemize}
    
    \item \textbf{Kiểm soát kích thước tốt}:
    \begin{itemize}
        \item PM có khả năng kiểm soát kích thước tốt so với hầu hết các quá trình đúc.
    \end{itemize}
    
    \item \textbf{Tự động hóa sản xuất}:
    \begin{itemize}
        \item Các phương pháp sản xuất PM có thể được tự động hóa để sản xuất kinh tế.
    \end{itemize}
\end{itemize}

\section{Hạn chế và nhược điểm}

\begin{itemize}
    \item Chi phí dụng cụ và thiết bị cao.
    \item Bột kim loại đắt tiền.
    \item Gặp vấn đề trong việc bảo quản và xử lý bột kim loại: bị xuống cấp theo thời gian, nguy cơ cháy nổ với một số kim loại nhất định.
    \item Hạn chế về hình học chi tiết do bột kim loại không dễ dàng chảy theo bên trong khuôn trong quá trình ép.
    \item Sự biến thiên mật độ khắp chi tiết có thể là vấn đề, đặc biệt đối với các hình học phức tạp.
\end{itemize}

\section{Vật liệu gia công PM}

\begin{itemize}
    \item Tỷ trọng lớn nhất của kim loại là hợp kim sắt, thép và nhôm.
    \item Các kim loại PM khác bao gồm: đồng, niken và kim loại chịu lửa như molypden và tungsten.
    \item Các carbide kim loại (hợp chất của carbon với kim loại) như tungsten carbide (WC) thường được bao gồm trong phạm vi của luyện kim bột.
\end{itemize}

\section{Bột kỹ thuật (Engineering Powders)}

\subsection*{Định nghĩa}
Bột (powder) có thể được định nghĩa là chất rắn dạng hạt được chia nhỏ.

\subsection*{Thành phần}
\begin{itemize}
    \item Bột kỹ thuật bao gồm kim loại và gốm.
\end{itemize}

\subsection*{Các đặc trưng hình học của bột kỹ thuật}
\begin{itemize}
    \item Kích thước hạt và phân bố kích thước hạt.
    \item Hình dạng hạt và cấu trúc bên trong.
    \item Diện tích bề mặt.
\end{itemize}

\section{Đo kích thước hạt}

\subsection*{Phương pháp phổ biến nhất}
Phương pháp phổ biến nhất sử dụng các màng sàng với kích thước lỗ khác nhau.

\subsection*{Số lưới (Mesh count)}
\begin{itemize}
    \item Số lưới đề cập đến số lỗ trên mỗi inch chiều dài của màng sàng.
    \item Ví dụ: Số lưới 200 có nghĩa là có 200 lỗ trên mỗi inch chiều dài.
    \item Vì màng sàng là hình vuông, số lưới bằng nhau ở cả hai hướng, và tổng số lỗ trên mỗi inch vuông là $200^2 = 40,000$.
    \item Số lưới cao hơn = kích thước hạt nhỏ hơn.
\end{itemize}

\begin{figure}[H]
    \centering
    \includegraphics[width=0.6\textwidth]{pictures/chapter6/c06_p01.png}
    \caption{Minh họa số lưới và kích thước hạt}
\end{figure}
\begin{figure}[H]
    \centering
    \includegraphics[width=0.6\textwidth]{pictures/chapter6/c06_p02.png}
    \caption{Minh họa hình dạng hạt}
\end{figure}

\section{Ma sát giữa các hạt và khả năng chảy của bột}

\subsection*{Ảnh hưởng của ma sát giữa các hạt}
\begin{itemize}
    \item Ma sát giữa các hạt ảnh hưởng đến khả năng của bột chảy dễ dàng và nén chặt.
\end{itemize}

\subsection*{Thử nghiệm góc nghỉ}
\begin{itemize}
    \item Một thử nghiệm phổ biến về ma sát giữa các hạt gọi là \textbf{góc nghỉ} (angle of repose). Thí nghiệm này được thực hiện bằng cách đổ một đống bột đổ ra từ một phễu hẹp.
    \item Góc lớn hơn có nghĩa là ma sát giữa các hạt lớn hơn.
\end{itemize}

\begin{figure}[H]
    \centering
    \includegraphics[width=0.6\textwidth]{pictures/chapter6/c06_p03.png}
    \caption{Thử nghiệm góc nghỉ}
\end{figure}

\section{Các nhận xét về ma sát giữa các hạt}

\begin{itemize}
    \item Kích thước hạt nhỏ hơn thường cho thấy ma sát lớn hơn và góc dốc hơn.
    \item Hình dạng hình cầu có ma sát giữa các hạt thấp nhất.
    \item Khi hình dạng lệch khỏi hình cầu, ma sát giữa các hạt có xu hướng tăng lên.
    \item Dòng chảy dễ dàng hơn của các hạt tương ứng với ma sát giữa các hạt thấp hơn.
    \item Chất bôi trơn thường được thêm vào bột để giảm ma sát giữa các hạt và tạo điều kiện dòng chảy trong quá trình ép.
\end{itemize}

\section{Các đại lượng đo mật độ hạt}

\subsection*{Mật độ thực (True density)}
\begin{itemize}
    \item Mật độ của thể tích thực của vật liệu.
    \item Mật độ của vật liệu nếu bột được nóng chảy thành khối rắn.
\end{itemize}

\subsection*{Mật độ đống (Bulk density)}
\begin{itemize}
    \item Mật độ của bột ở trạng thái rời sau khi đổ.
    \item Do có lỗ rỗng giữa các hạt, mật độ đống nhỏ hơn mật độ thực.
\end{itemize}

\section{Hệ số nén = Mật độ đống chia cho Mật độ thực}

\begin{itemize}
    \item Giá trị điển hình cho bột rời là 0.5 đến 0.7.
    \item Nếu có bột với các kích thước khác nhau, bột nhỏ hơn lấp đầy khoảng trống giữa các bột lớn hơn → hệ số nén cao hơn.
    \item Có thể tăng độ nén bằng cách rung bột, làm chúng nén chặt hơn → hệ số nén cao hơn.
    \item Áp suất được áp dụng trong quá trình nén làm tăng đáng kể hệ số nén của bột.
\end{itemize}

\section{Độ xốp (Porosity)}

\begin{itemize}
    \item Tỷ lệ thể tích của các lỗ rỗng (khoảng trống) trong bột so với thể tích tổng thể.
    \item Về nguyên tắc: Độ xốp + Hệ số nén = 1.0
    \item Vấn đề trở nên phức tạp do có thể tồn tại các lỗ rỗng đóng trong một số hạt.
    \item Nếu thể tích lỗ rỗng bên trong được tính trong độ xốp ở trên, thì phương trình là chính xác.
\end{itemize}

\section{Hóa học và màng bề mặt}

\begin{itemize}
    \item Bột kim loại được phân loại là:
    \begin{itemize}
        \item Nguyên tố - bao gồm kim loại nguyên chất.
        \item Hợp kim sẵn - mỗi hạt là một hợp kim.
    \end{itemize}
    \item Các màng bề mặt có thể có bao gồm: oxide, silica, vật liệu hữu cơ hấp thụ và độ ẩm.
    \item Theo quy tắc chung, các màng này phải được loại bỏ trước khi gia công tạo hình.
\end{itemize}

\section{Sản xuất bột kim loại}

\begin{itemize}
    \item Nói chung, các nhà sản xuất bột kim loại không phải là các công ty giống như những công ty sản xuất chi tiết.
    \item Bất kỳ kim loại nào cũng có thể được làm thành dạng bột.
    \item Ba phương pháp chính mà bột kim loại được sản xuất thương mại:
    \begin{enumerate}
        \item Phun (Atomization).
\subsubsection*{Phương pháp phun khí (Gas Atomization Method)}

\begin{itemize}
    \item Dòng khí tốc độ cao chảy qua vòi phun giãn nở, hút kim loại nóng chảy và phun nó thành buồng thu.
\end{itemize}

\begin{figure}[H]
    \centering
    \includegraphics[width=0.5\textwidth]{pictures/chapter6/c06_p04.png}
    \caption{Phương pháp phun khí}
\end{figure}

\subsubsection*{Phương pháp phun nước (Water Atomization Method)}

\begin{itemize}
    \item Dòng nước tốc độ cao chảy qua các vòi phun, làm nguội nhanh và làm đông đặc kim loại nóng chảy thành buồng thu.
\end{itemize}

\begin{figure}[H]
    \centering
    \includegraphics[width=0.5\textwidth]{pictures/chapter6/c06_p05.png}
    \caption{Phương pháp phun nước}
\end{figure}
        \item Hóa học (Chemical).
        \item Điện phân (Electrolytic).
    \end{enumerate}
    \item Ngoài ra, các phương pháp cơ học đôi khi được sử dụng để giảm kích thước bột.
\end{itemize}

\section{Ép và thiêu kết thông thường (Conventional Press and Sinter)}

\begin{itemize}
    \item Quy trình sản xuất chi tiết PM thông thường bao gồm:
    \begin{enumerate}
        \item Trộn và phối trộn bột.
        \item Nén - ép thành hình dạng mong muốn.
        \item Thiêu kết - gia nhiệt đến nhiệt độ dưới điểm nóng chảy để gây ra liên kết trạng thái rắn của các hạt và làm cứng chi tiết.
    \end{enumerate}
    \item Ngoài ra, các nguyên công phụ đôi khi được thực hiện để cải thiện độ chính xác kích thước, tăng mật độ và cho các lý do khác.
\end{itemize}

\begin{figure}[H]
    \centering
    \includegraphics[width=0.8\textwidth]{pictures/chapter6/c06_p06.png}
    \caption{Quy trình sản xuất PM thông thường: (1) Trộn, (2) Nén, (3) Thiêu kết}
\end{figure}

\subsection{Trộn và phối trộn bột (Blending and Mixing)}

\begin{itemize}
    \item Để có kết quả thành công trong quá trình nén và thiêu kết, bột ban đầu phải được đồng nhất hóa.
    \item \textbf{Trộn (Blending)}: bột có cùng thành phần hóa học nhưng có thể có kích thước hạt khác nhau được trộn lẫn vào nhau.
    \begin{itemize}
        \item Các kích thước hạt khác nhau thường được trộn để giảm độ xốp.
    \end{itemize}
    \item \textbf{Phối trộn (Mixing)}: bột có thành phần hóa học khác nhau được kết hợp.
\end{itemize}

\subsection{Nén (Compaction)}

\begin{itemize}
    \item Áp dụng áp suất cao lên bột để tạo thành hình dạng cần thiết.
    \item Phương pháp nén thông thường là ép (pressing), trong đó các chày đối diện ép bột chứa trong khuôn ép.
    \item Chi tiết sau khi ép được gọi là phôi xanh (green compact), từ green có nghĩa là chưa được gia công hoàn toàn.
    \item Độ bền xanh (green strength) của chi tiết khi được ép là đủ để xử lý nhưng thấp hơn nhiều so với sau khi thiêu kết.
\end{itemize}

\subsection{Ép thông thường trong PM (Conventional Pressing in PM)}

\begin{itemize}
    \item Ép trong PM: (1) lấp đầy khoang khuôn bằng bột bằng bộ nạp tự động, (2) vị trí ban đầu và (3) vị trí cuối cùng của chày trên và dưới trong quá trình ép, (4) đẩy chi tiết ra.
\end{itemize}

\begin{figure}[H]
    \centering
    \includegraphics[width=0.8\textwidth]{pictures/chapter6/c06_p07.png}
    \caption{Ép thông thường trong PM}
\end{figure}

\subsection{Thiêu kết (Sintering)}

\begin{itemize}
    \item Xử lý nhiệt để liên kết các hạt kim loại, do đó làm tăng độ bền và độ cứng.
    \item Thường được thực hiện ở 70\% đến 90\% nhiệt độ nóng chảy của kim loại (theo thang đo tuyệt đối).
    \item Các nhà nghiên cứu thường thống nhất rằng lực đẩy chính cho thiêu kết là giảm năng lượng bề mặt.
    \item Sự co ngót của chi tiết xảy ra trong quá trình thiêu kết do giảm kích thước lỗ rỗng.
\end{itemize}

\subsection{Chuỗi thiêu kết ở quy mô vi mô (Sintering Sequence on a Microscopic Scale)}

\begin{itemize}
    \item (1) Liên kết hạt được khởi tạo tại các điểm tiếp xúc; (2) các điểm tiếp xúc phát triển thành "cổ"; (3) các lỗ rỗng giữa các hạt giảm kích thước; (4) ranh giới hạt phát triển giữa các hạt thay cho các vùng có cổ.
\end{itemize}

\begin{figure}[H]
    \centering
    \includegraphics[width=0.5\textwidth]{pictures/chapter6/c06_p08.png}
    \caption{Chuỗi thiêu kết ở quy mô vi mô}
\end{figure}

\subsection{Chu trình thiêu kết và lò nung (Sintering Cycle and Furnace)}

\begin{itemize}
    \item (a) Chu trình xử lý nhiệt điển hình trong thiêu kết; và (b) mặt cắt sơ đồ của lò thiêu kết liên tục.
\end{itemize}

\begin{figure}[H]
    \centering
    \includegraphics[width=0.7\textwidth]{pictures/chapter6/c06_p09.png}
    \caption{Chu trình thiêu kết và lò nung}
\end{figure}

\subsection{Tăng mật độ và chỉnh kích thước (Densification and Sizing)}

\begin{itemize}
    \item Các nguyên công phụ được thực hiện trên chi tiết đã thiêu kết để tăng mật độ, cải thiện độ chính xác, hoặc hoàn thiện tạo hình bổ sung.
    \item \textbf{Ép lại (Repressing)}: ép trong khuôn đóng để tăng mật độ và cải thiện tính chất.
    \item \textbf{Chỉnh kích thước (Sizing)}: ép để cải thiện độ chính xác kích thước.
    \item \textbf{Đóng dấu (Coining)}: ép các chi tiết vào bề mặt của nó.
    \item \textbf{Gia công cắt gọt (Machining)}: cho các đặc điểm hình học không thể tạo bằng ép, chẳng hạn như ren và lỗ bên hông.
\end{itemize}

\section{Ngâm tẩm và thấm (Impregnation and Infiltration)}

\begin{itemize}
    \item Độ xốp là một đặc trưng độc đáo và cố hữu của công nghệ PM.
    \item Nó có thể được khai thác để tạo ra các sản phẩm đặc biệt bằng cách lấp đầy không gian lỗ rỗng có sẵn với dầu, polymer, hoặc kim loại.
    \item Hai phương pháp chính:
    \begin{enumerate}
        \item Ngâm tẩm (Impregnation).
        \item Thấm (Infiltration).
    \end{enumerate}
\end{itemize}

\subsection{Ngâm tẩm (Impregnation)}

\begin{itemize}
    \item Thuật ngữ được sử dụng khi dầu hoặc chất lỏng khác được thấm vào các lỗ rỗng của chi tiết PM đã thiêu kết.
    \item Các sản phẩm phổ biến là ổ bi ngâm dầu, bánh răng và các thành phần tương tự.
    \item Ứng dụng thay thế là khi các chi tiết được ngâm tẩm với nhựa polymer thấm vào các khoảng trống lỗ rỗng ở dạng lỏng và sau đó đông đặc để tạo ra chi tiết kín khí.
\end{itemize}

\subsection{Thấm (Infiltration)}

\begin{itemize}
    \item Nguyên công trong đó các lỗ rỗng của chi tiết PM được lấp đầy bằng kim loại nóng chảy.
    \item Điểm nóng chảy của kim loại lấp đầy phải thấp hơn của chi tiết PM.
    \item Nung nóng kim loại lấp đầy tiếp xúc với chi tiết đã thiêu kết để tác dụng mao dẫn kéo kim loại lấp đầy vào các lỗ rỗng.
    \item Cấu trúc thu được không xốp, và chi tiết đã thấm có mật độ đồng đều hơn, cũng như cải thiện độ dai và độ bền.
\end{itemize}

\section{Các kỹ thuật ép và thiêu kết thay thế}

\begin{itemize}
    \item Quy trình ép và thiêu kết thông thường là công nghệ tạo hình được sử dụng rộng rãi nhất trong luyện kim bột.
    \item Một số phương pháp bổ sung để sản xuất chi tiết PM:
    \begin{itemize}
        \item Ép đẳng tĩnh - áp suất thủy lực được áp dụng từ mọi hướng để đạt được sự nén chặt.
        \item Ép phun bột (PIM) - polymer khởi đầu có 50\% đến 85\% hàm lượng bột. Polymer được loại bỏ và chi tiết PM được thiêu kết.
        \item Ép nóng - kết hợp ép và thiêu kết.
    \end{itemize}
\end{itemize}

\section{Vật liệu và sản phẩm cho PM}

\begin{itemize}
    \item Nguyên liệu thô cho PM đắt hơn so với các phương pháp gia công kim loại khác do năng lượng bổ sung cần thiết để khử kim loại xuống dạng bột.
    \item Theo đó, PM chỉ cạnh tranh trong một phạm vi ứng dụng nhất định.
    \item Vật liệu và sản phẩm nào có vẻ phù hợp nhất với luyện kim bột?
\end{itemize}

\subsection{Vật liệu PM - Bột nguyên tố (Elemental Powders)}

\begin{itemize}
    \item Kim loại nguyên chất ở dạng hạt.
    \item Các bột nguyên tố phổ biến:
    \begin{itemize}
        \item Sắt (Iron).
        \item Nhôm (Aluminum).
        \item Đồng (Copper).
    \end{itemize}
    \item Bột nguyên tố có thể được trộn với bột kim loại khác để tạo ra các hợp kim khó pha chế bằng các phương pháp thông thường.
    \item Ví dụ: thép dụng cụ.
\end{itemize}

\subsection{Vật liệu PM - Bột hợp kim sẵn (Pre-Alloyed Powders)}

\begin{itemize}
    \item Mỗi hạt là một hợp kim bao gồm thành phần hóa học mong muốn.
    \item Các bột hợp kim sẵn phổ biến:
    \begin{itemize}
        \item Thép không gỉ (Stainless steels).
        \item Một số hợp kim đồng (Certain copper alloys).
        \item Thép tốc độ cao (High speed steel).
    \end{itemize}
\end{itemize}

\subsection{Sản phẩm PM (PM Products)}

\begin{itemize}
    \item Bánh răng, ổ bi, đai ốc răng, ốc vít, tiếp điểm điện, dụng cụ cắt và các chi tiết máy khác nhau.
    \item Ưu điểm của chi tiết PM: có thể được làm thành hình dạng gần chính xác hoặc chính xác.
    \item Khi sản xuất với số lượng lớn, bánh răng và ổ bi là lý tưởng cho PM vì:
    \begin{itemize}
        \item Hình học của chúng được xác định theo hai chiều.
        \item Cần có độ xốp trong chi tiết để làm bể chứa chất bôi trơn.
    \end{itemize}
\end{itemize}

\section{Hệ thống phân loại chi tiết PM}

\begin{itemize}
    \item Liên đoàn Công nghiệp Bột Kim loại (MPIF) định nghĩa bốn lớp chi tiết luyện kim bột theo thiết kế, theo mức độ khó khăn trong ép thông thường.
    \item Hữu ích vì nó chỉ ra một số hạn chế về hình dạng có thể đạt được với quá trình gia công PM thông thường.
\end{itemize}

\subsection{Bốn lớp chi tiết PM}

\begin{itemize}
    \item (a) Lớp I: Hình dạng đơn giản mỏng; (b) Lớp II: Đơn giản nhưng dày hơn; (c) Lớp III: Hai mức độ dày; và (d) Lớp IV: Nhiều mức độ dày.
\end{itemize}

\begin{figure}[H]
    \centering
    \includegraphics[width=0.8\textwidth]{pictures/chapter6/c06_p10.png}
    \caption{Bốn lớp chi tiết PM}
\end{figure}

\section{Hướng dẫn thiết kế cho chi tiết PM}

\subsection{Yêu cầu số lượng và khả năng độc đáo}

\begin{itemize}
    \item Số lượng lớn cần thiết để biện minh chi phí thiết bị và dụng cụ đặc biệt.
    \begin{itemize}
        \item Số lượng tối thiểu 10,000 đơn vị được đề xuất.
    \end{itemize}
    \item PM là duy nhất trong khả năng chế tạo chi tiết với mức độ xốp được kiểm soát.
    \begin{itemize}
        \item Độ xốp lên đến 50\% là có thể.
    \end{itemize}
    \item PM có thể được sử dụng để làm chi tiết từ kim loại và hợp kim khác thường.
    \begin{itemize}
        \item Vật liệu khó hoặc không thể sản xuất bằng các phương tiện khác.
    \end{itemize}
\end{itemize}

\subsection{Yêu cầu về hình học chi tiết}

\begin{itemize}
    \item Hình học chi tiết phải cho phép đẩy ra khỏi khuôn.
    \begin{itemize}
        \item Chi tiết phải có các cạnh thẳng đứng hoặc gần thẳng đứng, mặc dù các bậc được phép.
    \end{itemize}
    \item Các đặc điểm thiết kế trên các cạnh chi tiết như lỗ và gờ cắt phải được tránh.
    \item Các gờ cắt và lỗ thẳng đứng được phép vì chúng không cản trở việc đẩy ra.
    \item Lỗ thẳng đứng có thể có hình mặt cắt ngang khác với hình tròn mà không gặp khó khăn đáng kể.
\end{itemize}

\subsection{Lỗ bên hông và gờ cắt (Side Holes and Undercuts)}

Các đặc điểm chi tiết cần tránh trong PM: (a) lỗ bên hông và (b) gờ cắt bên hông vì việc đẩy chi tiết ra không thể thực hiện được.

\begin{figure}[H]
    \centering
    \includegraphics[width=0.5\textwidth]{pictures/chapter6/c06_p11.png}
    \caption{Lỗ bên hông và gờ cắt cần tránh}
\end{figure}

\subsection{Các hướng dẫn thiết kế khác}

\begin{itemize}
    \item Ren vít không thể được chế tạo bằng PM.
    \begin{itemize}
        \item Chúng phải được gia công vào chi tiết.
    \end{itemize}
    \item Vát cạnh và bán kính góc có thể thực hiện được trong PM.
    \begin{itemize}
        \item Nhưng vấn đề xảy ra trong độ cứng của chày khi các góc quá nhọn.
    \end{itemize}
    \item Độ dày thành tối thiểu nên là 1.5 mm (0.060 in) giữa các lỗ hoặc giữa lỗ và thành ngoài.
    \item Đường kính lỗ tối thiểu $\sim$ 1.5 mm (0.060 in).
\end{itemize}

\subsection{Vát cạnh và bán kính góc (Chamfers and Corner Radii)}

\begin{itemize}
    \item (a) Tránh các góc nhọn; (b) sử dụng các góc lớn hơn cho độ cứng của chày; (c) bán kính bên trong là mong muốn; (d) tránh bán kính góc ngoài đầy đủ vì chày dễ vỡ ở cạnh; (e) tốt hơn là kết hợp bán kính và vát cạnh.
\end{itemize}

\begin{figure}[H]
    \centering
    \includegraphics[width=0.7\textwidth]{pictures/chapter6/c06_p12.png}
    \caption{Vát cạnh và bán kính góc}
\end{figure}
\cleardoublepage

