\chapter{CÁC CÂU HỎI ÔN TẬP}
    \section{Các phương pháp đúc}
        \subsection{Nhóm đúc dùng khuôn cát và khuôn dùng một lần}

        Đặc trưng chung của nhóm này là khuôn chỉ sử dụng một lần,
        độ linh hoạt cao về hình dạng và kích thước chi tiết.

        \subsubsection{Đúc cát (Sand Casting)}
        \begin{itemize}
            \item \textbf{Nguyên lý}: Khuôn được tạo bằng cách đầm cát quanh mẫu,
            tháo mẫu và rót kim loại lỏng vào lòng khuôn.
            \item \textbf{Ưu điểm}: Đúc được hầu như mọi kim loại; không giới hạn kích thước;
            chi phí khuôn thấp.
            \item \textbf{Hạn chế}: Độ chính xác và chất lượng bề mặt thấp; thường cần gia công sau đúc.
            \item \textbf{Sản phẩm}: Block động cơ, thân máy, khung máy, chi tiết lớn.
        \end{itemize}

        \subsubsection{Đúc khuôn vỏ (Shell Molding)}
        \begin{itemize}
            \item \textbf{Đặc điểm phân biệt}: Khuôn mỏng bằng cát--nhựa trên mẫu kim loại nung nóng.
            \item \textbf{Ưu điểm}: Độ chính xác và bề mặt tốt hơn đúc cát; năng suất cao.
            \item \textbf{Hạn chế}: Kích thước chi tiết hạn chế; mẫu và thiết bị đắt.
            \item \textbf{Sản phẩm}: Vỏ hộp số, chi tiết cơ khí chính xác.
        \end{itemize}

        \subsubsection{Đúc mẫu cháy (Evaporative Pattern / Lost Foam)}
        \begin{itemize}
            \item \textbf{Đặc điểm phân biệt}: Mẫu xốp bị hóa hơi trong quá trình rót kim loại,
            không cần mặt phân khuôn.
            \item \textbf{Ưu điểm}: Tạo hình rất phức tạp.
            \item \textbf{Hạn chế}: Không kinh tế cho sản lượng nhỏ.
            \item \textbf{Sản phẩm}: Block động cơ ô tô, chi tiết phức tạp.
        \end{itemize}

        \subsubsection{Đúc khuôn thạch cao và khuôn gốm}
        \begin{itemize}
            \item \textbf{Đặc điểm phân biệt}: Khuôn phi kim có độ chính xác cao.
            \item \textbf{Ưu điểm}: Bề mặt nhẵn, đúc được thành mỏng.
            \item \textbf{Hạn chế}: Kích thước hạn chế; thường dùng cho kim loại nhiệt độ nóng chảy thấp.
            \item \textbf{Sản phẩm}: Chi tiết nhôm, chi tiết trang trí, cánh tuabin nhỏ.
        \end{itemize}

        \subsubsection{Đúc mẫu chảy (Investment Casting)}
        \begin{itemize}
            \item \textbf{Đặc điểm phân biệt}: Mẫu sáp phủ gốm, tạo khuôn rất chính xác.
            \item \textbf{Ưu điểm}: Độ chính xác và chất lượng bề mặt rất cao.
            \item \textbf{Hạn chế}: Quy trình phức tạp, chi phí cao.
            \item \textbf{Sản phẩm}: Cánh tuabin, chi tiết hàng không, trang sức.
        \end{itemize}

        %================================================
        \subsection{Nhóm đúc dùng khuôn kim loại}

        Khuôn được sử dụng nhiều lần, năng suất cao,
        phù hợp sản xuất loạt vừa và lớn.

        \subsubsection{Đúc khuôn kim loại (Permanent Mold Casting)}
        \begin{itemize}
            \item \textbf{Đặc điểm phân biệt}: Kim loại lỏng rót vào khuôn kim loại bằng trọng lực.
            \item \textbf{Ưu điểm}: Bề mặt tốt; ít rỗ; năng suất cao.
            \item \textbf{Hạn chế}: Chi phí khuôn cao; hình dạng chi tiết hạn chế.
            \item \textbf{Sản phẩm}: Piston, vỏ bơm, chi tiết nhôm.
        \end{itemize}

        \subsubsection{Đúc áp lực (Die Casting)}
        \begin{itemize}
            \item \textbf{Đặc điểm phân biệt}: Kim loại lỏng được ép vào khuôn kim loại
            bằng áp suất cao.
            \item \textbf{Ưu điểm}: Độ chính xác và bề mặt rất cao; phù hợp sản xuất hàng loạt.
            \item \textbf{Hạn chế}: Khuôn rất đắt; chủ yếu dùng kim loại màu.
            \item \textbf{Sản phẩm}: Vỏ điện tử, linh kiện ô tô.
        \end{itemize}

        %================================================
        \subsection{Nhóm đúc có cơ chế điền đầy đặc biệt}

        Phân biệt chủ yếu dựa trên cách kim loại điền đầy khuôn.

        \subsubsection{Đúc ly tâm (Centrifugal Casting)}
        \begin{itemize}
            \item \textbf{Cơ chế}: Kim loại lỏng chịu lực ly tâm trong khuôn quay.
            \item \textbf{Đặc điểm}: Không cần lõi; cơ tính tốt.
            \item \textbf{Sản phẩm}: Ống, bạc lót, vòng bi.
        \end{itemize}

        \subsubsection{Đúc áp suất thấp (Low-Pressure Casting)}
        \begin{itemize}
            \item \textbf{Cơ chế}: Kim loại lỏng được đẩy vào khuôn bằng áp suất khí thấp từ dưới lên.
            \item \textbf{Đặc điểm}: Điền đầy êm, ít rỗ khí.
            \item \textbf{Sản phẩm}: Mâm xe, chi tiết nhôm chất lượng cao.
        \end{itemize}

        \subsubsection{Đúc chân không (Vacuum Casting)}
        \begin{itemize}
            \item \textbf{Cơ chế}: Điền đầy khuôn trong môi trường chân không.
            \item \textbf{Đặc điểm}: Giảm ôxy hóa và khuyết tật khí.
            \item \textbf{Sản phẩm}: Chi tiết hợp kim chất lượng cao.
        \end{itemize}

        %================================================
        \subsection{Nhóm đúc theo trạng thái kim loại}

        \subsubsection{Đúc ép (Squeeze Casting)}
        \begin{itemize}
            \item \textbf{Đặc điểm phân biệt}: Kim loại lỏng chịu lực ép trong quá trình đông đặc.
            \item \textbf{Ưu điểm}: Cơ tính rất cao; ít rỗ khí.
            \item \textbf{Sản phẩm}: Chi tiết chịu tải, chi tiết ô tô.
        \end{itemize}

        \subsubsection{Đúc kim loại bán rắn (Semisolid-Metal Forming)}
        \begin{itemize}
            \item \textbf{Đặc điểm phân biệt}: Gia công kim loại ở trạng thái bán rắn.
            \item \textbf{Ưu điểm}: Độ chính xác cao; co ngót nhỏ.
            \item \textbf{Sản phẩm}: Linh kiện chính xác, chi tiết nhôm cao cấp.
        \end{itemize}

        %================================================
        \subsection{Các phương pháp đúc đặc thù}

        \subsubsection{Đúc vỏ mỏng (Slush Casting)}
        \begin{itemize}
            \item \textbf{Mục đích}: Tạo chi tiết rỗng mỏng, không chịu tải.
            \item \textbf{Sản phẩm}: Đồ trang trí, vỏ đèn.
        \end{itemize}

        \subsubsection{Đúc liên tục (Continuous Casting)}
        \begin{itemize}
            \item \textbf{Mục đích}: Tạo phôi kim loại liên tục.
            \item \textbf{Sản phẩm}: Phôi thép, phôi nhôm dạng tấm, thanh.
        \end{itemize}
    \section{Các phương pháp gia công biến dạng và sản phẩm của chúng}

        \subsection{Cán (Rolling)}
        \begin{itemize}
            \item \textbf{Phân loại:} Biến dạng khối (Bulk deformation).
            \item \textbf{Nguyên lý:}
            Phôi kim loại được đưa qua khe hở giữa hai hoặc nhiều trục cán quay,
            chịu ứng suất nén làm giảm chiều dày và tăng chiều dài phôi.
            \item \textbf{Ưu điểm:}
            \begin{itemize}
                \item Năng suất rất cao, phù hợp sản xuất hàng loạt.
                \item Có thể tạo ra phôi dài với tiết diện tương đối đồng đều.
            \end{itemize}

            \item \textbf{Hạn chế:}
            \begin{itemize}
                \item Độ chính xác hình dạng và kích thước hạn chế.
                \item Thường cần gia công tiếp theo.
            \end{itemize}
            \item \textbf{Sản phẩm tiêu biểu:}
            Tấm, lá kim loại; thanh, thép hình; ray đường sắt; phôi cán cho các quá trình khác.
        \end{itemize}

        \subsection{Rèn (Forging)}
            \begin{itemize}
                \item \textbf{Phân loại:} Biến dạng khối (Bulk deformation).

                \item \textbf{Nguyên lý:}
                Kim loại được biến dạng dẻo dưới tác dụng của lực nén lớn từ búa rèn
                hoặc máy ép, thường thông qua khuôn rèn.

                \item \textbf{Ưu điểm:}
                \begin{itemize}
                    \item Cơ tính cao nhờ dòng hạt liên tục.
                    \item Ít khuyết tật bên trong.
                \end{itemize}

                \item \textbf{Hạn chế:}
                \begin{itemize}
                    \item Chi phí khuôn cao.
                    \item Độ chính xác kích thước chưa cao.
                \end{itemize}

                \item \textbf{Sản phẩm tiêu biểu:}
                Trục khuỷu, bánh răng, tay biên, bu lông, các chi tiết chịu tải lớn.

            \end{itemize}
        
        \subsection{Ép đùn (Extrusion)}
            \begin{itemize}
                \item \textbf{Phân loại:} Biến dạng khối (Bulk deformation).

                \item \textbf{Nguyên lý:}
                Kim loại được ép chảy qua lỗ khuôn để tạo ra sản phẩm
                có tiết diện không đổi theo chiều dài.

                \item \textbf{Ưu điểm:}
                \begin{itemize}
                    \item Tạo được tiết diện phức tạp.
                    \item Bề mặt sản phẩm tương đối tốt.
                \end{itemize}

                \item \textbf{Hạn chế:}
                \begin{itemize}
                    \item Lực ép lớn.
                    \item Chiều dài sản phẩm bị giới hạn.
                \end{itemize}

                \item \textbf{Sản phẩm tiêu biểu:}
                Thanh nhôm định hình, ống, profile cửa, khung kết cấu nhẹ.
            \end{itemize}

        \subsection{Kéo dây và kéo thanh (Wire and Bar Drawing)}
        \begin{itemize}
            \item \textbf{Phân loại:} Biến dạng khối (Bulk deformation).

            \item \textbf{Nguyên lý:}
            Phôi kim loại được kéo qua lỗ khuôn,
            làm giảm tiết diện và tăng chiều dài.

            \item \textbf{Ưu điểm:}
            \begin{itemize}
                \item Độ chính xác kích thước cao.
                \item Bề mặt sản phẩm tốt.
            \end{itemize}

            \item \textbf{Hạn chế:}
            \begin{itemize}
                \item Giới hạn mức biến dạng.
                \item Có thể cần ủ trung gian.
            \end{itemize}

            \item \textbf{Sản phẩm tiêu biểu:}
            Dây điện, dây thép, dây lò xo, thanh tròn chính xác.
        \end{itemize}

        \subsection{Cắt kim loại tấm (Shearing / Blanking / Punching)}
        \begin{itemize}
            \item \textbf{Phân loại:} Gia công kim loại tấm (Sheet metalworking).

            \item \textbf{Nguyên lý:}
            Kim loại tấm bị tách rời dưới tác dụng của ứng suất trượt
            sinh ra bởi chày và cối.

            \item \textbf{Ưu điểm:}
            \begin{itemize}
                \item Năng suất cao.
                \item Độ chính xác biên dạng tốt.
            \end{itemize}

            \item \textbf{Hạn chế:}
            \begin{itemize}
                \item Tạo ứng suất dư và bavia ở mép cắt.
            \end{itemize}

            \item \textbf{Sản phẩm tiêu biểu:}
            Phôi dập, lỗ trên tấm, chi tiết phẳng.
        \end{itemize}

        \subsection{Uốn (Bending)}
        \begin{itemize}
            \item \textbf{Phân loại:} Gia công kim loại tấm (Sheet metalworking).

            \item \textbf{Nguyên lý:}
            Kim loại tấm bị biến dạng uốn quanh trục trung hòa
            dưới tác dụng của chày và cối.

            \item \textbf{Ưu điểm:}
            \begin{itemize}
                \item Thiết bị đơn giản.
                \item Dễ tự động hóa.
            \end{itemize}

            \item \textbf{Hạn chế:}
            \begin{itemize}
                \item Hiện tượng hồi phục đàn hồi (springback).
            \end{itemize}

            \item \textbf{Sản phẩm tiêu biểu:}
            Khung, giá đỡ, vỏ máy, chi tiết chữ U, V, L.
        \end{itemize}

        \subsection{Dập vuốt sâu (Deep Drawing)}
        \begin{itemize}
            \item \textbf{Phân loại:} Gia công kim loại tấm (Sheet metalworking).

            \item \textbf{Nguyên lý:}
            Tấm kim loại được biến dạng thành chi tiết rỗng
            nhờ chày, cối và lực giữ phôi.

            \item \textbf{Ưu điểm:}
            \begin{itemize}
                \item Tạo chi tiết rỗng liền khối.
                \item Ít phế liệu.
            \end{itemize}

            \item \textbf{Hạn chế:}
            \begin{itemize}
                \item Dễ nhăn hoặc rách nếu thông số không phù hợp.
            \end{itemize}

            \item \textbf{Sản phẩm tiêu biểu:}
            Lon nước giải khát, vỏ pin, vỏ hộp kim loại, chậu.
        \end{itemize}

        \subsection{Tạo hình lăn (Roll Forming)}
        \begin{itemize}
            \item \textbf{Phân loại:} Gia công kim loại tấm (Sheet metalworking).

            \item \textbf{Nguyên lý:}
            Tấm kim loại liên tục đi qua dãy trục lăn
            để dần dần đạt hình dạng mong muốn.

            \item \textbf{Ưu điểm:}
            \begin{itemize}
                \item Phù hợp sản xuất liên tục.
                \item Hình dạng ổn định.
            \end{itemize}

            \item \textbf{Hạn chế:}
            \begin{itemize}
                \item Chi phí đầu tư ban đầu lớn.
            \end{itemize}

            \item \textbf{Sản phẩm tiêu biểu:}
            Tôn lợp, thanh U--C, máng xối, panel kim loại.
        \end{itemize}

        \subsection{Ép xoay (Spinning)}
        \begin{itemize}
            \item \textbf{Phân loại:} Gia công kim loại tấm (Sheet metalworking).

            \item \textbf{Nguyên lý:}
            Tấm kim loại quay cùng khuôn,
            con lăn ép tấm sát vào khuôn để tạo hình đối xứng trục.

            \item \textbf{Ưu điểm:}
            \begin{itemize}
                \item Không cần khuôn phức tạp.
                \item Phù hợp sản xuất loạt nhỏ.
            \end{itemize}

            \item \textbf{Hạn chế:}
            \begin{itemize}
                \item Độ chính xác phụ thuộc điều khiển hoặc tay nghề.
            \end{itemize}

            \item \textbf{Sản phẩm tiêu biểu:}
            Chao đèn, chén kim loại, vỏ đối xứng trục.
        \end{itemize}
    \section{Các phương pháp gia công cắt gọt và khả năng thực hiện}
        \subsection{Tiện (Turning)}
        \begin{itemize}
            \item \textbf{Phân loại:} Gia công cắt gọt truyền thống (Conventional machining).

            \item \textbf{Nguyên lý:}
            Chi tiết quay quanh trục chính, dao tiện tịnh tiến theo phương dọc hoặc ngang
            để bóc tách vật liệu.

            \item \textbf{Ưu điểm:}
            \begin{itemize}
                \item Gia công được bề mặt tròn xoay chính xác.
                \item Máy và dao tương đối đơn giản.
            \end{itemize}

            \item \textbf{Hạn chế:}
            \begin{itemize}
                \item Chỉ gia công hiệu quả chi tiết dạng tròn xoay.
            \end{itemize}

            \item \textbf{Khả năng thực hiện:}
            Gia công thô và tinh; tiện trụ ngoài, trụ trong, mặt đầu, côn, ren.

            \item \textbf{Sản phẩm tiêu biểu:}
            Trục, bạc, ống, trục ren, chi tiết tròn xoay.
        \end{itemize}

        %================================================
        \subsection{Phay (Milling)}
        \begin{itemize}
            \item \textbf{Phân loại:} Gia công cắt gọt truyền thống.

            \item \textbf{Nguyên lý:}
            Dao phay quay, chi tiết hoặc dao tịnh tiến để cắt bỏ vật liệu từng lớp.

            \item \textbf{Ưu điểm:}
            \begin{itemize}
                \item Gia công được bề mặt phẳng và biên dạng phức tạp.
                \item Linh hoạt, dễ tự động hóa CNC.
            \end{itemize}

            \item \textbf{Hạn chế:}
            \begin{itemize}
                \item Lực cắt gián đoạn gây rung.
            \end{itemize}

            \item \textbf{Khả năng thực hiện:}
            Phay mặt, phay rãnh, phay biên dạng, phay khuôn 3D.

            \item \textbf{Sản phẩm tiêu biểu:}
            Khuôn mẫu, vỏ máy, mặt phẳng chính xác, chi tiết cơ khí phức tạp.
        \end{itemize}

        %================================================
        \subsection{Khoan (Drilling)}
        \begin{itemize}
            \item \textbf{Phân loại:} Gia công cắt gọt tạo lỗ.

            \item \textbf{Nguyên lý:}
            Mũi khoan quay và tiến dọc trục để tạo lỗ mới trên chi tiết.

            \item \textbf{Ưu điểm:}
            \begin{itemize}
                \item Thiết bị đơn giản, phổ biến.
            \end{itemize}

            \item \textbf{Hạn chế:}
            \begin{itemize}
                \item Độ chính xác và độ nhẵn bề mặt lỗ không cao.
            \end{itemize}

            \item \textbf{Khả năng thực hiện:}
            Tạo lỗ thô; thường cần doa hoặc khoét sau đó.

            \item \textbf{Sản phẩm tiêu biểu:}
            Lỗ lắp bu lông, lỗ dẫn, lỗ chuẩn.
        \end{itemize}

        %================================================
        \subsection{Doa (Reaming)}
        \begin{itemize}
            \item \textbf{Phân loại:} Gia công tinh lỗ.

            \item \textbf{Nguyên lý:}
            Dao doa cắt lớp mỏng để cải thiện kích thước và độ nhẵn lỗ.

            \item \textbf{Ưu điểm:}
            \begin{itemize}
                \item Độ chính xác kích thước cao.
            \end{itemize}

            \item \textbf{Hạn chế:}
            \begin{itemize}
                \item Không dùng để tạo lỗ mới.
            \end{itemize}

            \item \textbf{Khả năng thực hiện:}
            Gia công tinh lỗ sau khoan.

            \item \textbf{Sản phẩm tiêu biểu:}
            Lỗ lắp ổ trục, chốt định vị.
        \end{itemize}

        %================================================
        \subsection{Bào và xọc (Shaping and Slotting)}
        \begin{itemize}
            \item \textbf{Phân loại:} Gia công cắt gọt truyền thống.

            \item \textbf{Nguyên lý:}
            Dao chuyển động tịnh tiến qua lại để bóc vật liệu.

            \item \textbf{Ưu điểm:}
            \begin{itemize}
                \item Kết cấu máy đơn giản.
            \end{itemize}

            \item \textbf{Hạn chế:}
            \begin{itemize}
                \item Năng suất thấp.
            \end{itemize}

            \item \textbf{Khả năng thực hiện:}
            Gia công mặt phẳng, rãnh then.

            \item \textbf{Sản phẩm tiêu biểu:}
            Rãnh then, mặt phẳng đơn giản.
        \end{itemize}

        %================================================
        \subsection{Chuốt (Broaching)}
        \begin{itemize}
            \item \textbf{Phân loại:} Gia công cắt gọt năng suất cao.

            \item \textbf{Nguyên lý:}
            Dao chuốt nhiều răng cắt tăng dần,
            mỗi răng bóc một lượng nhỏ vật liệu.

            \item \textbf{Ưu điểm:}
            \begin{itemize}
                \item Độ chính xác cao.
                \item Năng suất rất lớn.
            \end{itemize}

            \item \textbf{Hạn chế:}
            \begin{itemize}
                \item Dao chuốt rất đắt, kém linh hoạt.
            \end{itemize}

            \item \textbf{Khả năng thực hiện:}
            Gia công rãnh then, then hoa, lỗ định hình.

            \item \textbf{Sản phẩm tiêu biểu:}
            Bánh răng trong, then hoa, lỗ đa giác.
        \end{itemize}

        %================================================
        \subsection{Mài (Grinding)}
        \begin{itemize}
            \item \textbf{Phân loại:} Gia công mài mòn (Abrasive machining).

            \item \textbf{Nguyên lý:}
            Dùng hạt mài quay tốc độ cao để bóc lớp vật liệu rất mỏng.

            \item \textbf{Ưu điểm:}
            \begin{itemize}
                \item Độ chính xác và độ nhẵn bề mặt rất cao.
            \end{itemize}

            \item \textbf{Hạn chế:}
            \begin{itemize}
                \item Năng suất thấp.
                \item Dễ sinh nhiệt.
            \end{itemize}

            \item \textbf{Khả năng thực hiện:}
            Gia công tinh, gia công vật liệu cứng.

            \item \textbf{Sản phẩm tiêu biểu:}
            Trục chính xác, bề mặt khuôn, chi tiết sau nhiệt luyện.
        \end{itemize}

    \section{Phương pháp gia công kim loại bột, nguyên lý, điều kiện}
    \subsection{Nguyên lý hoạt động}
    Gia công kim loại bột là quá trình sản xuất các chi tiết kim loại bằng cách nén bột kim loại vào khuôn để tạo hình dáng mong muốn, sau đó nung nóng (thiêu kết) để các hạt bột liên kết lại với nhau thành khối rắn.

    Quy trình hoạt động dựa trên 4 giai đoạn cốt lõi:

    \subsubsection*{Giai đoạn 1: Tạo bột và Phối trộn (Powder Production \& Blending)}
    \begin{itemize}
        \item Kim loại được biến thành bột mịn (thông qua phun kim loại lỏng, khử hóa học, hoặc nghiền).
        \item Các loại bột khác nhau và chất bôi trơn được trộn đều để đảm bảo sự đồng nhất về tính chất cơ học và giảm ma sát khi ép.
    \end{itemize}

    \subsubsection*{Giai đoạn 2: Nén/Ép (Compaction)}
    \begin{itemize}
        \item Bột được nén trong khuôn dưới áp lực cao.
        \item \textbf{Mục đích}: Tạo ra hình dáng hình học, tăng mật độ tiếp xúc giữa các hạt và tạo độ bền ban đầu (được gọi là độ bền tươi - green strength) để chi tiết không bị vỡ khi xử lý tiếp theo.
        \item Sản phẩm sau bước này gọi là "chi tiết tươi" (green compact), rất giòn và dễ vỡ.
    \end{itemize}

    \subsubsection*{Giai đoạn 3: Thiêu kết (Sintering)}
    \begin{itemize}
        \item Chi tiết tươi được nung nóng trong lò có kiểm soát khí quyển.
        \item \textbf{Nhiệt độ nung}: Dưới nhiệt độ nóng chảy của kim loại chính.
        \item \textbf{Cơ chế}: Các hạt bột liên kết với nhau thông qua cơ chế khuếch tán trạng thái rắn (diffusion) hoặc vận chuyển pha hơi, làm tăng sức bền và mật độ của chi tiết.
    \end{itemize}

    \subsubsection*{Giai đoạn 4: Hoàn thiện (Finishing - Tùy chọn)}
    \begin{itemize}
        \item Các nguyên công phụ như ép lại (coining) để tăng độ chính xác, gia công cắt gọt, thấm dầu (impregnation) hoặc mạ.
    \end{itemize}

    \subsection{Các điều kiện kỹ thuật (Process Conditions)}
    Để quá trình thành công, cần tuân thủ nghiêm ngặt các điều kiện ở từng giai đoạn:

    \subsubsection*{A. Điều kiện về Bột kim loại}
    \begin{itemize}
        \item \textbf{Kích thước hạt}: Phải được kiểm soát, thường nằm trong khoảng $0.1$ đến $1000~\mu$m.
        \item \textbf{Hình dáng hạt}: Ảnh hưởng đến khả năng chảy và mật độ khi nén (hạt hình cầu chảy tốt hơn hạt hình vảy).
        \item \textbf{An toàn}: Bột kim loại (đặc biệt là Nhôm, Magie, Titan) dễ gây nổ do diện tích bề mặt lớn, cần điều kiện tiếp địa và tránh tia lửa điện khi xử lý.
    \end{itemize}

    \subsubsection*{B. Điều kiện khi Nén (Compaction)}
    \begin{itemize}
        \item \textbf{Áp suất nén}: Phải đủ lớn để tạo mật độ và độ bền tươi.
        \begin{itemize}
            \item Nhôm: $70-275$ MPa.
            \item Sắt: $350-800$ MPa.
            \item Đồng thau (Brass): $400-700$ MPa.
        \end{itemize}
        \item \textbf{Khe hở khuôn}: Khe hở giữa chày và cối phải rất nhỏ (thường $< 25~\mu$m) để tránh bột lọt vào khe gây kẹt hoặc làm lệch chi tiết.
        \item \textbf{Ma sát}: Cần giảm ma sát giữa các hạt bột và thành khuôn (dùng chất bôi trơn) để mật độ phân bố đồng đều.
    \end{itemize}

    \subsubsection*{C. Điều kiện khi Thiêu kết (Sintering)}
    Đây là bước quan trọng nhất quyết định tính chất vật liệu.
    \begin{itemize}
        \item \textbf{Nhiệt độ}: Thường nằm trong khoảng 70\% - 90\% nhiệt độ nóng chảy của kim loại/hợp kim.
        \begin{itemize}
            \item Ví dụ: Đồng/Hợp kim đồng nung ở $760-900^{\circ}$C.
            \item Sắt: $1000-1150^{\circ}$C.
        \end{itemize}
        \item \textbf{Thời gian}: Từ 10 phút (cho sắt/đồng) đến 8 giờ (cho Tungsten/Tantalum).
        \item \textbf{Môi trường lò (Khí quyển)}: Phải được kiểm soát chặt chẽ để tránh oxy hóa.
        \begin{itemize}
            \item Khử oxy (Hydrogen), môi trường khí trơ, hoặc chân không (đối với thép không gỉ hoặc kim loại chịu nhiệt).
        \end{itemize}
    \end{itemize}

    \section{Hạt mài và độ cứng của chúng như thế nào?}
    \subsection{Độ cứng của hạt mài (Abrasive Hardness)}
    Đây là khả năng của bản thân hạt mài trong việc chống lại sự trầy xước hoặc biến dạng khi tiếp xúc với phôi. Các siêu hạt mài như kim cương và CBN có độ cứng vượt trội so với phôi thép thông thường.

    Hạt mài được chia thành hai nhóm chính dựa trên khả năng cắt và độ cứng của chúng:

    \subsubsection*{Hạt mài thông thường (Conventional abrasives)}
    \begin{itemize}
        \item \textbf{Oxit nhôm (Aluminum oxide)}: Thường dùng để mài các loại thép và hợp kim có độ bền cao.
        \item \textbf{Silicon carbide}: Thường dùng cho các vật liệu giòn như gang, hoặc các kim loại mềm hơn như nhôm và đồng.
    \end{itemize}

    \subsubsection*{Siêu hạt mài (Superabrasives)}
    \begin{itemize}
        \item \textbf{Nitrit bo lập phương (CBN - Cubic Boron Nitride)}: Có độ cứng rất cao, bền nhiệt, thích hợp cho các loại thép dụng cụ cứng.
        \item \textbf{Kim cương (Diamond)}: Là vật liệu cứng nhất, thường dùng để mài các vật liệu cực cứng như gốm sứ, cacbit hoặc thủy tinh.
    \end{itemize}

    \section{Các phương pháp gia công mài}
    
    \subsection{Mài bề mặt phẳng (Surface Grinding)}
    \begin{itemize}
        \item \textbf{Trục ngang - Bàn tịnh tiến:}
        \begin{itemize}
            \item \textbf{Ứng dụng:} Mài các chi tiết có bề mặt phẳng, dài và hẹp. Đây là phương pháp phổ biến nhất để đạt độ chính xác cao và bề mặt mịn cho các phôi dạng thanh hoặc tấm.
        \end{itemize}
        
        \item \textbf{Trục ngang - Bàn quay:}
        \begin{itemize}
            \item \textbf{Ứng dụng:} Mài các chi tiết hình tròn hoặc các nhóm chi tiết nhỏ xếp vòng quanh. Tạo ra các vết mài đồng tâm, phù hợp cho các loại lưỡi của tròn, vòng đệm hoặc đĩa ly hợp.
        \end{itemize}
        
        \item \textbf{Trục đứng - Bàn tịnh tiến:}
        \begin{itemize}
            \item \textbf{Ứng dụng:} Dùng để mài phá hoặc mài thô các bề mặt phẳng có diện tích lớn. Nhờ diện tích tiếp xúc của bánh mài lớn nên tốc độ loại bỏ vật liệu rất nhanh, phù hợp cho các tấm thép dày hoặc đế máy.
        \end{itemize}
        
        \item \textbf{Trục đứng - Bàn quay:}
        \begin{itemize}
            \item \textbf{Ứng dụng:} Sử dụng trong sản xuất hàng loạt. Nhiều chi tiết nhỏ (như mặt đầu vòng bi, bánh răng) được đặt trên bàn quay để gia công liên tục, giúp đạt năng suất cực cao.
        \end{itemize}
    \end{itemize}

    \subsection{Mài tròn trụ (Cylindrical Grinding)}
    \begin{itemize}
        \item \textbf{Mài ngoài tròn trụ (External cylindrical grinding)}
        \item \textbf{Mài trong tròn trụ (Internal cylindrical grinding)}
    \end{itemize}

    \subsection{Mài không tâm (Centerless Grinding)}
    \begin{itemize}
        \item \textbf{Ứng dụng:} Chuyên dùng cho các bộ phận như vòng bi đũa, chốt piston, van động cơ, trục cam và các linh kiện tương tự. Phương pháp này có thể gia công các chi tiết có đường kính rất nhỏ, tới 0,1 mm.
    \end{itemize}

    \subsection{Mài tiến dao chậm (Creep-feed Grinding)}
    \begin{itemize}
        \item \textbf{Ứng dụng:} Được sử dụng cho các hoạt động loại bỏ kim loại ở quy mô lớn, cạnh tranh với các quy trình gia công như phay và tiện. Chiều sâu cắt có thể lên tới 6 mm.
    \end{itemize}

    \subsection{Mài đai (Belt Grinding)}
    \begin{itemize}
        \item \textbf{Ứng dụng:} Ứng dụng để hoàn thiện các bề mặt phẳng hoặc cong của các bộ phận kim loại và phi kim loại, màu kim tương, và trong ngành gỗ. Các ứng dụng điển hình bao gồm gậy golf, súng trường, cánh tuabin, bộ phận cấy ghép phẫu thuật và dụng cụ y tế/nha khoa.
    \end{itemize}

    \subsection{Mài khôn (Honing)}
    \begin{itemize}
        \item \textbf{Ứng dụng:} Chủ yếu dùng để cải thiện độ bóng bề mặt của các lỗ sau khi đã doa, khoan hoặc mài trong. Nó cũng được dùng để loại bỏ các cạnh sắc trên dụng cụ cắt một cách thủ công.
    \end{itemize}

    \subsection{Rà (Lapping)}
    \begin{itemize}
        \item \textbf{Ứng dụng:} Dùng để hoàn thiện các bề mặt phẳng, hình trụ hoặc mặt cong. Các ứng dụng đặc biệt bao gồm rà các vật thể hình cầu, thấu kính thủy tinh và rà khớp các bánh răng (như bánh răng hypoid cho cầu sau xe).
    \end{itemize}

    \section{Cơ chế tạo phoi. Các dạng phoi. Điều kiện hình thành}
        \subsection{Cơ chế tạo phoi}
        Quá trình tạo phoi diễn ra thông qua sự biến dạng dẻo và trượt của vật liệu.

        \begin{itemize}
            \item \textbf{Sự trượt (Shearing)}: Khi dụng cụ cắt tiến vào phôi, vật liệu phía trước mũi dao bị nén và trượt liên tục dọc theo một mặt phẳng gọi là \textbf{mặt phẳng trượt (shear plane)}.
            
            \item \textbf{Vùng trượt}: Hiện tượng này diễn ra trong một vùng hẹp được gọi là vùng trượt chính. Cơ chế này tương tự như việc cắt lá bài trong một bộ bài trượt lên nhau.
            
            \item \textbf{Đặc điểm bề mặt}: Phoi có hai bề mặt khác nhau: mặt tiếp xúc với dao thường bóng mín do ma sát trượt, trong khi mặt ngoài có dạng răng cưa hoặc nhám do cơ chế trượt không đồng đều.
        \end{itemize}

        \subsection{Các dạng phoi và điều kiện hình thành}
        Tùy thuộc vào vật liệu phôi và các thông số cắt gọt, có 4 dạng phoi chính thường gặp:

        \subsubsection*{Phoi liên tục / Phoi dây (Continuous Chips)}
        \begin{itemize}
            \item \textbf{Đặc điểm}: Dạng dài dài, không bị đứt đoạn, cho bề mặt gia công bóng mín.
            
            \item \textbf{Điều kiện hình thành}:
            \begin{itemize}
                \item Vật liệu phôi dẻo (nhôm, thép mềm).
                \item Vận tốc cắt cao ($V$ lớn).
                \item Góc trước của dao lớn ($\alpha$ lớn).
                \item Ma sát tại mặt trước của dao thấp.
            \end{itemize}
        \end{itemize}

        \subsubsection*{Phoi có lẹo dao (Built-up Edge - BUE)}
        \begin{itemize}
            \item \textbf{Đặc điểm}: Các lớp vật liệu phôi bám chặt và tích tụ dần trên mũi dao, làm thay đổi hình dáng hình học của lưỡi cắt.
            
            \item \textbf{Điều kiện hình thành}:
            \begin{itemize}
                \item Vật liệu dẻo ở vận tốc thấp đến trung bình.
                \item Ma sát lớn giữa phoi và mặt trước dao.
                \item Góc trước của dao thấp.
                \item Thiếu dung dịch trơn nguội hiệu quả.
            \end{itemize}
        \end{itemize}

        \subsubsection*{Phoi răng cưa / Phoi xẻp (Serrated / Segmented Chips)}
        \begin{itemize}
            \item \textbf{Đặc điểm}: Là dạng phoi bán liên tục với các vùng biến dạng thấp xen kẽ vùng biến dạng cao, có hình dạng giống lưỡi cưa.
            
            \item \textbf{Điều kiện hình thành}:
            \begin{itemize}
                \item Vật liệu có độ dẻo nhiệt thấp và độ bền giảm nhanh khi nhiệt độ tăng (hiện tượng mềm hóa do nhiệt).
                \item Thường gặp nhất khi gia công các hợp kim Titan.
            \end{itemize}
        \end{itemize}

        \subsubsection*{Phoi vụn / Phoi đứt quãng (Discontinuous Chips)}
        \begin{itemize}
            \item \textbf{Đặc điểm}: Phoi bị gãy thành từng mảnh rời rạc hoặc liên kết rất yếu.
            
            \item \textbf{Điều kiện hình thành}:
            \begin{itemize}
                \item Vật liệu phôi giòn (gang xám, đồng thau đúc).
                \item Vật liệu có chứa các tạp chất cứng.
                \item Vận tốc cắt rất thấp hoặc rất cao.
                \item Chiều sâu cắt lớn.
                \item Góc trước của dao nhỏ.
                \item Hệ thống gia công (máy, dao) có độ cứng vững thấp gây rung động.
            \end{itemize}
        \end{itemize}

    \section{Các yếu tố ảnh hưởng đến tuổi thọ của dụng cụ cắt như thế nào?}
        Đây là nhóm yếu tố quan trọng nhất mà người vận hành có thể điều chỉnh trực tiếp.

        \subsection{Các thông số cắt gọt (Vận tốc, Bước tiến, Chiều sâu cắt)}
        \subsubsection*{Vận tốc cắt ($V$)}
        Là biến số có ảnh hưởng lớn nhất đến tuổi thọ dụng cụ. Mối quan hệ này được thể hiện qua phương trình tuổi thọ Taylor:
        \[
        VT^n = C
        \]
        Trong đó $T$ là tuổi thọ dụng cụ, $n$ và $C$ là các hằng số phụ thuộc vào điều kiện cắt. Chỉ cần giảm một lượng nhỏ vận tốc cắt có thể làm tăng đáng kể tuổi thọ dụng cụ (ví dụ: giảm 50\% vận tốc có thể tăng tuổi thọ lên 300\%).

        \subsubsection*{Bước tiến ($f$) và Chiều sâu cắt ($d$)}
        Có ảnh hưởng thấp hơn so với vận tốc cắt nhưng vẫn rất quan trọng. Khi tăng bước tiến hoặc chiều sâu cắt, tuổi thọ dụng cụ sẽ giảm xuống.

        \subsection{Vật liệu dụng cụ và Lớp phủ}
        Đặc tính vật lý của dụng cụ quyết định khả năng chịu nhiệt và chống mài mòn.

        \begin{itemize}
            \item \textbf{Loại vật liệu}: Các vật liệu như cacbit, gốm (ceramics), và cubic boron nitride (cBN) được thiết kế để chịu được nhiệt độ và áp suất cao hơn so với thép gió (HSS).
            
            \item \textbf{Lớp phủ (Coatings)}: Các lớp phủ như TiN, TiC, hoặc nhôm oxit giúp giảm ma sát và ngăn chặn quá trình khuếch tán nguyên tử, từ đó giảm đáng kể mòn khuyết (crater wear).
        \end{itemize}

        \subsection{Đặc điểm của vật liệu phôi}
        Tính trạng của vật liệu được gia công ảnh hưởng trực tiếp đến tốc độ mòn dụng cụ.

        \begin{itemize}
            \item \textbf{Độ cứng và Cấu trúc vi mô}: Vật liệu càng cứng (như martensite so với ferrite) thì dụng cụ càng nhanh mòn.
            
            \item \textbf{Tạp chất}: Sự hiện diện của các hạt cứng, lớp váy oxit, hoặc gỉ sét trên bề mặt phôi sẽ gây ra mòn mài mòn (abrasive wear) rất mạnh.
        \end{itemize}

        \subsection{Nhiệt độ trong vùng cắt}
        Nhiệt độ là tác nhân gián tiếp nhưng cực kỳ nguy hiểm đối với dụng cụ.

        \begin{itemize}
            \item \textbf{Sự mềm hóa}: Nhiệt độ quá cao làm giảm độ bền, độ cứng và khả năng chống mài mòn của dụng cụ.
            
            \item \textbf{Mòn khuếch tán}: Ở nhiệt độ cao, các nguyên tử di chuyển qua lại giữa phoi và dụng cụ, gây ra mòn khuyết trên mặt trước của dao.
        \end{itemize}

        \subsection{Các yếu tố khác}
        \subsubsection*{Hình học dụng cụ}
        Góc trước ($\alpha$), bán kính mũi dao và độ sắc của lưỡi cắt ảnh hưởng đến cách phôi trượt và phân bố nhiệt.

        \subsubsection*{Dung dịch cắt gọt}
        Giúp làm mát và bôi trơn, giảm ma sát tại mặt tiếp xúc dụng cụ-phôi.

        \subsubsection*{Độ cứng vững của hệ thống}
        Nếu máy công cụ hoặc đồ gá không đủ cứng vững, hiện tượng rung động và chattering sẽ xảy ra, dẫn đến mẻ dao (chipping) hoặc gãy dao đột ngột.
    \section{Cấu tạo và chức năng các bộ phận của khuôn đúc}
        \subsection{Các bộ phận chính của khuôn đúc}

        \begin{itemize}
            \item \textbf{Lòng khuôn (Mold cavity):}  
            Không gian rỗng trong khuôn có hình dạng và kích thước của vật đúc,
            quyết định trực tiếp hình học của sản phẩm.

            \item \textbf{Hộp khuôn (Flask – cope và drag):}  
            Kết cấu bao ngoài dùng để giữ và định vị khuôn,
            đảm bảo khuôn không bị biến dạng khi rót kim loại
            (chủ yếu dùng trong đúc cát).

            \item \textbf{Mặt phân khuôn (Parting line):}  
            Mặt tiếp giáp giữa hai nửa khuôn,
            cho phép tháo mẫu và lắp khuôn trước khi rót kim loại.

            \item \textbf{Mẫu (Pattern):}  
            Mô hình có hình dạng tương ứng với vật đúc,
            dùng để tạo lòng khuôn;
            có thể là mẫu gỗ, kim loại, nhựa, sáp hoặc xốp
            tùy theo phương pháp đúc.

            \item \textbf{Lõi (Core):}  
            Bộ phận đặt trong khuôn để tạo các khoang rỗng,
            lỗ hoặc kênh bên trong vật đúc.

            \item \textbf{Gối lõi (Core print):}  
            Phần định vị lõi trong khuôn,
            đảm bảo lõi không dịch chuyển khi rót kim loại.

        \end{itemize}

        \subsection{Hệ thống rót kim loại}

        \begin{itemize}
            \item \textbf{Cốc rót (Pouring cup):}  
            Nơi tiếp nhận kim loại lỏng ban đầu,
            giúp rót kim loại ổn định và giảm bắn tóe.

            \item \textbf{Ống rót (Sprue):}  
            Dẫn kim loại lỏng từ cốc rót xuống hệ thống kênh dẫn.

    \subsubsection*{Độ cứng vững của hệ thống}
    Nếu máy công cụ hoặc đồ gá không đủ cứng vững, hiện tượng rung động và chattering sẽ xảy ra, dẫn đến mẻ dao (chipping) hoặc gãy dao đột ngột.
        

    \section{Các phương pháp gia công tiên tiến (Advanced Machining Processes) giúp giải quyết các vấn đề gì?}
        \subsection{Giải quyết vấn đề về vật liệu: độ cứng và độ bền}
            \hspace*{0.6cm}Hạn chế lớn nhất của gia công truyền thống là nguyên tắc "độ cứng tương đối": dụng cụ cắt phải cứng hơn phôi. Khi vật liệu phôi có độ cứng vượt quá 40-50 HRC (như thép đã tôi, cacbit vonfram) hoặc có độ bền nhiệt cao (như siêu hợp kim nền Niken/Titan), dao cắt truyền thống bị mòn rất nhanh, gãy vỡ, hoặc không thể cắt được
            \subsubsection{Gia công Vật liệu Siêu cứng và Giòn (Ceramics, Thủy tinh, Carbide)}
                \begin{itemize}
                    \item Gia công Siêu âm (Ultrasonic Machining - USM): Dụng cụ dao động với tần số cao (khoảng 20 kHz) và biên độ nhỏ truyền động năng cho các hạt mài trong dung dịch huyền phù (slurry). Các hạt mài này va đập vào bề mặt phôi, gây ra các vết nứt vi mô và tách vật liệu ra dưới dạng bột mịn.
                    \item Gia công Tia hạt mài (Abrasive Jet Machining - AJM): Đối với các tấm vật liệu mỏng và giòn như mica, thủy tinh, silicon, lực kẹp và lực cắt của máy phay có thể làm gãy chi tiết ngay lập tức. AJM (Mục 7.4.4 ) giải quyết bằng cách sử dụng luồng khí chứa hạt mài tốc độ cao.
                \end{itemize}
            \subsubsection{Gia công Vật liệu Dẫn điện Siêu bền (Hợp kim Titan, Thép đã tôi)}
                \begin{itemize}
                    \item Gia công Tia lửa điện (Electrical Discharge Machining - EDM):  Vì vật liệu bị bóc tách bởi năng lượng nhiệt của tia lửa điện (nóng chảy và bay hơi) nên không phải quan tâm đến độ cứng vật lý của phôi. Điều đó dẫn đến cách mạng trong công nghiệp sản xuất, khi người ta có thể tôi cứng khối thép trước, sau đó dùng EDM để gia công hốc khuôn với độ chính xác tuyệt đối thay vì phải cắt gọt rồi mới tôi cứng như trước.
                    \item Gia công Điện hóa (Electrochemical Machining - ECM): ECM hòa tan vật liệu theo cơ chế điện phân từng nguyên tử một. Quá trình này không sinh nhiệt, không gây áp lực cơ học. Do đó có thể gia công các vật liệu siêu bền mà không để lại ứng suất dư hay làm biến đổi cấu trúc luyện kim.
                \end{itemize}
            \subsubsection{Gia công Vật liệu Mềm, Dẻo và Nhạy cảm Nhiệt}
                \begin{itemize}
                    \item Cắt bằng Tia nước (Water Jet Cutting - WJC): sử dụng dòng nước áp suất siêu cao để cắt xuyên qua vật liệu. Phương pháp sử dụng không sinh nhiệt, lực cắt điểm nhỏ, không làm cháy mép cạnh, biến tính vật liệu, hay biến dạng cơ học. 
                \end{itemize}  

        \subsection{Giải quyết vấn đề về hình học trong gia công}
            \hspace*{0.6cm}Sự phức tạp của hình học là một trong những rào cản lớn trong gia công theo các phương pháp truyền thống, kể cả với những máy CNC 5 trục hiện đại ngày nay, vẫn tồn tại hạn chế tự dụng cụ cắt và bán kính dao, các phương pháp gia công không truyền thống đưa ra được những giải pháp phù hợp giải quyết các vấn đề này, cụ thể
            \subsubsection{Gia công các hốc sâu và góc nhọn}
                \hspace*{0.6cm}Giải pháp EDM xung định hình, điện cực EDM có thể được gia công với các cạnh sắc nhọn. Khi điện cực đi sâu vào phôi, nó xói mòn vật liệu tạo thành hốc có hình dạng âm bản chính xác của điện cực.
            \subsubsection{Cắt khe hẹp và chi tiết mảnh}
            \hspace*{0.6cm}Ví dụ như các lá tản nhiệt, khe hở trong linh kiện điện tử, ...
            \newline    
            \hspace*{0.6cm}Giải pháp Wire-EDM (Cắt dây): Sử dụng một dây dẫn điện mảnh (thường bằng đồng hoặc molypden, đường kính từ 0.02 - 0.3 mm) làm điện cực cắt. Quá trình phóng điện tạo ra rãnh cắt (kerf) chỉ rộng hơn đường kính dây một chút (do khe hở phóng điện - gap). Độ chính xác của phương pháp cắt dây có thể đạt tới micromet.
            \subsubsection{Gia công nhiều lỗ đồng thời}
                \hspace*{0.6cm}Giải pháp ECM: ECM cho phép gia công hàng loạt lỗ đồng thời (gang drilling) bằng cách sử dụng một điện cực chùm.
        
        \subsection{Giải quyết vấn đề về bảo toàn bề mặt gia công và chất lượng gia công}
            \subsubsection{Loại bỏ ứng suất dư và biến dạng cơ học}
                \begin{itemize}
                    \item ECM: do cơ chế hòa tan nguyên tử không tiếp xúc, ECM không gây ra bất kỳ ứng suất cơ học nào lên bề mặt. Điều này cần thiết cho sản xuất các chi tiết chịu tải trọng lớn.
                    \item WJC: Cắt lạnh bằng nước không làm thay đổi cấu trúc vật liệu và không gây ứng suất nhiệt, giải quyết vấn đề cong vênh của các tấm kim loại mỏng sau khi cắt.
                \end{itemize}
            \subsubsection{Khử Bavia (Deburring)}
                \begin{itemize}
                    \item Giải pháp ECD (Electrochemical Deburring): Điện cực được đưa vào gần vị trí có bavia. Do mật độ dòng điện tập trung cao nhất tại các đỉnh nhọn (hiệu ứng mũi nhọn), bavia bị hòa tan cực nhanh trong khi kích thước lỗ cơ bản không bị ảnh hưởng.
                    \item Giải pháp AJM: Dùng để tẩy bavia cho các chi tiết nhỏ, tinh vi, hoặc làm sạch khuôn mà không làm mòn bề mặt khuôn quá mức.
                \end{itemize}
            \subsubsection{Bảo toàn Cấu trúc Luyện kim (Metallurgical Integrity)}
                \hspace*{0.6cm}Giải pháp ECG (Mài điện hóa): ECG là sự kết hợp giữa mài và điện hóa. Trong đó, 95\% vật liệu được bóc tách bằng điện hóa, chỉ 5\% bằng hạt mài cơ học. Điều này giúp giảm đáng kể nhiệt sinh ra, giải quyết vấn đề nứt tế vi (micro-cracks) và biến cứng bề mặt thường gặp khi mài các vật liệu cacbit hay thép gió. 
        \subsection{Giải quyết bài toán kinh tế và hiệu quả sản xuất}
                \subsubsection{Tối ưu chỉ phí dụng cụ cắt}
                    \begin{itemize}
                        \item Vấn đề mòn dao: Khi gia công vật liệu siêu cứng, dao mòn nhanh làm thay đổi kích thước chi tiết liên tục, đòi hỏi phải bù dao hoặc thay dao thường xuyên.
                        \item Giải pháp ECM: Trong ECM, dụng cụ (catốt) không bị tiêu hao trong quá trình điện phân dụng cụ có thể gia công hàng ngàn chi tiết với độ chính xác không đổi. Điều này giải quyết bài toán kinh tế cho sản xuất hàng loạt các chi tiết phức tạp
                        \item Giải pháp Wire-EDM: Dây cắt trong Wire-EDM được cấp liên tục từ cuộn dây mới. Do đó, "lưỡi cắt" luôn luôn mới và có kích thước chuẩn
                    \end{itemize}
                \subsubsection{Giảm thiểu quy trình}
                    \begin{itemize}
                        \item Tích hợp: Một máy Wire-EDM có thể thay thế cho cả phay, bào, và mài trong việc chế tạo chày cối, thực hiện xong chỉ trong một lần gá đặt (one setup).
                    \end{itemize}
                \subsubsection{Tự động hóa và linh hoạt}
                    \hspace*{0.6cm}Khả năng kết hợp được với CNC, giúp các phương pháp hoạt động chính xác và hiệu quả với các chương trình được lập trình sẵn, điều này giúp loại bỏ yếu tố sai số từ tay nghề thợ, giảm chi phí nhân công và đảm bảo tính lặp lại trong quy trình sản xuất.
                    
            \item \textbf{Đường dẫn (Runner):}  
            Phân phối kim loại lỏng từ ống rót
            đến các cửa vào khuôn.

            \item \textbf{Cửa vào khuôn (Ingate):}  
            Điều khiển hướng và tốc độ kim loại lỏng
            đi vào lòng khuôn.
        \end{itemize}

        \subsection{Hệ thống bù và thoát khí}

        \begin{itemize}
            \item \textbf{Đậu ngót (Riser):}  
            Cung cấp kim loại lỏng bổ sung
            để bù lại sự co ngót của vật đúc khi đông đặc.

            \item \textbf{Lỗ thoát khí (Vent):}  
            Cho phép không khí và khí sinh ra
            trong khuôn thoát ra ngoài,
            tránh rỗ khí trong vật đúc.
        \end{itemize}

        \subsection{Các bộ phận đặc trưng của khuôn kim loại và khuôn đặc biệt}

        \begin{itemize}
            \item \textbf{Nửa khuôn cố định và nửa khuôn di động:}  
            Hai phần chính của khuôn kim loại,
            cho phép đóng mở khuôn và tháo sản phẩm.

            \item \textbf{Buồng ép (Shot chamber):}  
            Chứa kim loại lỏng trước khi ép vào khuôn
            trong đúc áp lực.

            \item \textbf{Piston ép (Plunger):}  
            Tạo áp suất cao để đẩy kim loại lỏng
            điền đầy lòng khuôn.

            \item \textbf{Chốt đẩy (Ejector pin):}  
            Đẩy vật đúc ra khỏi khuôn sau khi đông đặc.

            \item \textbf{Hệ thống làm mát:}  
            Điều khiển tốc độ đông đặc của kim loại,
            nâng cao chất lượng và tuổi thọ khuôn.
        \end{itemize}
