\chapter{CÁC CÂU HỎI ÔN TẬP}
    \section{Các phuơng pháp đúc}
        \subsection{Đúc cát (Sand Casting)}
            \begin{itemize}
                \item \textbf{Nguyên lý}: Khuôn được tạo bằng cách đầm cát quanh mẫu, tháo mẫu và rót kim loại lỏng vào lòng khuôn.
                \item \textbf{Ưu điểm}: Đúc được hầu như mọi kim loại; không giới hạn kích thước và khối lượng; chi phí khuôn thấp.
                \item \textbf{Hạn chế}: Độ chính xác và chất lượng bề mặt thấp; dung sai rộng; thường cần gia công sau đúc.
                \item \textbf{Sản phẩm}: Block động cơ, thân máy, khung máy, bánh đà, chi tiết lớn.
            \end{itemize}

            \subsection{Đúc khuôn vỏ (Shell Molding)}
            \begin{itemize}
                \item \textbf{Nguyên lý}: Cát trộn nhựa được phủ lên mẫu kim loại nung nóng để tạo vỏ khuôn mỏng và cứng.
                \item \textbf{Ưu điểm}: Độ chính xác kích thước và bề mặt tốt hơn đúc cát; năng suất cao.
                \item \textbf{Hạn chế}: Kích thước chi tiết hạn chế; mẫu và thiết bị đắt.
                \item \textbf{Sản phẩm}: Vỏ hộp số, chi tiết cơ khí chính xác, chi tiết động cơ nhỏ.
            \end{itemize}

            \subsection{Đúc mẫu cháy (Evaporative Pattern / Lost Foam)}
            \begin{itemize}
                \item \textbf{Nguyên lý}: Mẫu xốp được đặt trong khuôn cát và bị hóa hơi khi rót kim loại nóng chảy.
                \item \textbf{Ưu điểm}: Tạo hình rất phức tạp; không cần mặt phân khuôn.
                \item \textbf{Hạn chế}: Mẫu xốp yếu; không kinh tế cho sản lượng nhỏ.
                \item \textbf{Sản phẩm}: Block động cơ ô tô, chi tiết phức tạp.
            \end{itemize}

            \subsection{Đúc khuôn thạch cao (Plaster Mold Casting)}
            \begin{itemize}
                \item \textbf{Nguyên lý}: Khuôn được làm từ thạch cao bao quanh mẫu, sau đó sấy khô và rót kim loại.
                \item \textbf{Ưu điểm}: Độ chính xác cao; bề mặt nhẵn; đúc được thành mỏng.
                \item \textbf{Hạn chế}: Chỉ dùng cho kim loại có nhiệt độ nóng chảy thấp; khuôn dùng một lần.
                \item \textbf{Sản phẩm}: Chi tiết nhôm, kẽm, phụ kiện chính xác, chi tiết trang trí.
            \end{itemize}

            \subsection{Đúc khuôn gốm (Ceramic Mold Casting)}
            \begin{itemize}
                \item \textbf{Nguyên lý}: Sử dụng vật liệu gốm chịu nhiệt để tạo khuôn đúc chính xác.
                \item \textbf{Ưu điểm}: Đúc được chi tiết phức tạp; độ chính xác cao.
                \item \textbf{Hạn chế}: Kích thước chi tiết hạn chế; chi phí khuôn cao.
                \item \textbf{Sản phẩm}: Cánh tuabin nhỏ, chi tiết chịu nhiệt.
            \end{itemize}

            \subsection{Đúc mẫu chảy (Investment Casting)}
            \begin{itemize}
                \item \textbf{Nguyên lý}: Mẫu sáp được phủ gốm, sau đó nung chảy mẫu và rót kim loại vào vỏ khuôn.
                \item \textbf{Ưu điểm}: Độ chính xác và chất lượng bề mặt rất cao; đúc được hầu hết kim loại.
                \item \textbf{Hạn chế}: Chi phí cao; quy trình phức tạp; kích thước chi tiết hạn chế.
                \item \textbf{Sản phẩm}: Cánh tuabin, chi tiết hàng không, y sinh, bánh răng chính xác.
            \end{itemize}

            \subsection{Đúc khuôn kim loại (Permanent Mold Casting)}
            \begin{itemize}
                \item \textbf{Nguyên lý}: Kim loại lỏng được rót vào khuôn kim loại có thể tái sử dụng nhiều lần.
                \item \textbf{Ưu điểm}: Độ chính xác và bề mặt tốt; ít rỗ; năng suất cao.
                \item \textbf{Hạn chế}: Chi phí khuôn cao; hình dạng chi tiết bị giới hạn.
                \item \textbf{Sản phẩm}: Piston, vỏ bơm, chi tiết nhôm.
            \end{itemize}

            \subsection{Đúc áp lực (Die Casting)}
            \begin{itemize}
                \item \textbf{Nguyên lý}: Kim loại lỏng được ép vào khuôn kim loại bằng áp suất cao.
                \item \textbf{Ưu điểm}: Độ chính xác rất cao; bề mặt rất tốt; phù hợp sản xuất hàng loạt.
                \item \textbf{Hạn chế}: Khuôn rất đắt; giới hạn kích thước; chủ yếu dùng kim loại màu.
                \item \textbf{Sản phẩm}: Vỏ điện tử, vỏ động cơ nhỏ, linh kiện ô tô.
            \end{itemize}

            \subsection{Đúc ly tâm (Centrifugal Casting)}
            \begin{itemize}
                \item \textbf{Nguyên lý}: Kim loại lỏng được đổ vào khuôn quay, nhờ lực ly tâm tạo hình.
                \item \textbf{Ưu điểm}: Chất lượng cao; ít khuyết tật; cơ tính tốt.
                \item \textbf{Hạn chế}: Chỉ tạo được chi tiết dạng tròn hoặc ống; thiết bị đắt.
                \item \textbf{Sản phẩm}: Ống, bạc lót, vòng bi, ống chịu áp lực.
            \end{itemize}