\chapter{CÁC CÂU HỎI ÔN TẬP}
    \section{Các phuơng pháp đúc}
        \subsection{Đúc cát (Sand Casting)}
            \begin{itemize}
                \item \textbf{Nguyên lý}: Khuôn được tạo bằng cách đầm cát quanh mẫu, tháo mẫu và rót kim loại lỏng vào lòng khuôn.
                \item \textbf{Ưu điểm}: Đúc được hầu như mọi kim loại; không giới hạn kích thước và khối lượng; chi phí khuôn thấp.
                \item \textbf{Hạn chế}: Độ chính xác và chất lượng bề mặt thấp; dung sai rộng; thường cần gia công sau đúc.
                \item \textbf{Sản phẩm}: Block động cơ, thân máy, khung máy, bánh đà, chi tiết lớn.
            \end{itemize}

            \subsection{Đúc khuôn vỏ (Shell Molding)}
            \begin{itemize}
                \item \textbf{Nguyên lý}: Cát trộn nhựa được phủ lên mẫu kim loại nung nóng để tạo vỏ khuôn mỏng và cứng.
                \item \textbf{Ưu điểm}: Độ chính xác kích thước và bề mặt tốt hơn đúc cát; năng suất cao.
                \item \textbf{Hạn chế}: Kích thước chi tiết hạn chế; mẫu và thiết bị đắt.
                \item \textbf{Sản phẩm}: Vỏ hộp số, chi tiết cơ khí chính xác, chi tiết động cơ nhỏ.
            \end{itemize}

            \subsection{Đúc mẫu cháy (Evaporative Pattern / Lost Foam)}
            \begin{itemize}
                \item \textbf{Nguyên lý}: Mẫu xốp được đặt trong khuôn cát và bị hóa hơi khi rót kim loại nóng chảy.
                \item \textbf{Ưu điểm}: Tạo hình rất phức tạp; không cần mặt phân khuôn.
                \item \textbf{Hạn chế}: Mẫu xốp yếu; không kinh tế cho sản lượng nhỏ.
                \item \textbf{Sản phẩm}: Block động cơ ô tô, chi tiết phức tạp.
            \end{itemize}

            \subsection{Đúc khuôn thạch cao (Plaster Mold Casting)}
            \begin{itemize}
                \item \textbf{Nguyên lý}: Khuôn được làm từ thạch cao bao quanh mẫu, sau đó sấy khô và rót kim loại.
                \item \textbf{Ưu điểm}: Độ chính xác cao; bề mặt nhẵn; đúc được thành mỏng.
                \item \textbf{Hạn chế}: Chỉ dùng cho kim loại có nhiệt độ nóng chảy thấp; khuôn dùng một lần.
                \item \textbf{Sản phẩm}: Chi tiết nhôm, kẽm, phụ kiện chính xác, chi tiết trang trí.
            \end{itemize}

            \subsection{Đúc khuôn gốm (Ceramic Mold Casting)}
            \begin{itemize}
                \item \textbf{Nguyên lý}: Sử dụng vật liệu gốm chịu nhiệt để tạo khuôn đúc chính xác.
                \item \textbf{Ưu điểm}: Đúc được chi tiết phức tạp; độ chính xác cao.
                \item \textbf{Hạn chế}: Kích thước chi tiết hạn chế; chi phí khuôn cao.
                \item \textbf{Sản phẩm}: Cánh tuabin nhỏ, chi tiết chịu nhiệt.
            \end{itemize}

            \subsection{Đúc mẫu chảy (Investment Casting)}
            \begin{itemize}
                \item \textbf{Nguyên lý}: Mẫu sáp được phủ gốm, sau đó nung chảy mẫu và rót kim loại vào vỏ khuôn.
                \item \textbf{Ưu điểm}: Độ chính xác và chất lượng bề mặt rất cao; đúc được hầu hết kim loại.
                \item \textbf{Hạn chế}: Chi phí cao; quy trình phức tạp; kích thước chi tiết hạn chế.
                \item \textbf{Sản phẩm}: Cánh tuabin, chi tiết hàng không, y sinh, bánh răng chính xác.
            \end{itemize}

            \subsection{Đúc khuôn kim loại (Permanent Mold Casting)}
            \begin{itemize}
                \item \textbf{Nguyên lý}: Kim loại lỏng được rót vào khuôn kim loại có thể tái sử dụng nhiều lần.
                \item \textbf{Ưu điểm}: Độ chính xác và bề mặt tốt; ít rỗ; năng suất cao.
                \item \textbf{Hạn chế}: Chi phí khuôn cao; hình dạng chi tiết bị giới hạn.
                \item \textbf{Sản phẩm}: Piston, vỏ bơm, chi tiết nhôm.
            \end{itemize}

            \subsection{Đúc áp lực (Die Casting)}
            \begin{itemize}
                \item \textbf{Nguyên lý}: Kim loại lỏng được ép vào khuôn kim loại bằng áp suất cao.
                \item \textbf{Ưu điểm}: Độ chính xác rất cao; bề mặt rất tốt; phù hợp sản xuất hàng loạt.
                \item \textbf{Hạn chế}: Khuôn rất đắt; giới hạn kích thước; chủ yếu dùng kim loại màu.
                \item \textbf{Sản phẩm}: Vỏ điện tử, vỏ động cơ nhỏ, linh kiện ô tô.
            \end{itemize}

            \subsection{Đúc ly tâm (Centrifugal Casting)}
            \begin{itemize}
                \item \textbf{Nguyên lý}: Kim loại lỏng được đổ vào khuôn quay, nhờ lực ly tâm tạo hình.
                \item \textbf{Ưu điểm}: Chất lượng cao; ít khuyết tật; cơ tính tốt.
                \item \textbf{Hạn chế}: Chỉ tạo được chi tiết dạng tròn hoặc ống; thiết bị đắt.
                \item \textbf{Sản phẩm}: Ống, bạc lót, vòng bi, ống chịu áp lực.
            \end{itemize}
    \section{Các phương pháp gia công biến dạng và sản phẩm của chúng}

        \subsection{Cán (Rolling)}
        \begin{itemize}
            \item \textbf{Phân loại:} Biến dạng khối (Bulk deformation).
            \item \textbf{Nguyên lý:}
            Phôi kim loại được đưa qua khe hở giữa hai hoặc nhiều trục cán quay,
            chịu ứng suất nén làm giảm chiều dày và tăng chiều dài phôi.
            \item \textbf{Ưu điểm:}
            \begin{itemize}
                \item Năng suất rất cao, phù hợp sản xuất hàng loạt.
                \item Có thể tạo ra phôi dài với tiết diện tương đối đồng đều.
            \end{itemize}

            \item \textbf{Hạn chế:}
            \begin{itemize}
                \item Độ chính xác hình dạng và kích thước hạn chế.
                \item Thường cần gia công tiếp theo.
            \end{itemize}
            \item \textbf{Sản phẩm tiêu biểu:}
            Tấm, lá kim loại; thanh, thép hình; ray đường sắt; phôi cán cho các quá trình khác.
        \end{itemize}

        \subsection{Rèn (Forging)}
            \begin{itemize}
                \item \textbf{Phân loại:} Biến dạng khối (Bulk deformation).

                \item \textbf{Nguyên lý:}
                Kim loại được biến dạng dẻo dưới tác dụng của lực nén lớn từ búa rèn
                hoặc máy ép, thường thông qua khuôn rèn.

                \item \textbf{Ưu điểm:}
                \begin{itemize}
                    \item Cơ tính cao nhờ dòng hạt liên tục.
                    \item Ít khuyết tật bên trong.
                \end{itemize}

                \item \textbf{Hạn chế:}
                \begin{itemize}
                    \item Chi phí khuôn cao.
                    \item Độ chính xác kích thước chưa cao.
                \end{itemize}

                \item \textbf{Sản phẩm tiêu biểu:}
                Trục khuỷu, bánh răng, tay biên, bu lông, các chi tiết chịu tải lớn.

            \end{itemize}
        
        \subsection{Ép đùn (Extrusion)}
            \begin{itemize}
                \item \textbf{Phân loại:} Biến dạng khối (Bulk deformation).

                \item \textbf{Nguyên lý:}
                Kim loại được ép chảy qua lỗ khuôn để tạo ra sản phẩm
                có tiết diện không đổi theo chiều dài.

                \item \textbf{Ưu điểm:}
                \begin{itemize}
                    \item Tạo được tiết diện phức tạp.
                    \item Bề mặt sản phẩm tương đối tốt.
                \end{itemize}

                \item \textbf{Hạn chế:}
                \begin{itemize}
                    \item Lực ép lớn.
                    \item Chiều dài sản phẩm bị giới hạn.
                \end{itemize}

                \item \textbf{Sản phẩm tiêu biểu:}
                Thanh nhôm định hình, ống, profile cửa, khung kết cấu nhẹ.
            \end{itemize}

        \subsection{Kéo dây và kéo thanh (Wire and Bar Drawing)}
        \begin{itemize}
            \item \textbf{Phân loại:} Biến dạng khối (Bulk deformation).

            \item \textbf{Nguyên lý:}
            Phôi kim loại được kéo qua lỗ khuôn,
            làm giảm tiết diện và tăng chiều dài.

            \item \textbf{Ưu điểm:}
            \begin{itemize}
                \item Độ chính xác kích thước cao.
                \item Bề mặt sản phẩm tốt.
            \end{itemize}

            \item \textbf{Hạn chế:}
            \begin{itemize}
                \item Giới hạn mức biến dạng.
                \item Có thể cần ủ trung gian.
            \end{itemize}

            \item \textbf{Sản phẩm tiêu biểu:}
            Dây điện, dây thép, dây lò xo, thanh tròn chính xác.
        \end{itemize}

        \subsection{Cắt kim loại tấm (Shearing / Blanking / Punching)}
        \begin{itemize}
            \item \textbf{Phân loại:} Gia công kim loại tấm (Sheet metalworking).

            \item \textbf{Nguyên lý:}
            Kim loại tấm bị tách rời dưới tác dụng của ứng suất trượt
            sinh ra bởi chày và cối.

            \item \textbf{Ưu điểm:}
            \begin{itemize}
                \item Năng suất cao.
                \item Độ chính xác biên dạng tốt.
            \end{itemize}

            \item \textbf{Hạn chế:}
            \begin{itemize}
                \item Tạo ứng suất dư và bavia ở mép cắt.
            \end{itemize}

            \item \textbf{Sản phẩm tiêu biểu:}
            Phôi dập, lỗ trên tấm, chi tiết phẳng.
        \end{itemize}

        \subsection{Uốn (Bending)}
        \begin{itemize}
            \item \textbf{Phân loại:} Gia công kim loại tấm (Sheet metalworking).

            \item \textbf{Nguyên lý:}
            Kim loại tấm bị biến dạng uốn quanh trục trung hòa
            dưới tác dụng của chày và cối.

            \item \textbf{Ưu điểm:}
            \begin{itemize}
                \item Thiết bị đơn giản.
                \item Dễ tự động hóa.
            \end{itemize}

            \item \textbf{Hạn chế:}
            \begin{itemize}
                \item Hiện tượng hồi phục đàn hồi (springback).
            \end{itemize}

            \item \textbf{Sản phẩm tiêu biểu:}
            Khung, giá đỡ, vỏ máy, chi tiết chữ U, V, L.
        \end{itemize}

        \subsection{Dập vuốt sâu (Deep Drawing)}
        \begin{itemize}
            \item \textbf{Phân loại:} Gia công kim loại tấm (Sheet metalworking).

            \item \textbf{Nguyên lý:}
            Tấm kim loại được biến dạng thành chi tiết rỗng
            nhờ chày, cối và lực giữ phôi.

            \item \textbf{Ưu điểm:}
            \begin{itemize}
                \item Tạo chi tiết rỗng liền khối.
                \item Ít phế liệu.
            \end{itemize}

            \item \textbf{Hạn chế:}
            \begin{itemize}
                \item Dễ nhăn hoặc rách nếu thông số không phù hợp.
            \end{itemize}

            \item \textbf{Sản phẩm tiêu biểu:}
            Lon nước giải khát, vỏ pin, vỏ hộp kim loại, chậu.
        \end{itemize}

        \subsection{Tạo hình lăn (Roll Forming)}
        \begin{itemize}
            \item \textbf{Phân loại:} Gia công kim loại tấm (Sheet metalworking).

            \item \textbf{Nguyên lý:}
            Tấm kim loại liên tục đi qua dãy trục lăn
            để dần dần đạt hình dạng mong muốn.

            \item \textbf{Ưu điểm:}
            \begin{itemize}
                \item Phù hợp sản xuất liên tục.
                \item Hình dạng ổn định.
            \end{itemize}

            \item \textbf{Hạn chế:}
            \begin{itemize}
                \item Chi phí đầu tư ban đầu lớn.
            \end{itemize}

            \item \textbf{Sản phẩm tiêu biểu:}
            Tôn lợp, thanh U--C, máng xối, panel kim loại.
        \end{itemize}

        \subsection{Ép xoay (Spinning)}
        \begin{itemize}
            \item \textbf{Phân loại:} Gia công kim loại tấm (Sheet metalworking).

            \item \textbf{Nguyên lý:}
            Tấm kim loại quay cùng khuôn,
            con lăn ép tấm sát vào khuôn để tạo hình đối xứng trục.

            \item \textbf{Ưu điểm:}
            \begin{itemize}
                \item Không cần khuôn phức tạp.
                \item Phù hợp sản xuất loạt nhỏ.
            \end{itemize}

            \item \textbf{Hạn chế:}
            \begin{itemize}
                \item Độ chính xác phụ thuộc điều khiển hoặc tay nghề.
            \end{itemize}

            \item \textbf{Sản phẩm tiêu biểu:}
            Chao đèn, chén kim loại, vỏ đối xứng trục.
        \end{itemize}
    \section{Các phương pháp gia công cắt gọt và khả năng thực hiện}
        \subsection{Tiện (Turning)}
        \begin{itemize}
            \item \textbf{Phân loại:} Gia công cắt gọt truyền thống (Conventional machining).

            \item \textbf{Nguyên lý:}
            Chi tiết quay quanh trục chính, dao tiện tịnh tiến theo phương dọc hoặc ngang
            để bóc tách vật liệu.

            \item \textbf{Ưu điểm:}
            \begin{itemize}
                \item Gia công được bề mặt tròn xoay chính xác.
                \item Máy và dao tương đối đơn giản.
            \end{itemize}

            \item \textbf{Hạn chế:}
            \begin{itemize}
                \item Chỉ gia công hiệu quả chi tiết dạng tròn xoay.
            \end{itemize}

            \item \textbf{Khả năng thực hiện:}
            Gia công thô và tinh; tiện trụ ngoài, trụ trong, mặt đầu, côn, ren.

            \item \textbf{Sản phẩm tiêu biểu:}
            Trục, bạc, ống, trục ren, chi tiết tròn xoay.
        \end{itemize}

        %================================================
        \subsection{Phay (Milling)}
        \begin{itemize}
            \item \textbf{Phân loại:} Gia công cắt gọt truyền thống.

            \item \textbf{Nguyên lý:}
            Dao phay quay, chi tiết hoặc dao tịnh tiến để cắt bỏ vật liệu từng lớp.

            \item \textbf{Ưu điểm:}
            \begin{itemize}
                \item Gia công được bề mặt phẳng và biên dạng phức tạp.
                \item Linh hoạt, dễ tự động hóa CNC.
            \end{itemize}

            \item \textbf{Hạn chế:}
            \begin{itemize}
                \item Lực cắt gián đoạn gây rung.
            \end{itemize}

            \item \textbf{Khả năng thực hiện:}
            Phay mặt, phay rãnh, phay biên dạng, phay khuôn 3D.

            \item \textbf{Sản phẩm tiêu biểu:}
            Khuôn mẫu, vỏ máy, mặt phẳng chính xác, chi tiết cơ khí phức tạp.
        \end{itemize}

        %================================================
        \subsection{Khoan (Drilling)}
        \begin{itemize}
            \item \textbf{Phân loại:} Gia công cắt gọt tạo lỗ.

            \item \textbf{Nguyên lý:}
            Mũi khoan quay và tiến dọc trục để tạo lỗ mới trên chi tiết.

            \item \textbf{Ưu điểm:}
            \begin{itemize}
                \item Thiết bị đơn giản, phổ biến.
            \end{itemize}

            \item \textbf{Hạn chế:}
            \begin{itemize}
                \item Độ chính xác và độ nhẵn bề mặt lỗ không cao.
            \end{itemize}

            \item \textbf{Khả năng thực hiện:}
            Tạo lỗ thô; thường cần doa hoặc khoét sau đó.

            \item \textbf{Sản phẩm tiêu biểu:}
            Lỗ lắp bu lông, lỗ dẫn, lỗ chuẩn.
        \end{itemize}

        %================================================
        \subsection{Doa (Reaming)}
        \begin{itemize}
            \item \textbf{Phân loại:} Gia công tinh lỗ.

            \item \textbf{Nguyên lý:}
            Dao doa cắt lớp mỏng để cải thiện kích thước và độ nhẵn lỗ.

            \item \textbf{Ưu điểm:}
            \begin{itemize}
                \item Độ chính xác kích thước cao.
            \end{itemize}

            \item \textbf{Hạn chế:}
            \begin{itemize}
                \item Không dùng để tạo lỗ mới.
            \end{itemize}

            \item \textbf{Khả năng thực hiện:}
            Gia công tinh lỗ sau khoan.

            \item \textbf{Sản phẩm tiêu biểu:}
            Lỗ lắp ổ trục, chốt định vị.
        \end{itemize}

        %================================================
        \subsection{Bào và xọc (Shaping and Slotting)}
        \begin{itemize}
            \item \textbf{Phân loại:} Gia công cắt gọt truyền thống.

            \item \textbf{Nguyên lý:}
            Dao chuyển động tịnh tiến qua lại để bóc vật liệu.

            \item \textbf{Ưu điểm:}
            \begin{itemize}
                \item Kết cấu máy đơn giản.
            \end{itemize}

            \item \textbf{Hạn chế:}
            \begin{itemize}
                \item Năng suất thấp.
            \end{itemize}

            \item \textbf{Khả năng thực hiện:}
            Gia công mặt phẳng, rãnh then.

            \item \textbf{Sản phẩm tiêu biểu:}
            Rãnh then, mặt phẳng đơn giản.
        \end{itemize}

        %================================================
        \subsection{Chuốt (Broaching)}
        \begin{itemize}
            \item \textbf{Phân loại:} Gia công cắt gọt năng suất cao.

            \item \textbf{Nguyên lý:}
            Dao chuốt nhiều răng cắt tăng dần,
            mỗi răng bóc một lượng nhỏ vật liệu.

            \item \textbf{Ưu điểm:}
            \begin{itemize}
                \item Độ chính xác cao.
                \item Năng suất rất lớn.
            \end{itemize}

            \item \textbf{Hạn chế:}
            \begin{itemize}
                \item Dao chuốt rất đắt, kém linh hoạt.
            \end{itemize}

            \item \textbf{Khả năng thực hiện:}
            Gia công rãnh then, then hoa, lỗ định hình.

            \item \textbf{Sản phẩm tiêu biểu:}
            Bánh răng trong, then hoa, lỗ đa giác.
        \end{itemize}

        %================================================
        \subsection{Mài (Grinding)}
        \begin{itemize}
            \item \textbf{Phân loại:} Gia công mài mòn (Abrasive machining).

            \item \textbf{Nguyên lý:}
            Dùng hạt mài quay tốc độ cao để bóc lớp vật liệu rất mỏng.

            \item \textbf{Ưu điểm:}
            \begin{itemize}
                \item Độ chính xác và độ nhẵn bề mặt rất cao.
            \end{itemize}

            \item \textbf{Hạn chế:}
            \begin{itemize}
                \item Năng suất thấp.
                \item Dễ sinh nhiệt.
            \end{itemize}

            \item \textbf{Khả năng thực hiện:}
            Gia công tinh, gia công vật liệu cứng.

            \item \textbf{Sản phẩm tiêu biểu:}
            Trục chính xác, bề mặt khuôn, chi tiết sau nhiệt luyện.
        \end{itemize}



            
        