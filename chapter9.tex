\chapter{CÁC CÂU HỎI ÔN TẬP}
    \section{Các phuơng pháp đúc}
        \subsection{Đúc cát (Sand Casting)}
            \begin{itemize}
                \item \textbf{Nguyên lý}: Khuôn được tạo bằng cách đầm cát quanh mẫu, tháo mẫu và rót kim loại lỏng vào lòng khuôn.
                \item \textbf{Ưu điểm}: Đúc được hầu như mọi kim loại; không giới hạn kích thước và khối lượng; chi phí khuôn thấp.
                \item \textbf{Hạn chế}: Độ chính xác và chất lượng bề mặt thấp; dung sai rộng; thường cần gia công sau đúc.
                \item \textbf{Sản phẩm}: Block động cơ, thân máy, khung máy, bánh đà, chi tiết lớn.
            \end{itemize}

            \subsection{Đúc khuôn vỏ (Shell Molding)}
            \begin{itemize}
                \item \textbf{Nguyên lý}: Cát trộn nhựa được phủ lên mẫu kim loại nung nóng để tạo vỏ khuôn mỏng và cứng.
                \item \textbf{Ưu điểm}: Độ chính xác kích thước và bề mặt tốt hơn đúc cát; năng suất cao.
                \item \textbf{Hạn chế}: Kích thước chi tiết hạn chế; mẫu và thiết bị đắt.
                \item \textbf{Sản phẩm}: Vỏ hộp số, chi tiết cơ khí chính xác, chi tiết động cơ nhỏ.
            \end{itemize}

            \subsection{Đúc mẫu cháy (Evaporative Pattern / Lost Foam)}
            \begin{itemize}
                \item \textbf{Nguyên lý}: Mẫu xốp được đặt trong khuôn cát và bị hóa hơi khi rót kim loại nóng chảy.
                \item \textbf{Ưu điểm}: Tạo hình rất phức tạp; không cần mặt phân khuôn.
                \item \textbf{Hạn chế}: Mẫu xốp yếu; không kinh tế cho sản lượng nhỏ.
                \item \textbf{Sản phẩm}: Block động cơ ô tô, chi tiết phức tạp.
            \end{itemize}

            \subsection{Đúc khuôn thạch cao (Plaster Mold Casting)}
            \begin{itemize}
                \item \textbf{Nguyên lý}: Khuôn được làm từ thạch cao bao quanh mẫu, sau đó sấy khô và rót kim loại.
                \item \textbf{Ưu điểm}: Độ chính xác cao; bề mặt nhẵn; đúc được thành mỏng.
                \item \textbf{Hạn chế}: Chỉ dùng cho kim loại có nhiệt độ nóng chảy thấp; khuôn dùng một lần.
                \item \textbf{Sản phẩm}: Chi tiết nhôm, kẽm, phụ kiện chính xác, chi tiết trang trí.
            \end{itemize}

            \subsection{Đúc khuôn gốm (Ceramic Mold Casting)}
            \begin{itemize}
                \item \textbf{Nguyên lý}: Sử dụng vật liệu gốm chịu nhiệt để tạo khuôn đúc chính xác.
                \item \textbf{Ưu điểm}: Đúc được chi tiết phức tạp; độ chính xác cao.
                \item \textbf{Hạn chế}: Kích thước chi tiết hạn chế; chi phí khuôn cao.
                \item \textbf{Sản phẩm}: Cánh tuabin nhỏ, chi tiết chịu nhiệt.
            \end{itemize}

            \subsection{Đúc mẫu chảy (Investment Casting)}
            \begin{itemize}
                \item \textbf{Nguyên lý}: Mẫu sáp được phủ gốm, sau đó nung chảy mẫu và rót kim loại vào vỏ khuôn.
                \item \textbf{Ưu điểm}: Độ chính xác và chất lượng bề mặt rất cao; đúc được hầu hết kim loại.
                \item \textbf{Hạn chế}: Chi phí cao; quy trình phức tạp; kích thước chi tiết hạn chế.
                \item \textbf{Sản phẩm}: Cánh tuabin, chi tiết hàng không, y sinh, bánh răng chính xác.
            \end{itemize}

            \subsection{Đúc khuôn kim loại (Permanent Mold Casting)}
            \begin{itemize}
                \item \textbf{Nguyên lý}: Kim loại lỏng được rót vào khuôn kim loại có thể tái sử dụng nhiều lần.
                \item \textbf{Ưu điểm}: Độ chính xác và bề mặt tốt; ít rỗ; năng suất cao.
                \item \textbf{Hạn chế}: Chi phí khuôn cao; hình dạng chi tiết bị giới hạn.
                \item \textbf{Sản phẩm}: Piston, vỏ bơm, chi tiết nhôm.
            \end{itemize}

            \subsection{Đúc áp lực (Die Casting)}
            \begin{itemize}
                \item \textbf{Nguyên lý}: Kim loại lỏng được ép vào khuôn kim loại bằng áp suất cao.
                \item \textbf{Ưu điểm}: Độ chính xác rất cao; bề mặt rất tốt; phù hợp sản xuất hàng loạt.
                \item \textbf{Hạn chế}: Khuôn rất đắt; giới hạn kích thước; chủ yếu dùng kim loại màu.
                \item \textbf{Sản phẩm}: Vỏ điện tử, vỏ động cơ nhỏ, linh kiện ô tô.
            \end{itemize}

            \subsection{Đúc ly tâm (Centrifugal Casting)}
            \begin{itemize}
                \item \textbf{Nguyên lý}: Kim loại lỏng được đổ vào khuôn quay, nhờ lực ly tâm tạo hình.
                \item \textbf{Ưu điểm}: Chất lượng cao; ít khuyết tật; cơ tính tốt.
                \item \textbf{Hạn chế}: Chỉ tạo được chi tiết dạng tròn hoặc ống; thiết bị đắt.
                \item \textbf{Sản phẩm}: Ống, bạc lót, vòng bi, ống chịu áp lực.
            \end{itemize}
    \section{Các phương pháp gia công biến dạng và sản phẩm của chúng}

        \subsection{Cán (Rolling)}
        \begin{itemize}
            \item \textbf{Phân loại:} Biến dạng khối (Bulk deformation).
            \item \textbf{Nguyên lý:}
            Phôi kim loại được đưa qua khe hở giữa hai hoặc nhiều trục cán quay,
            chịu ứng suất nén làm giảm chiều dày và tăng chiều dài phôi.
            \item \textbf{Ưu điểm:}
            \begin{itemize}
                \item Năng suất rất cao, phù hợp sản xuất hàng loạt.
                \item Có thể tạo ra phôi dài với tiết diện tương đối đồng đều.
            \end{itemize}

            \item \textbf{Hạn chế:}
            \begin{itemize}
                \item Độ chính xác hình dạng và kích thước hạn chế.
                \item Thường cần gia công tiếp theo.
            \end{itemize}
            \item \textbf{Sản phẩm tiêu biểu:}
            Tấm, lá kim loại; thanh, thép hình; ray đường sắt; phôi cán cho các quá trình khác.
        \end{itemize}

        \subsection{Rèn (Forging)}
            \begin{itemize}
                \item \textbf{Phân loại:} Biến dạng khối (Bulk deformation).

                \item \textbf{Nguyên lý:}
                Kim loại được biến dạng dẻo dưới tác dụng của lực nén lớn từ búa rèn
                hoặc máy ép, thường thông qua khuôn rèn.

                \item \textbf{Ưu điểm:}
                \begin{itemize}
                    \item Cơ tính cao nhờ dòng hạt liên tục.
                    \item Ít khuyết tật bên trong.
                \end{itemize}

                \item \textbf{Hạn chế:}
                \begin{itemize}
                    \item Chi phí khuôn cao.
                    \item Độ chính xác kích thước chưa cao.
                \end{itemize}

                \item \textbf{Sản phẩm tiêu biểu:}
                Trục khuỷu, bánh răng, tay biên, bu lông, các chi tiết chịu tải lớn.

            \end{itemize}
        
        \subsection{Ép đùn (Extrusion)}
            \begin{itemize}
                \item \textbf{Phân loại:} Biến dạng khối (Bulk deformation).

                \item \textbf{Nguyên lý:}
                Kim loại được ép chảy qua lỗ khuôn để tạo ra sản phẩm
                có tiết diện không đổi theo chiều dài.

                \item \textbf{Ưu điểm:}
                \begin{itemize}
                    \item Tạo được tiết diện phức tạp.
                    \item Bề mặt sản phẩm tương đối tốt.
                \end{itemize}

                \item \textbf{Hạn chế:}
                \begin{itemize}
                    \item Lực ép lớn.
                    \item Chiều dài sản phẩm bị giới hạn.
                \end{itemize}

                \item \textbf{Sản phẩm tiêu biểu:}
                Thanh nhôm định hình, ống, profile cửa, khung kết cấu nhẹ.
            \end{itemize}

        \subsection{Kéo dây và kéo thanh (Wire and Bar Drawing)}
        \begin{itemize}
            \item \textbf{Phân loại:} Biến dạng khối (Bulk deformation).

            \item \textbf{Nguyên lý:}
            Phôi kim loại được kéo qua lỗ khuôn,
            làm giảm tiết diện và tăng chiều dài.

            \item \textbf{Ưu điểm:}
            \begin{itemize}
                \item Độ chính xác kích thước cao.
                \item Bề mặt sản phẩm tốt.
            \end{itemize}

            \item \textbf{Hạn chế:}
            \begin{itemize}
                \item Giới hạn mức biến dạng.
                \item Có thể cần ủ trung gian.
            \end{itemize}

            \item \textbf{Sản phẩm tiêu biểu:}
            Dây điện, dây thép, dây lò xo, thanh tròn chính xác.
        \end{itemize}

        \subsection{Cắt kim loại tấm (Shearing / Blanking / Punching)}
        \begin{itemize}
            \item \textbf{Phân loại:} Gia công kim loại tấm (Sheet metalworking).

            \item \textbf{Nguyên lý:}
            Kim loại tấm bị tách rời dưới tác dụng của ứng suất trượt
            sinh ra bởi chày và cối.

            \item \textbf{Ưu điểm:}
            \begin{itemize}
                \item Năng suất cao.
                \item Độ chính xác biên dạng tốt.
            \end{itemize}

            \item \textbf{Hạn chế:}
            \begin{itemize}
                \item Tạo ứng suất dư và bavia ở mép cắt.
            \end{itemize}

            \item \textbf{Sản phẩm tiêu biểu:}
            Phôi dập, lỗ trên tấm, chi tiết phẳng.
        \end{itemize}

        \subsection{Uốn (Bending)}
        \begin{itemize}
            \item \textbf{Phân loại:} Gia công kim loại tấm (Sheet metalworking).

            \item \textbf{Nguyên lý:}
            Kim loại tấm bị biến dạng uốn quanh trục trung hòa
            dưới tác dụng của chày và cối.

            \item \textbf{Ưu điểm:}
            \begin{itemize}
                \item Thiết bị đơn giản.
                \item Dễ tự động hóa.
            \end{itemize}

            \item \textbf{Hạn chế:}
            \begin{itemize}
                \item Hiện tượng hồi phục đàn hồi (springback).
            \end{itemize}

            \item \textbf{Sản phẩm tiêu biểu:}
            Khung, giá đỡ, vỏ máy, chi tiết chữ U, V, L.
        \end{itemize}

        \subsection{Dập vuốt sâu (Deep Drawing)}
        \begin{itemize}
            \item \textbf{Phân loại:} Gia công kim loại tấm (Sheet metalworking).

            \item \textbf{Nguyên lý:}
            Tấm kim loại được biến dạng thành chi tiết rỗng
            nhờ chày, cối và lực giữ phôi.

            \item \textbf{Ưu điểm:}
            \begin{itemize}
                \item Tạo chi tiết rỗng liền khối.
                \item Ít phế liệu.
            \end{itemize}

            \item \textbf{Hạn chế:}
            \begin{itemize}
                \item Dễ nhăn hoặc rách nếu thông số không phù hợp.
            \end{itemize}

            \item \textbf{Sản phẩm tiêu biểu:}
            Lon nước giải khát, vỏ pin, vỏ hộp kim loại, chậu.
        \end{itemize}

        \subsection{Tạo hình lăn (Roll Forming)}
        \begin{itemize}
            \item \textbf{Phân loại:} Gia công kim loại tấm (Sheet metalworking).

            \item \textbf{Nguyên lý:}
            Tấm kim loại liên tục đi qua dãy trục lăn
            để dần dần đạt hình dạng mong muốn.

            \item \textbf{Ưu điểm:}
            \begin{itemize}
                \item Phù hợp sản xuất liên tục.
                \item Hình dạng ổn định.
            \end{itemize}

            \item \textbf{Hạn chế:}
            \begin{itemize}
                \item Chi phí đầu tư ban đầu lớn.
            \end{itemize}

            \item \textbf{Sản phẩm tiêu biểu:}
            Tôn lợp, thanh U--C, máng xối, panel kim loại.
        \end{itemize}

        \subsection{Ép xoay (Spinning)}
        \begin{itemize}
            \item \textbf{Phân loại:} Gia công kim loại tấm (Sheet metalworking).

            \item \textbf{Nguyên lý:}
            Tấm kim loại quay cùng khuôn,
            con lăn ép tấm sát vào khuôn để tạo hình đối xứng trục.

            \item \textbf{Ưu điểm:}
            \begin{itemize}
                \item Không cần khuôn phức tạp.
                \item Phù hợp sản xuất loạt nhỏ.
            \end{itemize}

            \item \textbf{Hạn chế:}
            \begin{itemize}
                \item Độ chính xác phụ thuộc điều khiển hoặc tay nghề.
            \end{itemize}

            \item \textbf{Sản phẩm tiêu biểu:}
            Chao đèn, chén kim loại, vỏ đối xứng trục.
        \end{itemize}
    \section{Các phương pháp gia công cắt gọt và khả năng thực hiện}
        \subsection{Tiện (Turning)}
        \begin{itemize}
            \item \textbf{Phân loại:} Gia công cắt gọt truyền thống (Conventional machining).

            \item \textbf{Nguyên lý:}
            Chi tiết quay quanh trục chính, dao tiện tịnh tiến theo phương dọc hoặc ngang
            để bóc tách vật liệu.

            \item \textbf{Ưu điểm:}
            \begin{itemize}
                \item Gia công được bề mặt tròn xoay chính xác.
                \item Máy và dao tương đối đơn giản.
            \end{itemize}

            \item \textbf{Hạn chế:}
            \begin{itemize}
                \item Chỉ gia công hiệu quả chi tiết dạng tròn xoay.
            \end{itemize}

            \item \textbf{Khả năng thực hiện:}
            Gia công thô và tinh; tiện trụ ngoài, trụ trong, mặt đầu, côn, ren.

            \item \textbf{Sản phẩm tiêu biểu:}
            Trục, bạc, ống, trục ren, chi tiết tròn xoay.
        \end{itemize}

        %================================================
        \subsection{Phay (Milling)}
        \begin{itemize}
            \item \textbf{Phân loại:} Gia công cắt gọt truyền thống.

            \item \textbf{Nguyên lý:}
            Dao phay quay, chi tiết hoặc dao tịnh tiến để cắt bỏ vật liệu từng lớp.

            \item \textbf{Ưu điểm:}
            \begin{itemize}
                \item Gia công được bề mặt phẳng và biên dạng phức tạp.
                \item Linh hoạt, dễ tự động hóa CNC.
            \end{itemize}

            \item \textbf{Hạn chế:}
            \begin{itemize}
                \item Lực cắt gián đoạn gây rung.
            \end{itemize}

            \item \textbf{Khả năng thực hiện:}
            Phay mặt, phay rãnh, phay biên dạng, phay khuôn 3D.

            \item \textbf{Sản phẩm tiêu biểu:}
            Khuôn mẫu, vỏ máy, mặt phẳng chính xác, chi tiết cơ khí phức tạp.
        \end{itemize}

        %================================================
        \subsection{Khoan (Drilling)}
        \begin{itemize}
            \item \textbf{Phân loại:} Gia công cắt gọt tạo lỗ.

            \item \textbf{Nguyên lý:}
            Mũi khoan quay và tiến dọc trục để tạo lỗ mới trên chi tiết.

            \item \textbf{Ưu điểm:}
            \begin{itemize}
                \item Thiết bị đơn giản, phổ biến.
            \end{itemize}

            \item \textbf{Hạn chế:}
            \begin{itemize}
                \item Độ chính xác và độ nhẵn bề mặt lỗ không cao.
            \end{itemize}

            \item \textbf{Khả năng thực hiện:}
            Tạo lỗ thô; thường cần doa hoặc khoét sau đó.

            \item \textbf{Sản phẩm tiêu biểu:}
            Lỗ lắp bu lông, lỗ dẫn, lỗ chuẩn.
        \end{itemize}

        %================================================
        \subsection{Doa (Reaming)}
        \begin{itemize}
            \item \textbf{Phân loại:} Gia công tinh lỗ.

            \item \textbf{Nguyên lý:}
            Dao doa cắt lớp mỏng để cải thiện kích thước và độ nhẵn lỗ.

            \item \textbf{Ưu điểm:}
            \begin{itemize}
                \item Độ chính xác kích thước cao.
            \end{itemize}

            \item \textbf{Hạn chế:}
            \begin{itemize}
                \item Không dùng để tạo lỗ mới.
            \end{itemize}

            \item \textbf{Khả năng thực hiện:}
            Gia công tinh lỗ sau khoan.

            \item \textbf{Sản phẩm tiêu biểu:}
            Lỗ lắp ổ trục, chốt định vị.
        \end{itemize}

        %================================================
        \subsection{Bào và xọc (Shaping and Slotting)}
        \begin{itemize}
            \item \textbf{Phân loại:} Gia công cắt gọt truyền thống.

            \item \textbf{Nguyên lý:}
            Dao chuyển động tịnh tiến qua lại để bóc vật liệu.

            \item \textbf{Ưu điểm:}
            \begin{itemize}
                \item Kết cấu máy đơn giản.
            \end{itemize}

            \item \textbf{Hạn chế:}
            \begin{itemize}
                \item Năng suất thấp.
            \end{itemize}

            \item \textbf{Khả năng thực hiện:}
            Gia công mặt phẳng, rãnh then.

            \item \textbf{Sản phẩm tiêu biểu:}
            Rãnh then, mặt phẳng đơn giản.
        \end{itemize}

        %================================================
        \subsection{Chuốt (Broaching)}
        \begin{itemize}
            \item \textbf{Phân loại:} Gia công cắt gọt năng suất cao.

            \item \textbf{Nguyên lý:}
            Dao chuốt nhiều răng cắt tăng dần,
            mỗi răng bóc một lượng nhỏ vật liệu.

            \item \textbf{Ưu điểm:}
            \begin{itemize}
                \item Độ chính xác cao.
                \item Năng suất rất lớn.
            \end{itemize}

            \item \textbf{Hạn chế:}
            \begin{itemize}
                \item Dao chuốt rất đắt, kém linh hoạt.
            \end{itemize}

            \item \textbf{Khả năng thực hiện:}
            Gia công rãnh then, then hoa, lỗ định hình.

            \item \textbf{Sản phẩm tiêu biểu:}
            Bánh răng trong, then hoa, lỗ đa giác.
        \end{itemize}

        %================================================
        \subsection{Mài (Grinding)}
        \begin{itemize}
            \item \textbf{Phân loại:} Gia công mài mòn (Abrasive machining).

            \item \textbf{Nguyên lý:}
            Dùng hạt mài quay tốc độ cao để bóc lớp vật liệu rất mỏng.

            \item \textbf{Ưu điểm:}
            \begin{itemize}
                \item Độ chính xác và độ nhẵn bề mặt rất cao.
            \end{itemize}

            \item \textbf{Hạn chế:}
            \begin{itemize}
                \item Năng suất thấp.
                \item Dễ sinh nhiệt.
            \end{itemize}

            \item \textbf{Khả năng thực hiện:}
            Gia công tinh, gia công vật liệu cứng.

            \item \textbf{Sản phẩm tiêu biểu:}
            Trục chính xác, bề mặt khuôn, chi tiết sau nhiệt luyện.
        \end{itemize}

    \section{Phương pháp gia công kim loại bột, nguyên lý, điều kiện}
    \subsection{Nguyên lý hoạt động}
    Gia công kim loại bột là quá trình sản xuất các chi tiết kim loại bằng cách nén bột kim loại vào khuôn để tạo hình dáng mong muốn, sau đó nung nóng (thiêu kết) để các hạt bột liên kết lại với nhau thành khối rắn.

    Quy trình hoạt động dựa trên 4 giai đoạn cốt lõi:

    \subsubsection*{Giai đoạn 1: Tạo bột và Phối trộn (Powder Production \& Blending)}
    \begin{itemize}
        \item Kim loại được biến thành bột mịn (thông qua phun kim loại lỏng, khử hóa học, hoặc nghiền).
        \item Các loại bột khác nhau và chất bôi trơn được trộn đều để đảm bảo sự đồng nhất về tính chất cơ học và giảm ma sát khi ép.
    \end{itemize}

    \subsubsection*{Giai đoạn 2: Nén/Ép (Compaction)}
    \begin{itemize}
        \item Bột được nén trong khuôn dưới áp lực cao.
        \item \textbf{Mục đích}: Tạo ra hình dáng hình học, tăng mật độ tiếp xúc giữa các hạt và tạo độ bền ban đầu (được gọi là độ bền tươi - green strength) để chi tiết không bị vỡ khi xử lý tiếp theo.
        \item Sản phẩm sau bước này gọi là "chi tiết tươi" (green compact), rất giòn và dễ vỡ.
    \end{itemize}

    \subsubsection*{Giai đoạn 3: Thiêu kết (Sintering)}
    \begin{itemize}
        \item Chi tiết tươi được nung nóng trong lò có kiểm soát khí quyển.
        \item \textbf{Nhiệt độ nung}: Dưới nhiệt độ nóng chảy của kim loại chính.
        \item \textbf{Cơ chế}: Các hạt bột liên kết với nhau thông qua cơ chế khuếch tán trạng thái rắn (diffusion) hoặc vận chuyển pha hơi, làm tăng sức bền và mật độ của chi tiết.
    \end{itemize}

    \subsubsection*{Giai đoạn 4: Hoàn thiện (Finishing - Tùy chọn)}
    \begin{itemize}
        \item Các nguyên công phụ như ép lại (coining) để tăng độ chính xác, gia công cắt gọt, thấm dầu (impregnation) hoặc mạ.
    \end{itemize}

    \subsection{Các điều kiện kỹ thuật (Process Conditions)}
    Để quá trình thành công, cần tuân thủ nghiêm ngặt các điều kiện ở từng giai đoạn:

    \subsubsection*{A. Điều kiện về Bột kim loại}
    \begin{itemize}
        \item \textbf{Kích thước hạt}: Phải được kiểm soát, thường nằm trong khoảng $0.1$ đến $1000~\mu$m.
        \item \textbf{Hình dáng hạt}: Ảnh hưởng đến khả năng chảy và mật độ khi nén (hạt hình cầu chảy tốt hơn hạt hình vảy).
        \item \textbf{An toàn}: Bột kim loại (đặc biệt là Nhôm, Magie, Titan) dễ gây nổ do diện tích bề mặt lớn, cần điều kiện tiếp địa và tránh tia lửa điện khi xử lý.
    \end{itemize}

    \subsubsection*{B. Điều kiện khi Nén (Compaction)}
    \begin{itemize}
        \item \textbf{Áp suất nén}: Phải đủ lớn để tạo mật độ và độ bền tươi.
        \begin{itemize}
            \item Nhôm: $70-275$ MPa.
            \item Sắt: $350-800$ MPa.
            \item Đồng thau (Brass): $400-700$ MPa.
        \end{itemize}
        \item \textbf{Khe hở khuôn}: Khe hở giữa chày và cối phải rất nhỏ (thường $< 25~\mu$m) để tránh bột lọt vào khe gây kẹt hoặc làm lệch chi tiết.
        \item \textbf{Ma sát}: Cần giảm ma sát giữa các hạt bột và thành khuôn (dùng chất bôi trơn) để mật độ phân bố đồng đều.
    \end{itemize}

    \subsubsection*{C. Điều kiện khi Thiêu kết (Sintering)}
    Đây là bước quan trọng nhất quyết định tính chất vật liệu.
    \begin{itemize}
        \item \textbf{Nhiệt độ}: Thường nằm trong khoảng 70\% - 90\% nhiệt độ nóng chảy của kim loại/hợp kim.
        \begin{itemize}
            \item Ví dụ: Đồng/Hợp kim đồng nung ở $760-900^{\circ}$C.
            \item Sắt: $1000-1150^{\circ}$C.
        \end{itemize}
        \item \textbf{Thời gian}: Từ 10 phút (cho sắt/đồng) đến 8 giờ (cho Tungsten/Tantalum).
        \item \textbf{Môi trường lò (Khí quyển)}: Phải được kiểm soát chặt chẽ để tránh oxy hóa.
        \begin{itemize}
            \item Khử oxy (Hydrogen), môi trường khí trơ, hoặc chân không (đối với thép không gỉ hoặc kim loại chịu nhiệt).
        \end{itemize}
    \end{itemize}

    \section{Hạt mài và độ cứng của chúng như thế nào?}
    \subsection{Độ cứng của hạt mài (Abrasive Hardness)}
    Đây là khả năng của bản thân hạt mài trong việc chống lại sự trầy xước hoặc biến dạng khi tiếp xúc với phôi. Các siêu hạt mài như kim cương và CBN có độ cứng vượt trội so với phôi thép thông thường.

    Hạt mài được chia thành hai nhóm chính dựa trên khả năng cắt và độ cứng của chúng:

    \subsubsection*{Hạt mài thông thường (Conventional abrasives)}
    \begin{itemize}
        \item \textbf{Oxit nhôm (Aluminum oxide)}: Thường dùng để mài các loại thép và hợp kim có độ bền cao.
        \item \textbf{Silicon carbide}: Thường dùng cho các vật liệu giòn như gang, hoặc các kim loại mềm hơn như nhôm và đồng.
    \end{itemize}

    \subsubsection*{Siêu hạt mài (Superabrasives)}
    \begin{itemize}
        \item \textbf{Nitrit bo lập phương (CBN - Cubic Boron Nitride)}: Có độ cứng rất cao, bền nhiệt, thích hợp cho các loại thép dụng cụ cứng.
        \item \textbf{Kim cương (Diamond)}: Là vật liệu cứng nhất, thường dùng để mài các vật liệu cực cứng như gốm sứ, cacbit hoặc thủy tinh.
    \end{itemize}

\section{Các phương pháp gia công mài}
\begin{itemize}
    \item Mài bề mặt phẳng
    \begin{itemize}
        \item Trục chính nằm ngang với bàn di chuyển tịnh tiến
        \item Trục chính nằm ngang với bàn quay
        \item Trục chính thẳng đứng với bàn di chuyển tịnh tiến
        \item Trục chính thẳng đứng với bàn quay
    \end{itemize}
    \item Mài tròn trụ
    \begin{itemize}
        \item Mài ngoài tròn trụ
        \item Mài trong tròn trụ
    \end{itemize}
    \item Mài không tâm
    \item Mài tiến dao chậm
    \item Mài đai
\end{itemize}

\section{Cơ chế tạo phoi. Các dạng phoi. Điều kiện hình thành}
    \subsection{Cơ chế tạo phoi}
    Quá trình tạo phoi diễn ra thông qua sự biến dạng dẻo và trượt của vật liệu.

    \begin{itemize}
        \item \textbf{Sự trượt (Shearing)}: Khi dụng cụ cắt tiến vào phôi, vật liệu phía trước mũi dao bị nén và trượt liên tục dọc theo một mặt phẳng gọi là \textbf{mặt phẳng trượt (shear plane)}.
        
        \item \textbf{Vùng trượt}: Hiện tượng này diễn ra trong một vùng hẹp được gọi là vùng trượt chính. Cơ chế này tương tự như việc cắt lá bài trong một bộ bài trượt lên nhau.
        
        \item \textbf{Đặc điểm bề mặt}: Phoi có hai bề mặt khác nhau: mặt tiếp xúc với dao thường bóng mín do ma sát trượt, trong khi mặt ngoài có dạng răng cưa hoặc nhám do cơ chế trượt không đồng đều.
    \end{itemize}

    \subsection{Các dạng phoi và điều kiện hình thành}
    Tùy thuộc vào vật liệu phôi và các thông số cắt gọt, có 4 dạng phoi chính thường gặp:

    \subsubsection*{Phoi liên tục / Phoi dây (Continuous Chips)}
    \begin{itemize}
        \item \textbf{Đặc điểm}: Dạng dài dài, không bị đứt đoạn, cho bề mặt gia công bóng mín.
        
        \item \textbf{Điều kiện hình thành}:
        \begin{itemize}
            \item Vật liệu phôi dẻo (nhôm, thép mềm).
            \item Vận tốc cắt cao ($V$ lớn).
            \item Góc trước của dao lớn ($\alpha$ lớn).
            \item Ma sát tại mặt trước của dao thấp.
        \end{itemize}
    \end{itemize}

    \subsubsection*{Phoi có lẹo dao (Built-up Edge - BUE)}
    \begin{itemize}
        \item \textbf{Đặc điểm}: Các lớp vật liệu phôi bám chặt và tích tụ dần trên mũi dao, làm thay đổi hình dáng hình học của lưỡi cắt.
        
        \item \textbf{Điều kiện hình thành}:
        \begin{itemize}
            \item Vật liệu dẻo ở vận tốc thấp đến trung bình.
            \item Ma sát lớn giữa phoi và mặt trước dao.
            \item Góc trước của dao thấp.
            \item Thiếu dung dịch trơn nguội hiệu quả.
        \end{itemize}
    \end{itemize}

    \subsubsection*{Phoi răng cưa / Phoi xẻp (Serrated / Segmented Chips)}
    \begin{itemize}
        \item \textbf{Đặc điểm}: Là dạng phoi bán liên tục với các vùng biến dạng thấp xen kẽ vùng biến dạng cao, có hình dạng giống lưỡi cưa.
        
        \item \textbf{Điều kiện hình thành}:
        \begin{itemize}
            \item Vật liệu có độ dẻo nhiệt thấp và độ bền giảm nhanh khi nhiệt độ tăng (hiện tượng mềm hóa do nhiệt).
            \item Thường gặp nhất khi gia công các hợp kim Titan.
        \end{itemize}
    \end{itemize}

    \subsubsection*{Phoi vụn / Phoi đứt quãng (Discontinuous Chips)}
    \begin{itemize}
        \item \textbf{Đặc điểm}: Phoi bị gãy thành từng mảnh rời rạc hoặc liên kết rất yếu.
        
        \item \textbf{Điều kiện hình thành}:
        \begin{itemize}
            \item Vật liệu phôi giòn (gang xám, đồng thau đúc).
            \item Vật liệu có chứa các tạp chất cứng.
            \item Vận tốc cắt rất thấp hoặc rất cao.
            \item Chiều sâu cắt lớn.
            \item Góc trước của dao nhỏ.
            \item Hệ thống gia công (máy, dao) có độ cứng vững thấp gây rung động.
        \end{itemize}
    \end{itemize}

        